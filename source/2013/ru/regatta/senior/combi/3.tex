\problem
Назовем \emph{ценной цифрой} числа любую из цифр, встречающихся в его
десятичной записи не большее число раз, чем каждая из остальных цифр
(это может быть и цифра, не встречающаяся в записи $n$).
Петя составляет бесконечную последовательность цифр: на $n$-е место он ставит
любую из ценных цифр числа $n$.
Может ли Петя получить последовательность, периодическую с некоторого места?
\solution
\emph{Ответ:} нет.
Пусть длина периода $T < 10^m$.
Рассмотрим число $n$, расположенное дальше предпериода и состоящее из не менее
чем $3m$ девяток, не менее чем $3m$ восьмерок, \ldots, не менее чем $3m$ двоек,
и $3m$ нулей на конце.
У всех чисел от $n$ до  $n + T$ цифра 1 встречается реже остальных.
Следовательно, весь период состоит из единиц.
Но для номеров, состоящих только из единиц, все элементы последовательности
будут не единицами.
Противоречие.
\endproblem
% $problem-source: А.\,Шаповалов
