На плоскости отмечено $2011$ точек.
Назовем пару отмеченных точек $A$ и $B$ \emph{изолированной}, если все остальные
точки находятся строго вне круга, построенного на $AB$ как на диаметре.
Какое наибольшее количество изолированных пар может быть?

\solution
\label{solution:2011/pers-geomt/senior/4}%
\emph{Ответ:}
$3 \cdot 2011 - 8 = 6025$.
\emph{Оценка.}
Обозначим $2011$ за $n$.
Соединим изолированные пары точек отрезками и будем их рассматривать как ребра
графа с вершинами в наших точках.
Заметим, что никакие два ребра не имеют общих точек, кроме концов:
это сразу следует из того, что один из углов выпуклого четырехугольника не
острый.
То же верно и для ребер, лежащих на одной прямой~--- поэтому наш граф
оказывается плоским.
Более того, он остается плоским, если добавить к нему все стороны выпуклой
оболочки наших точек.
Более того, если выпуклая оболочка есть $k$-угольник (возможно, вырожденный),
можно добавить еще $k - 3$ ребра, триангулируя внешнюю грань.
Докажем, что в любом случае мы добавили не менее 2 ребер.
Если $k \geq 5$, это уже объяснено.
Если $k = 4$, то мы добавляем одно внешнее ребро и еще хотя бы одну из сторон
выпуклой оболочки
(в самом деле, если они все изолированные, то внутри не может быть ни одной
точки, но $n > 4$).
Наконец, если $k = 3$, то мы добавляем хотя бы две стороны выпуклой
оболочки~--- иначе, опять же, внутри не может быть точек.
Итак, мы добавили хотя бы два ребра и граф по-прежнему является плоским.
Значит, количество его ребер не превосходит $3 n - 6$, поэтому изолированных
пар не больше, чем $3 n - 8$.
\\
\begin{figure}
\centering
    \jeolmfigure[width=0.5\textwidth]{4-solution}
\caption{к задаче \ref{solution:2011/pers-geomt/senior/4}, $k = 2$.}
\label{fig:solution:2011/pers-geomt/senior/4}
\end{figure}%
\emph{Пример.}
Теперь объясним, как строить пример с $3 n - 8$ изолированными парами для
$n = 5 k + 1$.
См. рис. \ref{fig:solution:2011/pers-geomt/senior/4}.
Для $k = 1$ возьмем вершины правильного пятиугольника и его центр.
Если имеется пример для $k$ и выпуклая оболочка точек в этом примере~---
правильный пятиугольник, то мы можем построить пример для $k + 1$:
добавим пять точек на серединных перпендикулярах к его сторонам (точнее, на
лучах, построенных вовне пятиугольника) на равном и очень большом расстоянии от
центра.
Несложно убедиться, что старые изолированные пары останутся изолированными и
появится 15 новых изолированных пар.
При $k = 402$ получаем требуемое.

