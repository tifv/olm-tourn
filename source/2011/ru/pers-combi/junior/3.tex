\problem
На доске написаны $8$ чисел.
Двое ходят по очереди.
За один ход нужно заменить два различных числа двумя равными с такой же суммой.
Если в какой-то момент можно будет разбить все числа на две четверки с равными
суммами, то первый выиграет.
Может ли второй ему помешать?

\solution
\emph{Ответ:} не может.
Назовем число \emph{хорошим}, если оно входит в группу из двух или более равных
чисел.
Покажем, что каждым ходом первый сможет увеличивать число хороших чисел по
сравнению со свои предыдущим ходом, или оставлять его прежним, но увеличивать
размер наибольшей группы.
Список размеров таких групп назовем типом позиции.
После хода второго первый легко делает тип $(2, 2)$.
Если второй соединит два числа из пар, второй соединит два других числа из этих
пар и получит тип $(4)$.
При любом другом ходе второго первый легко получит $(2, 2, 2)$.
Из типа $(4)$ второй делает $(4, 2)$ или $(3, 2)$.
Из $(3, 2)$ первый делает $(2, 2, 2)$.
В позиции $(2, 2, 2)$ или $(4, 2)$ первый угрожает сделать из чисел вне групп
еще одну пару, и раскидав пары по четверкам, сделать равные суммы.
Если второй соединит элементы групп, то суммы в этих четверках он не меняет.
Если же он соединит элемент группы с элементом вне группы, то первый соединит
второй элемент пары с другим элементов вне групп и получит $(2, 2, 2, 2)$.

\endproblem
