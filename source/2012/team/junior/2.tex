\textsf{(2)}
У Саши есть 5 кульков с конфетами.
Выбирая всевозможными способами пару кульков и подсчитывая суммарное число
конфет в них, Саша заметил, что суммы принимают только три значения:
53, 66 и 79.
Сколько конфет в каждом кульке?

\solution
Для каждого кулька выпишем число конфет в нем.
Среди этих чисел есть одинаковые
(иначе, добавляя одно к остальным, получили бы четыре разных суммы).
Сумма двух одинаковых~--- число четное, то есть 66.
Значит, все одинаковые числа равны 33.
Теперь ясно, что все нечетные числа равны 33
(иначе вместе с 33 не получим 66), а каждое четное равно
либо $53 - 33 = 20$, либо $79 - 33 = 46$, причем оба эти случая реализуются.
\emph{Ответ:} 20, 33, 33, 33, 46.

