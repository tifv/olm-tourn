\problem
Дан правильный $777$-угольник.
Петя и~Вася по~очереди проводят его диагонали, начинает Петя.
Диагонали не~должны пересекаться внутри многоугольника.
Проигрывает тот, после чьего хода диагонали разобьют 777-угольник на~части,
среди которых есть четырехугольная.
Кто из~игроков выигрывает, как~бы ни~играл соперник?
\solution
\emph{Ответ:} выигрывает Вася.
\par
Своим ходом Петя разбивает 777-угольник на~$m$-угольник и~$n$-угольник,
где $m < n$.
В~ответ Вася должен отсечь от~$n$-угольника ещё один $m$-угольник.
Оставшаяся часть будет $(781 - 2 m)$-угольником, то~есть, нечетноугольником,
а~значит, не~четырехугольником.
Далее Вася ответными ходами поддерживает ситуацию с~одной или несколькими
парами многоугольников с~одинаковым числом сторон и~отдельным
нечетноугольником.
А~именно, при ходе Пети в~многоугольник пары, Вася отвечает аналогичным ходом
в~другой многоугольник пары.
А~при ходе Пети в~нечетноугольник Вася в~ответ отсекает второй многоугольник
для пары.
Так как у~Васи всегда есть ход, он~не~проигрывает.
А~так как игра заканчивается (число диагоналей в~777-угольнике конечно!),
Вася выигрывает.
\endproblem
% $problem-source: Rioplatense Olympiad 2013, Level 3% Problem 4
