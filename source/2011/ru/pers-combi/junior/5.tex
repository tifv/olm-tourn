\problem
В некоторых клетках бесконечной клетчатой полоски стоят черные фишки, в
некоторых белые, а в остальных не стоит ничего.
Всего фишек конечное количество, в каждой клетке не более одной.
Разрешается производить одну из следующих операций:
\\\emph{(1)}
если подряд идущие (в таком порядке) клетки $A$, $B$, $C$ таковы, что в $A$ и в
$C$ стоят фишки разного цвета, то в клетку $B$ можно поставить фишку любого
цвета, если эта клетка пуста, или убрать оттуда фишку, если там стоит фишка;
\\\emph{(2)}
если подряд идущие (в таком порядке) клетки $A$, $B$, $C$, $D$ заполнены
фишками, причем фишки в клетках $A$ и $D$ одинаковые, то можно поменять местами
фишки в клетках $B$ и $C$.
\\Докажите, что такими операциями нельзя поменять цвет фишки, стоящей между
двумя одноцветными фишками
(при этом сохранить общее количество фишек, и в каждой из остальных клеток, в
которой изначально стояла фишка, оставить фишку того же цвета).
\solution
Поставим в соответствие пустой клетке число $0$, клетке с черной фишкой число
$-1$ и клеткам с белой фишкой число $1$.
Тогда при описанных операциях не будет меняться сумма попарных произведений
чисел в соседних клетках.
Однако при требуемой замене эта сумма поменяется, поэтому такая замена
невозможна.
\par
\emph{Другое решение.}
Рассмотрим последовательность черных и белых фишек (опустим все пустые поля).
Рассмотрим количество групп рядом стоящих черных фишек.
Оно не меняется при описанных операциях, а при требуемой замене изменится.
\endproblem
