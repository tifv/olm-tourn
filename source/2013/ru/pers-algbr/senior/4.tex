\problem
Найдется ли последовательность из 1000000 натуральных чисел, взаимно простых с
10, в которой каждое следующее делится на предыдущее, но имеет меньшую сумму
цифр?
\solution
\emph{Ответ:} найдется.
\par
Будем строить такую строку индукцией по длине.
Пусть у нас есть нужная строка из $m$ чисел, построим такую строку из $m + 1$
числа.
Пусть в последнем числе $n$ цифр, и $N > n$.
Если умножить числа на $(10^N + 1)$, мы удвоим все суммы цифр и не нарушим
делимости.
Положим $N = 6 n$, тогда
\(
    10^{6 n} + 1
=
    (10^{2n} + 1) (10^{4n} - 10^{2n} + 1)
=
    (1\ldots01) \cdot (9\ldots90\ldots01)
\).
Сумма цифр последнего множителя равна $18 n + 1$, что больше суммы цифр любого
из умноженных чисел.
Удлинив умноженную строку влево этим множителем, получим требуемую строку.
\endproblem
% $problem-source: А.\,Шаповалов
