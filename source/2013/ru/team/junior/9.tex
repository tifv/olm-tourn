\problem\problemscore{11}
Можно ли раскрасить точки плоскости в 2013 цветов
(то есть сопоставить каждой точке плоскости натуральное число от 1 до 2013)
так, чтобы на любой прямой и на любой окружности (ненулевого радиуса)
встречались бы точки всех цветов?
\solution
\emph{Ответ:} можно.
\par
Сперва покажем, как покрасить в 2013 цветов точки прямой так, чтобы на любом
отрезке были точки всех 2013 цветов.
Для этого, например, пронумеруем простые числа $p_k$, и несократимые дроби вида
$\frac{a}{p_k}$ покрасим в цвет, равный остатку $k$ по модулю 2013.
Остальные точки (в частности, все иррациональные) покрасим как угодно.
На любом отрезке найдутся точки вида $\frac{a}{p}$ ($a$ и $p$ взаимно просты)
при достаточно больших $p$, что и требуется.
\par
Теперь введем на плоскости систему координат и покрасим параболу вида
$y = x^2 + a$ в цвет, в который точка $a$ была покрашена на прямой.
Заметим, что все вертикальные (с уравнением $x = d$) прямые пересекают все
такие параболы.
Рассмотрим прямые вида $y = k x + b$.
Параболы, которые пересекают такие прямые,~--- это параболы с таким $a$, что
уравнение $x^2 - k x - b = -a$ имеет решение.
Это верно для чисел $a$ из некоторого луча (на числовой прямой), то есть
на таких прямых есть точки любого цвета.
Теперь рассмотрим произвольную окружность.
Выберем внутри нее произвольный вертикальный отрезок и заметим, что парабола,
пересекающая выбранный вертикальный отрезок обязательно пересекает и
окружность.
Несложно понять, что множество $a$ таких, что парабола $y = x^2 + a$ пересекает
некоторый вертикальный отрезок~--- это тоже отрезок (на числовой прямой),
причем такой же длины.
Отсюда ясно, что среди таких парабол встретятся параболы всех цветов.
\endproblem
% $problem-source: Жюри
