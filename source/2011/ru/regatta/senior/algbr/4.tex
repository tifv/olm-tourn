\problem
Решите в простых числах уравнение $p^2 + p q + q^2 = r^2$.

\solution
\emph{Ответ:} $p = 3$, $q = 5$, $r = 7$ или $p = 5$, $q = 3$, $r = 7$.
Преобразуем это выражение к виду $(p + q)^2 - r^2 = p q$.
Отсюда ясно, что $(p + q + r) (p + q - r) = p q$.
Так как $p + q + r > p$ и $p + q + r > q$, то мы получаем, что $p + q + r = p q$
и $p + q - r = 1$.
Складывая имеем: $2 p + 2 q = p q + 1$, откуда $(p - 2) (q - 2) = 3$, откуда
$p = 5$, $q = 3$ или наоборот.
В обоих случаях имеем $r = 7$.

\endproblem
