% nospell begin
\def\ifsolutiondefined{%
    \csname ifsolution:2014/ru/team/junior/6\endcsname}%
\expandafter\providecommand
    \csname ifsolution:2014/ru/team/junior/6\endcsname
    {\iffalse}%
\def\definesolution{%
    \expandafter\gdef
    \csname ifsolution:2014/ru/team/junior/6\endcsname
    {\iftrue}}%
% nospell end

\begingroup % figure
\ifsolution\def\solutioninclude#1{#1}\else\def\solutioninclude#1{}\fi
\solutioninclude{\ifsolutiondefined\else
\begin{figure}\centering
    \jeolmfigure[width=0.5\textwidth]{solution}
    \caption{к решению задачи \ref{2014/ru/team/senior/3:solution}}
    \label{2014/ru/team/senior/3:solution:fig}
\end{figure}%
\fi}%
\endgroup % figure

\problem\problemscore{6}
% duplicate of [2014/ru/team/junior/6]
Точка~$I_b$~--- центр вневписанной окружности треугольника $ABC$, касающейся
стороны~$AC$.
Другая вневписанная окружность касается стороны~$AB$ в~точке~$C_1$.
Докажите, что точки $B$, $C$, $C_1$ и~середина отрезка~$B I_b$ лежат на~одной
окружности.
\solution
\label{2014/ru/team/senior/3:solution}%
\ifsolutiondefined
См.~решение задачи \ref{2014/ru/team/junior/6:solution}.
\else
\definesolution
Рис.~\ref{2014/ru/team/senior/3:solution:fig}.
Пусть $M$~--- середина $B I_b$.
Опустим из~точки~$M$ перпендикуляры $MA'$ и~$MC'$ на~стороны $BC$ и~$BA$.
Так как $M$ лежит на~биссектрисе угла $B$, то~$MA' = MC'$.
Опустим из~точки~$I_b$ перпендикуляры $I_b A_2$ и~$I_b C_2$ на~стороны $BC$
и~$BA$.
Тогда $2 BA' = B A_2 = B C_2 = 2 B C' = p$, где $2 p = a + b + c$.
Это означает, что
\(
    C_1 C'
=
    \lvert BC' - B C_1 \rvert
=
    \lvert p / 2 - (p - a) \rvert
=
    \lvert a - p / 2 \rvert
\)
и~$CA' = \lvert a - p / 2 \rvert$, то~есть, что $C_1 C' = C A'$.
Следовательно, треугольник $C_1 C' M$ равен треугольнику $CA'M$ по~двум
сторонам и~углу между ними.
Тогда
\(
    \angle B C_1 M + \angle BCM
=
    180^\circ - \angle M C_1 C' + \angle BCM
=
    180^\circ
\),
что означает, что точки $B$, $C$, $C_1$ и~$M$ лежат на~одной окружности.
\fi % \ifsolutiondefined
\endproblem
% $problem-source: по мотивам олимпиады матмеха СПбГУ 2014
