\problem
На клетчатой доске $100 \times 100$ стоят 77 коней, 33 ладьи и ферзь.
Известно, что каждая фигура бьет ровно одну другую и побита ровно одной другой.
Докажите, что ферзь бьет свою фигуру по диагонали.

\solution
Выбросим пары бьющих друг друга фигур одинакового названия.
Разобьем все остальные фигуры на циклы~--- группы фигур, бьющих друг друга
по~кругу.
В каком-то цикле нечетное число фигур.
Если ферзя в нём нет, то в цикле чередуются ладьи и кони, и цикл~--- четный.
Если ферзь есть, но бьет не по диагонали, его можно заменить на ладью, и опять
получим четный цикл. 
\endproblem
% $problem-source: А.\,Шаповалов
