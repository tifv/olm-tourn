\problem
Четырехугольник $ABCD$ вписан в окружность.
Известно, что $AB = AC$ и $BC = CD$.
Диагонали четырехугольника $ABCD$ пересекаются в точке $O$, а точка $X$~---
середина дуги $CD$, не содержащей точку $A$. Докажите, что $XO \perp AB$.

\solution
[{\begin{figure}
\centering
    \jeolmfigure[width=0.5\textwidth]{../4-solution}
\caption{к задаче \ref{solution:2012/regatta/junior/geomt/4}}
\label{fig:solution:2012/regatta/junior/geomt/4}
\end{figure}}]%
\label{solution:2012/regatta/junior/geomt/4}%
См.\,рис.\,\ref{fig:solution:2012/regatta/junior/geomt/4}.
Докажем, что $O$~--- точка пересечения высот треугольника $ABX$.
Для этого достаточно доказать, что $O$ лежит на высотах, проведённых из вершин
$A$ и $B$.
Обозначим половину угла $CAB$ через $\alpha$.
Тогда
$\angle CAB = \angle CDB = \angle DBC = \angle DAC = 2 \alpha$,
так как эти углы опираются на одну и ту же дугу, а
$\angle BCA = \angle CBA = 90^\circ - \alpha$,
так как треугольник $ABC$ равнобедренный.
Из определения точки $X$ следует, что
$\angle DBX = \angle XBC = \angle XAC = \angle DAX = \alpha$.
Так как
\(
    \angle ACB + \angle XBC
=
    90^\circ - \alpha + \alpha
=
    90^\circ
\)
и
\(
    \angle ADB + \angle DAX
=
    \angle ACB + \angle DAX
=
    90^\circ - \alpha + \alpha
=
    90^\circ
\),
то углы между хордами в парах $AC$, $BX$ и $BD$, $AX$ прямые.
Получаем, что $AC$ и $BD$~--- высоты в треугольнике $ABX$.
Следовательно, $XO$~--- также высота.

\endproblem
