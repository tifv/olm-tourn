\problem
Найдите все действительные $\alpha$, для которых система уравнений
\[
    \frac{a^3}{b + c + \alpha}
=
    \frac{b^3}{c + a + \alpha}
=
    \frac{c^3}{a + b + \alpha}
\]
имеет решение в различных действительных $a$, $b$, $c$ из $[-1; 1]$.
\solution
\emph{Ответ:} $\alpha \in (-1/3, 0) \cap (0, 1/3)$.
\par
Положим $\alpha + a + b + c = s$, $a^3 / (b + c + \alpha) = p$.
Ясно, что $p \ne 0$, иначе было бы $a = b = c = 0$.
Так что $a b c \ne 0$.
Числа $a$, $b$, $c$ суть корни многочлена \[f(x)=x^3+px-ps=0.\]
По теореме Виета $a + b + c = 0$, так что
$\alpha = s = (p s) / p = a b c / (a b + b c + a c)$,
$\alpha \ne 0$
и
$\alpha^{-1} = 1 / a + 1 / b + 1 / c$.
Если, например, $a, b > 0$, $c = -a - b < 0$, то
$a + b \leq 1$ и $\alpha^{-1}$
может принимать любые значения, какие принимает
$1 / a + 1 / b - 1 / (a + b)$
при этих условиях
(достаточно взять $p = a b + b c + a c$, ведь теорема Виета работает в обе
стороны).
Записывая это выражение в виде $1 / a + a / (b a + b^2)$ видим, что оно убывает
по $b$, так что минимального значения принимает при $b = 1 - a$, когда равно
$1 / a b - 1 = 1 / (a - a^2) - 1 \geq 3$, равенство при $a = b = 1 / 2$.
Но равные значения $a$ и $b$ брать нельзя, так что $\alpha^{-1}$ принимает все
значения, большие $3$.
Аналогично, когда $a, b < 0$, получаем все значения, меньшие $-3$.
\endproblem
