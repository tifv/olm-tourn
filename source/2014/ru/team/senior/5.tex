\problem\problemscore{7}
Сферы $S_1$, $S_2$ и~$S_3$ касаются друг друга внешним образом и~касаются
некоторой плоскости в~точках~$A$, $B$ и~$C$.
Сфера~$S$ касается сфер $S_1$, $S_2$ и~$S_3$ внешним образом и~касается данной
плоскости в~точке~$D$.
Докажите, что проекции точки~$D$ на~стороны треугольника $ABC$ являются
вершинами правильного треугольника.
\solution
\label{2014/ru/team/senior/5:solution}%
Обозначим через $r_1$, $r_2$, $r_3$ и~$r$ радиусы сфер
$S_1$, $S_2$, $S_3$ и~$S$, соответственно.
Как легко видеть, $AB = 2 \sqrt{r_1 r_2}$, $BC = 2 \sqrt{r_2 r_3}$,
$AD = 2 \sqrt{r_1 r}$, $DC = 2 \sqrt{r_3 r}$.
%Рис.~\ref{2014/ru/team/senior/5:solution:fig}.
Значит, $AB / CB = AD / CD$.
Обозначим проекции точки~$D$ на~стороны $AB$, $BC$ и~$CA$ через $X$, $Y$ и~$Z$.
Заметим, четырехугольник $AXDZ$~--- вписанный, а~$AD$~--- диаметр описанной
вокруг $AXDZ$ окружности.
Тогда по~теореме синусов $XZ = AD \cdot \sin(\angle A)$.
Аналогично, $YZ = CD \cdot \sin(\angle{C})$.
Поэтому
\[
    \frac{XZ}{YZ}
=
    \frac{
        AD \cdot \sin(\angle A)
    }{
        CD \cdot \sin(\angle C)
    }
=
    \frac{
        AB \cdot \sin(\angle A)
    }{
        CB \cdot \sin(\angle C)
    }
=
    1
\, , \]
где последнее равенство следует из~теоремы синусов для треугольника $ABC$.
Итак, $XZ = YZ$.
Аналогично $YZ = YX$.
\endproblem
% $problem-source: фольклор
