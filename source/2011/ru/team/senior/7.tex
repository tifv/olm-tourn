\problem\problemscore{8}
Найдите все функции $f \colon [0, +\infty) \to \mathbb{R}$ такие, что
$f(x + f(x) + 2 y) = f(2 x) + 2 f(y)$
при всех неотрицательных $x$, $y$ и уравнение $f(x) = 0$ имеет конечное
(возможно, нулевое) количество решений.
\solution
\emph{Ответ:} $f(x) = x$.
\par
Положим $c(x) = f(x) - x$.
Заметим, что если $c(x) < 0$, то подставляя $y = -c(x) / 2$ получаем
$f(y) = 0$.
Таким образом, функция $c$ принимает конечное число отрицательных значений.
Переписывая уравнения в терминах $c(\cdot)$ получаем
$c(x) + c(2 x + 2 y + c(x)) = c(2 x) + 2 c(y)$.
Если функция $c$ принимает конечное, но ненулевое множество отрицательных
значений, то рассмотрим наименьшее из них $-M = f(a)$ и подставим
$y = 2 x = a$.
Получим $- 3 M = c(x) + c(2 x + 2 y + c(x)) \geq -2 M$, противоречие.
Таким образом, $c(x) \geq 0$ при всех $x$.
При $x = 0$ получаем $c(2 y + A) = 2 c(y)$, где $A = c(0)$.
Зафиксируем $x \geq A / 2$ и выберем произвольное $z \geq A$, тогда полагая
$y = (z - A) / 2$ получим $c(z + (2 x + c(x) - A)) = c(z) + (c(2 x) - c(x))$.
Обозначим $c(2 x) - c(x) = V$, $2 x + c(x) - A = U \geq 0$.
Тогда $c(z + U) = c(z) + V$ при $z \geq A$, откуда $c(z + n U) = c(z) + n V$
при натуральных $n$, и получаем $V \geq 0$
(так как функция $c$ не принимает отрицательных значений).
Наша цель доказать, что $V = 0$.
При $U = 0$ это понятно.
Пусть $U > 0$.
Положим $t_n = z + n U$, тогда $c(t_n) = \lambda t_n + \alpha$, где
$\lambda = V / U$, $\alpha = c(z) - \lambda z$.
Для всех достаточно больших $t$ выберем максимальное $n$ такое, что
$t - n U \geq A$, тогда можно будет положить $z = t - n U$ и мы получим
\(
    c(t)
=
    c(z + n U)
\geq
    n V
=
    \lambda n U
\geq
    \lambda (t - A - u)
=
    \lambda t + K
\)
для некоторой константы $K$.
Теперь зафиксируем $y = 0$ и положим $2 x = t_n$ при большом $n$.
Получим в правой части $\lambda t_n + \alpha + c(A)$, а в левой не меньше, чем
$(\lambda / 2 + \lambda + \lambda^2) t_n + \mathrm{const}$.
Такое бывает при всех $n$ только когда $\lambda = 0$, то есть $V = 0$, то есть
$c(2 x) = c(x)$ при $x \geq A / 2$.
Отсюда
$c(A) = c(2 A) = c(2 \cdot A / 2 + A) = 2 c(A / 2) = 2 c(A)$,
то есть $c(A) = 0$.
Тогда $A = 2 c(0) = c(2 \cdot 0 + A) = c(A) = 0$.
Итак, $c(2 y) = 2 c(y)$ при всех $y$ и $c(2 y) = c(y)$ при $y \geq A / 2$.
Отсюда легко следует, что $c(y) = 0$ при всех $y$, что и требовалось доказать.
\endproblem
