% nospell begin
\def\ifsolutiondefined{%
    \csname ifsolution:2012/ru/pers-algbr/junior/2\endcsname}%
\expandafter\providecommand
    \csname ifsolution:2012/ru/pers-algbr/junior/2\endcsname
    {\iffalse}%
\def\definesolution{%
    \expandafter\gdef
    \csname ifsolution:2012/ru/pers-algbr/junior/2\endcsname
    {\iftrue}}%
% nospell end

\problem
% duplicated in [2012/ru/regatta/senior/algbr/4]
Последовательность $\{ a_n \}$ определена рекуррентно:
\[
    a_1 = \frac{1}{2}
,\qquad
    a_n = \frac{a_{n - 1}}{2 n \cdot a_{n - 1} + 1}
    \quad\text{при $n > 1$}
.\]
Найдите сумму $a_1 + a_2 + \ldots + a_{2012}$.
\solution
\label{2012/ru/pers-algbr/junior/2:solution}%
\emph{Ответ:} $2012 / 2013$.
\par
\ifsolutiondefined
См.~решение задачи \ref{2012/ru/regatta/senior/algbr/4:solution}.
\else
\definesolution
Введем в рассмотрение последовательность $b_n = 1 / a_n$.
Для неё рекуррентное соотношение переписывается в виде $b_n = b_{n - 1} + 2 n$.
Получаем
$b_n = 2 + 4 + \ldots + 2 n = n (n + 1)$,
откуда
\[a_n = \frac{1}{n (n + 1)} = \frac{1}{n} - \frac{1}{n + 1}.\]
Складывая такие числа по всем $n$ от 1 до 2012, получим, что сумма равна
$2012 / 2013$.
\fi % \ifsolutiondefined
\endproblem
