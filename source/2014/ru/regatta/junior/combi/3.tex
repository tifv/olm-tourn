\problem
Несколько ладей побили все белые клетки шахматной доски $40 \times 40$.
Какое наибольшее число черных клеток могли остаться не~побитыми?
(Ладья бьет клетку, на~которой стоит.)
\solution
\emph{Ответ:} $400$.
\par
\emph{Оценка.}
Если есть не~побитая клетка, то~на~проходящих через неё рядах (вертикали
и~горизонтали) нет ладей.
Чтобы побить все клетки одного цвета на~таком ряду, надо поставить ладьи как
минимум на~$20$ перпендикулярных рядах.
Таким образом, свободны от~ладей не~более чем $20$~рядов по~каждому
из~направлений.
Непобитые клетки стоят на~пересечении таких рядов, поэтому их~не~более $20^2$.
\par
\emph{Пример.}
Можно считать, что левая нижняя клетка~--- черная.
Расставим по~$20$ ладей на~белых клетках левого и~нижнего краев.
Тогда на~пересечении непобитых рядов все клетки~--- черные, их~$400$, а~все
белые~--- побиты.
\endproblem
% $problem-source: А.\,Шаповалов
