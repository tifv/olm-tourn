\textsf{(6)}
Сколько решений в натуральных числах при данном натуральном $n$ имеет уравнение
$3 x^2 + 5 y^2 = 2^n$?

\solution
\emph{Ответ:}
ни одного при $n = 1$ и при четном $n$ и $k$ для $n = 2 k + 1$ при $k \geq 1$.
\emph{Решение.}
При четных $n$ нет решений по модулю 3.
Пусть $n = 2 k + 1$, $k \geq 1$.
Докажем, что существует и единственно нечетное решение
(то есть такое, что $x$ нечетно)~--- этого будет достаточно, поскольку каждое
решение сокращается до единственного нечетного при каком-то меньшем $k$.
Заметим, что нечетных решений уравнения $3 x^2 + 5 y^2 = M$ столько же, сколько
нечетных решений уравнения $w^2 + 15 y^2 = 3 M$ (надо взять $w = 3 x$).
Докажем, что при нечетном $n \geq 3$ уравнение $w^2 + 15 y^2 = 3 \cdot 2^n$
имеет единственное решение.
Индукция по $n$.
База $n = 3$ понятна.
Пусть $n \geq 5$ и для $n - 2$ вместо $n$ существование и единственность
нечетного решения проверены.
Сначала проверим существование для $n$.
Пусть $a^2 + 15 b^2 = 3 \cdot 2^{n - 2}$ с нечетными $a, b$.
Тогда
\(
    (a \pm 15 b)^2 + 15(a \mp b)^2
=
    16(a^2 + 15 b^2)
=
    3 \cdot 2^{n + 2}
\).
Так что в качестве решения можно попробовать взять
$(\frac{a + 15 b}{2}, \frac{a - b}{2})$
или
$(\frac{a - 15 b}{2}, \frac{a + b}{2})$.
В одной из пар оба числа нечетны, её и возьмем.
Теперь докажем единственность.
Покажем, что любое нечетное решение $(c, d)$ уравнения
$c^2 + 15 d^2 = 3 \cdot 2^n$ получено указанным выше способом.
То есть найдутся нечетные (не обязательные) $a$, $b$ такие, что
$\pm c = \frac{a + 15 b}{2}, d = \frac{a - b}{2}$.
Заметим, что $(c - d)(c + d) = c^2 - d^2 = 3 \cdot 2^n - 16 d^2$ делится на 16,
но не на 32, так что одно из чисел $c - d, c + d$ делится на 8, но не на 16.
Если это $c - d$ (иначе заменим $c$ на $-c$), то берем $b = (c - d) / 8$,
$a = b + 2 b = (c + 15 d) / 8$.

