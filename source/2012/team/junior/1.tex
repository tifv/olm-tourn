\problem
\problemmark{2}
Три жулика, каждый с двумя чемоданами, хотят переправиться через реку.
Есть трехместная лодка, каждое место в которой может быть занято человеком или
чемоданом.
Никто из жуликов не доверит свой чемодан спутникам в свое отсутствие, но готов
оставить чемоданы на безлюдном берегу.
Смогут ли они переправиться?
(Лодку, приставшую к берегу, считаем частью берега.)

\solution
\emph{Ответ:} смогут.
\emph{Пример.}
Обозначим жуликов буквами $A$, $B$, $C$.
Сначала $C$ перевозит свои чемоданы, затем он (без багажа) возвращается обратно
и перевозит $A$ и $B$ (без багажа).
После этого $A$ и $B$ возвращаются, и $A$ перевозит свои чемоданы.
Наконец $A$ и $C$ возвращаются и перевозят $B$, который возвращается один за
своими чемоданами.

\endproblem
