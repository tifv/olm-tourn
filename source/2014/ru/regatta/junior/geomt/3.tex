\ifsolution
\begin{figure}\centering
    \jeolmfigure[width=0.5\textwidth]{solution}
    \caption{к задаче \ref{2014/ru/regatta/junior/geomt/3:solution}}
    \label{2014/ru/regatta/junior/geomt/3:solution:fig}
\end{figure}%
\fi % \ifsolution

\problem
В~выпуклом четырехугольнике $ABCD$ биссектрисы углов $A$ и~$C$ параллельны
и~пересекают диагональ~$BD$ в~двух различных точках $P$ и~$Q$, при этом
$BP = DQ$.
Докажите, что четырехугольник~$ABCD$~--- параллелограмм.
\solution
\label{2014/ru/regatta/junior/geomt/3:solution}%
Пусть $AP$ и~$CQ$~--- отрезки биссектрис, и~$AP > CQ$.
Продлим отрезок $QC$ за~точку $C$ так, чтобы полученный отрезок $QE$ был
равен $AP$
(рис.~\ref{2014/ru/regatta/junior/geomt/3:solution:fig}).
Тогда треугольники $APD$ и~$QBE$ равны по~двум сторонам и~углу между ними.
Поэтому $\angle PAD = \angle BEQ$, $AD = BE$ и~аналогично
$\angle PAB = \angle DEQ$, $AB = DE$.
Значит, $BADE$ параллелограмм.
Если точки $C$ и~$E$ не~совпадают, то~треугольники $BEC$ и~$CED$ равны
по~стороне и~двум углам.
Тогда треугольник $BCD$ равнобедренный и~$Q$~--- середина стороны $BD$.
Но~тогда точка $P$ должна совпадать с~точкой $Q$, что противоречит условию.
\endproblem
% $problem-source: А.\,Шаповалов
