\problem
На плоскости отмечено $2011$ точек.
Назовем пару отмеченных точек $A$ и $B$ \emph{изолированной}, если все остальные
точки находятся строго вне круга, построенного на $AB$ как на диаметре.
Какое наименьшее количество изолированных пар может быть?
\solution
\emph{Ответ:} $2010$.
Соединим точки $A$ и $B$ отрезком, если пара $AB$ является изолированной, и
рассмотрим получившийся граф.
Заметим, что он связный.
Действительно, предположим, что он распадается на ряд компонент связности;
тогда найдем пару точек из разных компонент с наименьшим расстоянием между
ними.
Несложно видеть, что такая пара является изолированной, так любая точка,
попавшая в их круг, ближе к каждой из них, чем они друг к другу, но не может
лежать сразу с обеими в одной компоненте.
Противоречие.
В связном графе на $2011$ вершинах не менее $2010$ ребер, тем самым мы
показали, что изолированных пар не менее $2010$.
Примером являются $2011$ точек, лежащих на одной прямой (или на
полуокружности).
\endproblem
