\problem\problemscore{7}
В~круге проведены несколько хорд так, что любые две из~них пересекаются внутри
круга.
Докажите, что можно пересечь все хорды одним диаметром.
\solution
На~каждой из~двух дуг, стягивающих концы хорды, лежит по~одному концу каждой
другой хорды (иначе~бы хорды не~пересекались).
Поэтому, если пронумеровать концы хорд в~порядке обхода окружности, номера
концов каждой хорды будут отличаться ровно на~100.
Проведем произвольный диаметр, не~проходящий через концы хорд.
Если вращать его непрерывно, то~при переходе через любой конец число концов
справа от~него будет меняться не~более, чем на~1.
Если вначале число концов справа от~него больше 100, то~при повороте
на~$180^\circ$ оно станет меньше 100.
Ввиду дискретной непрерывности в~какой-то момент это число было равным 100.
Значит, в~этот момент концы каждой хорды были по~разные стороны от~диаметра,
и~он~пересекал все хорды.
\endproblem
% $problem-source: А.\,Шаповалов
