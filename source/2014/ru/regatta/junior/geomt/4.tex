\ifsolution
\begin{figure}\centering
    \jeolmfigure[width=0.5\textwidth]{solution}
    \caption{к задаче \ref{2014/ru/regatta/junior/geomt/4:solution}}
    \label{2014/ru/regatta/junior/geomt/4:solution:fig}
\end{figure}%
\fi % \ifsolution

\problem
Бумажный прямоугольник $ABCD$ ($AB = 3$, $BC = 9$) перегнули так, что
вершины $A$ и~$C$ совпали.
Какова площадь получившегося пятиугольника?
\solution
\label{2014/ru/regatta/junior/geomt/4:solution}%
\emph{Ответ:} $19.5$.
\par
Рис.~\ref{2014/ru/regatta/junior/geomt/4:solution:fig}.
Заметим, что перегнули по~серединному перпендикуляру $\ell$ к~отрезку $AC$.
Пусть $M$~--- точка пересечения прямых $\ell$ и~$BC$, $N$~--- середина $AC$,
а~из~точки $D$ получилась точка $D'$.
Несложно заметить, что $BM = D'K = 4$, $AM = MC = AK = 5$,
$AN = 3 \sqrt{10} / 2$ и~$MN = NK = \sqrt{10} / 2$.
Тогда $S(ABMKD') = 3 \cdot 4 + \sqrt{10} / 2 \cdot 3 \sqrt{10} / 2 = 19.5$.
\endproblem
% $problem-source: дополнительные задачи Кенгуру 2014
