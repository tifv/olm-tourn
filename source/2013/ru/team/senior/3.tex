\problem\problemscore{4}
% "\text{\,м}" -> "\text{метр}"
Есть куб со стороной $1 \text{\,м}$ и набор из пяти красок.
Сначала Петя режет куб на равные меньшие кубики размера не более
% "\text{\,см}" -> "\text{сантиметр}"
$1 \text{\,см}$ (размер он выбирает сам).
Затем Вася красит кубики как хочет (не обязательно все одинаково), но так,
чтобы каждая грань была одноцветна.
Наконец, Петя складывает куб, используя все кубики.
Докажите, что независимо от действий Васи Петя может добиться того, чтобы
каждая грань большого куба была одноцветной.
\solution
Пусть $k$~--- число способов раскрасить кубик данным набором красок.
Разрежем куб на $n^3$ кубиков, где $n > 6 k$ и достаточно велико, чтобы ребра
частей были короче 1 см.
Расставим кубики так, чтобы они переводились друг в друга параллельным
переносом.
Тогда найдется не менее $\frac{n^3}{k}$ одинаково раскрашенных кубиков.
Это больше чем $6 n^2$, что больше числа кубиков, примыкавших к граням
исходного куба.
Будем строить большой куб с такой же раскраской, как у этих кубиков.
Поместим эти одинаково окрашенные кубики на границу.
При этом кубики, примыкающие к боковым граням, получаются друг из друга
параллельным переносом, поэтому боковые грани будут одноцветны. 
\endproblem
% $problem-source: А.\,Шаповалов
