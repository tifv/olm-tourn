\problem
На доске выписано $2 n^2 + 5 n$ натуральных чисел, не обязательно различных.
Каждое число равно количеству выписанных чисел, не равных ему.
Каково наибольшее количество различных чисел среди выписанных?

\solution
\emph{Ответ:} $2 n + 1 $.
Обозначим $M = 2 n^2 + 5 n$.
\emph{Пример.}
Выпишем 1 раз число $M - 1$, 2 раза число $M - 2$, \ldots, $2 n$ раз число
$M - 2 n$, и, наконец, $4 n$ раз $M - 4 n$.
\\
\emph{Оценка.}
Разобьем выписанные числа на группы равных.
Если в группе $k$ чисел, то эти числа равны $M - k$.
Тогда для различных чисел размеры их групп различны.
Однако уже сумма $2 n + 2$ различных натуральных чисел не менее
$1 + 2 + \ldots + (2 n + 2) = (2 n + 3) (n + 1) = 2 n^2 + 5 n + 3 > M$. 
\endproblem
% $criteria: 1 балл ответ, по 2 балла за оценку и за пример.
% $problem-source: А.\,Шаповалов
