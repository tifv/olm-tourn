\problem
В стране несколько (больше 3) городов, некоторые из которых связаны дорогами.
Оказалось, что при закрытии любого города, от любого города до любого можно
добраться не заезжая в закрытый город, однако, при закрытии любой дороги это
свойство нарушается.
Докажите, что в этой стране нет трех городов, попарно связанных дорогами.

\solution
Докажем от противного: пусть такие три города $A$, $B$ и $C$ нашлись.
Кроме того, есть еще как минимум один город (по условию), то есть из одного из
городов $A$, $B$ или $C$ выходит еще как минимум одна дорога, пусть это дорога
$AD$.
Закроем город $A$, тогда от города $D$ (по условию) можно добраться до городов
$B$ и $C$, пусть путь до города $C$ не содержит дорогу $BC$.
Заметим, что мы получили таким образом, что дорогу $AC$ можно объехать по
маршруту $ABC$ и $ADC$.
Значит при закрытии любого города ребро $AC$ можно будет объехать, то есть при
его выкидывании свойство из условия не нарушается.

\endproblem
