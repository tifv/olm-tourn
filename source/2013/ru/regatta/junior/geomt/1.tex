\ifsolution
\begin{figure}\centering
    \jeolmfigure[width=0.5\textwidth]{solution}
    \caption{к решению задачи \ref{solution:2013/ru/regatta/junior/geomt/1}.}
    \label{fig:solution:2013/ru/regatta/junior/geomt/1}
\end{figure}%
\fi % \ifsolution

% $tex$packages:
% - mathabx # \wideparen

\problem
Точки $M$, $N$, $P$~--- середины сторон $AB$, $CD$ и $DA$ вписанного
четырехугольника $ABCD$.
Известно, что $\angle MPD = 150^\circ$, $\angle BCD = 140^\circ$.
Найдите угол $\angle PND$. 
\solution%
\label{solution:2013/ru/regatta/junior/geomt/1}%
\emph{Ответ:} $110^\circ$.
Рис.~\ref{fig:solution:2013/ru/regatta/junior/geomt/1}.
Средняя линия $PN$ параллельна $AC$, поэтому
$\angle PND = \angle ACD = \frac{1}{2} \wideparen{AD}$.
С другой стороны,
$\frac{1}{2} \wideparen{AB} = \angle ADB = \angle APM = 30^\circ$
и $\frac{1}{2} \wideparen{BCD} = 180^\circ - \angle BCD = 40^\circ$.
Поскольку
$\wideparen{AD} + \wideparen{AB} + \wideparen{BCD} = 360^\circ$,
то $\angle PND = 110^\circ$.
\endproblem
% $problem-source: Д.\,Максимов
