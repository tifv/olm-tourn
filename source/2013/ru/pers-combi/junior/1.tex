\problem
В~школьной олимпиаде по~математике участвовало 60 человек, по~физике~--- 50,
по~информатике~--- 40.
Составили три списка: тех, кто участвовал ровно в одной из олимпиад, ровно
в~двух, ровно в~трех.
Во~всех списках одно и то же число людей.
Сколько человек в~каждом списке?
\solution
\emph{Ответ:} 25.
Пусть в каждом списке по $x$ человек.
Если сложить математиков, физиков и информатиков, то люди из первого списка
будут учтены по разу, из второго~--- по два раза, из третьего~--- по 3.
Получаем уравнение $x + 2 x + 3 x = 40 + 50 + 60$, откуда $x = 25$.
\endproblem
% $problem-source: А.\,Шаповалов
