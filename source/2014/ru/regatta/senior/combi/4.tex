\problem
Докажите, что число способов раскрасить ребра $n$-угольной призмы в~4 данных
цвета так, чтобы на~ребрах каждой грани встречались все цвета, не~превосходит
$(8 \cdot 6^{n-1} - 12 \cdot 2^{n-1})$.
\solution
Поставим призму на~стол и~занумеруем боковые ребра против часовой стрелки,
глядя сверху.
Любую требуемую раскраску можно получить так:
выбираем цвет для 1-го ребра (4~варианта),
раскрашиваем 3 оставшиеся ребра грани, содержащей 1-е и~2-е ребра
($3! = 6$ вариантов),
раскрашиваем 3 оставшиеся ребра грани, содержащей 2-е и~3-е ребра
(6 вариантов),
\ldots,
раскрашиваем 3 оставшиеся ребра грани, содержащей $(n-1)$-е и~$n$-е ребра
(6 вариантов),
раскрашивает 2 оставшиеся ребра грани, содержащей $n$-е и~1-е ребра
(2 варианта).
Так будет получено $4 \cdot 6^{n-1} \cdot 2$ раскрасок.
При этом ни~один требуемый вариант не~будет пропущен и~не~повторится,
но~некоторые варианты не~годятся 
(могут совпасть цвета $n$-го и~1-го рёбер; на~верхней или нижней грани может
не~найтись трёх цветов).
В~частности, сосчитаем варианты, когда все ребра нижней грани~--- одного цвета.
Выберем цвет нижней грани (4~варианта), а~далее повторяем предыдущую
конструкцию: выбираем цвет для 1-го ребра (3~варианта),
раскрашиваем 2~оставшихся ребра грани, содержащей 1-е и~2-е ребра (2~варианта),
раскрашиваем 2 оставшихся ребра грани, содержащей 2-е и~3-е ребра (2~варианта),
\ldots,
раскрашиваем 2 оставшихся ребра грани, содержащей $(n-1)$-е и~$n$-е ребра
(2~варианта),
раскрашивает оставшееся ребро грани, содержащей $n$-е и~1-е ребра
(1 вариант).
Так будет получено $4 \cdot 3 \cdot 2^{n-1}$ раскрасок.
Их~и~вычтем.
\endproblem
% $problem-source: А.\,Шаповалов
