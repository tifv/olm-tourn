\problem
В равнобедренном треугольнике $ABC$ ($AB = AC$) угол $BAC$ равен $40^\circ$.
Точки $S$ и $T$ на сторонах $AB$ и $BC$ соответственно таковы, что
$\angle BAT = \angle BCS = 10^\circ$.
Отрезки $AT$ и $CS$ пересекаются в точке $P$.
Докажите, что $BT = 2 PT$.

\solution
[{\begin{figure}
\centering
    \jeolmfigure[width=0.5\textwidth]{2-solution}
\caption{к задаче \ref{solution:2011/regatta/junior/geomt/2}.}
\label{fig:solution:2011/regatta/junior/geomt/2}
\end{figure}}]%
\label{solution:2011/regatta/junior/geomt/2}%
Заметим, что углы при основании равнобедренного треугольника $ABC$ равны
$70^\circ$.
Тогда $\angle SCA = 60^\circ$, и, значит, угол между $CS$ и $AT$ равен
$90^\circ$
(см.\,рис.\,\ref{fig:solution:2011/regatta/junior/geomt/2}).
Четырехугольник $STCA$ вписанный
(в нём углы опирающиеся на сторону $ST$ равны по условию),
а значит треугольник $STP$ тоже прямоугольный с углами $60^\circ$ и $30^\circ$.
Значит, $ST = 2 PT$.
С другой стороны, опять же из вписанности, имеем
$\angle TSB = \angle BCA = 70^\circ$ и
$\angle STB = \angle BAC = 40^\circ$,
то есть треугольник $STB$~--- равнобедренный.
Значит $BT = ST = 2 PT$, что и требовалось доказать.

\endproblem
