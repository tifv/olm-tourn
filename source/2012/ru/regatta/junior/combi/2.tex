\problem
Олимпиада по математике проводится в два дня, каждый день предлагается
одинаковое число задач, пронумерованных от $1$ до $N$.
Оказалось, что у каждого школьника количества решенных им задач в первый и
второй день отличаются на $1$.
При этом для каждого номера от $1$ до $N$ число школьников, решивших задачи с
этим номером в разные дни, отличается на $2$.
Докажите, что в олимпиаде участвовало четное число школьников.
\solution
Посчитаем число задач, решенных всеми участниками.
Заметим, что для каждого участника число решенных им за два дня задач нечетно
(так как равно $k + (k + 1)$).
Однако, каждую пару задач с одинаковыми номера решили четное число школьников
($m + (m + 2)$).
Это означает, что общее число задач, решенное всеми школьниками четно, откуда
следует, что и число участников турнира тоже четно~--- ведь каждый вкладывает
в общую сумму нечетное число задач.
\endproblem
