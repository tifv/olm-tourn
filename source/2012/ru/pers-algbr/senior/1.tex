\problem
Функция $f(x)$, определенная при всех вещественных $x$, удовлетворяет
равенствам
$f(x) = f(x + T_1)$ при всех $x > A_1$ и
$f(x) = f(x + T_2)$ при всех $x > A_2$,
где $T_1$, $T_2$~--- данные положительные числа,
$A_1$, $A_2$ --- данные вещественные числа.
Докажите, что $f(x) = f(x + T_2)$ при всех $x > A_1$.
\solution
Если $A_1 \geq A_2$, то это очевидно.
В противном случае при всех $x > A_1$ выполнено равенство:
\[
    f\left(
        x
        +
        \left\lceil
            \frac{A_2 - A_1}{T_1}
        \right\rceil
        T_1
    \right)
=
    f\left(
        x
        +
        \left\lceil
            \frac{A_2 - A_1}{T_1}
        \right\rceil
        T_1
        +
        T_2
    \right)
\]
(где за $\lceil w \rceil$ обозначено наименьшее целое число, не меньшее $w$),
так как
\[
    x
    +
    \left\lceil
        \frac{A_2 - A_1}{T_1}
    \right\rceil
    T_1
\geq
    x + A_2 - A_1
>
    A_1 + A_2 - A_1
=
    A_2
.\]
Однако
\[
    f(x) = f(x + T_1) = f(x + 2 T_1)
= \ldots =
    f\left(
        x
        +
        \left\lceil
            \frac{A_2 - A_1}{T_1}
        \right\rceil T_1
    \right)
,\]
и аналогично
\[
	f(x + T_2)
=
     f\left(
        x
        +
        \left\lceil
        \frac{A_2 - A_1}{T_1}
        \right\rceil
        T_1
        +
        T_2
     \right)
.\]
Поэтому $f(x) = f(x + T_2)$.
\endproblem
