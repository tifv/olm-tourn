Верно ли, что любой треугольник площади 3 можно покрыть выпуклым
многоугольником площади 5, имеющим ось симметрии?

\solution
Пусть нам дан треугольник $ABC$.
Не умаляя общности обозначим длины его сторон $BC$, $AC$ и $AB$ через
$a \geq b \geq c$ соответственно.
Отложим на луче $AB$ точку $D$ такую, что $AD = AC = b$.
Тогда треугольник $CAD$ равнобедренный, а следовательно, имеет ось симметрии
(серединный перпендикуляр к основанию).
Проведем из точки $C$ высоту $CH$ на сторону $AB$ и обозначим её длину через
$h$.
Тогда $S_{ABC} = \dfrac{h \cdot c}{2}$, а $S_{CAD} = \dfrac{h \cdot b}{2}$.
То есть
\(
    \dfrac{S_{CAD}}{S_{ABC}}
=
    \dfrac{h \cdot b}{2} \cdot \dfrac{2}{h \cdot c}
=
    \dfrac{b}{c}
\).
Аналогично можно построить равнобедренный треугольник с вершиной в точке $C$ и
двумя сторонами, равными $a$, и площадью, превосходящей $S_{ABC}$ в
$\dfrac{a}{b}$ раз.
Нам осталось показать, что либо $\dfrac{a}{b} \leq \dfrac{5}{3}$, либо
$\dfrac{b}{c} \leq \dfrac{5}{3}$.
Предположим противное.
Тогда $\dfrac{5}{3} b < a$ и $c < \dfrac{3}{5} b$.
Отсюда $\dfrac{5}{3} b < a < b + c < b + \dfrac{3}{5} b = \dfrac{8}{5} b$.
Получаем противоречие так, как $\dfrac{8}{5} < \dfrac{5}{3}$.

