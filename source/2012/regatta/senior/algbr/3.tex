\problem
Функция $f(x)$ такова, что
$f(1) = 0$, $f(2 n) = f(n) + 1$ и $f(2 n + 1) = f(2 n) - 1$ при $n \in \NN$.
Найдите сумму $f(1) + f(2) + \ldots + f(127)$.

\solution
Легко понять, рассуждая по индукции, что $f(n)$~--- это количество нулей в
двоичной записи числа.
Действительно, условие $f(1) = 0$~--- это база, а переход от $n$ к $2 n$ и
от $2 n$ к $2 n + 1$ осуществляется с помощью данных соотношений.
Заметим также, что $127 = 1111111_2$, то мы суммируем значения $f$ для всех
не более чем семизначных (в двоичной записи) чисел.
Посчитаем, сколько раз встретился ноль на $k$-ом месте.
На местах с $k + 1$ до 7 возможны все варианты (их $2^{7 - k}$) на местах с
первого по $k - 1$ возможны все варианты, кроме одного (все нули), их
$2^{k - 1} - 1$.
На первом месте ноль невозможен, значит надо просуммировать слагаемые вида
$(2^{k - 1} - 1) 2^{7 - k}$ по всем $k$ от 2 до 7.
Получается 321.

\endproblem
