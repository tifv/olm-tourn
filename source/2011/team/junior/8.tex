\problem
[{\begin{figure}
\centering
    \jeolmfigure[width=0.5\textwidth]{8-problem}
\caption{к задаче \ref{problem:2011/team/junior/8}.}
\label{fig:problem:2011/team/junior/8}
\end{figure}}]%
\problemscore{12}
\label{problem:2011/team/junior/8}%
В узлах правильного шестиугольника со стороной 3, разбитого на правильные
треугольники со стороной 1
(см.\,рис.\,\ref{fig:problem:2011/team/junior/8})
расставили натуральные числа от $1$ до $37$.
Будем называть треугольник \emph{хорошим}, если направление обхода его вершин от
меньшего числа к большему по часовой стрелке.
Докажите, что при любой расстановке хороших треугольников не меньше $19$.

\solution
Раскрасим треугольники в черный и белый цвет в шахматном порядке, а на всех
проведенных отрезках введем каноническую ориентацию~--- такую, которая обходит,
например, черные треугольники по часовой стрелке.
Теперь на каждом отрезке разбиения поставим стрелку от меньшего числа к
большему и сотрем все отрезки, ориентация которых не совпала с канонической.
Тогда ориентация треугольника определяется его цветом и количеством не стертых
у него сторон.
Пусть имеется $a$ и $27 - a$ белых треугольников разных ориентаций и $b$ и
$27 - b$ черных треугольников соответствующих ориентаций.
Пусть первая ориентация характеризуется двумя нестертыми ребрами у белых
треугольников и одним нестертым ребром у черного треугольника, а вторая
ориентация~--- наоборот.
Общее количество ребер у белых треугольников тогда
$2 a + (27 - a) = a + 27$, а у черных треугольников общее количество количество
ребер $b + 2 (27 - b) = 54 - b$.
Пусть оказалось, что $a + b < 19$, тогда
$(54 - b) - (a + 27) = 27 - (a + b) \geq 9$.
Но ребра, принадлежащие только треугольникам одного цвета~--- это только ребра
на границе~--- их ровно 9, то есть в этом неравенстве возможно только равенство
и то, только в случае, если ребра черных треугольников на границе все
присутствуют (то есть ориентированы канонически), а ребра белых треугольников
на границе все отсутствуют, то есть не ориентированы канонически.
Но это будет означать, что граница шестиугольника ориентирована по циклу, что
невозможно, так как числа на границе не могут только возрастать.

\endproblem
