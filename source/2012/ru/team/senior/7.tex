\problem\problemscore{7}
В стране $n$ аэропортов, некоторые из них связаны двусторонними беспосадочными
авиалиниями.
Сеть авиалиний связна:
из любого аэропорта можно добраться до любого другого.
Оказалось, что не менее чем $k$ аэропортов~--- узловые:
при закрытии любого из них связность сети авиалиний нарушается.
При данных $n$ и $k \leq n - 2$ определите наибольшее возможное число авиалиний
в стране.

\solution
\emph{Ответ:} $(n - k) (n - k - 1) / 2 + k$.
\emph{Пример:}
между аэропортами с номерами от 1 до $n - k$ есть все авиалинии; кроме того,
есть авиалинии между аэропортами $(i, i + 1)$ при
$n - k \leq i \leq n - 1$.
\emph{Оценка.}
Покажем, что больше авиалиний быть не может.
Перейдем на язык графов, тогда узловые аэропорты~--- это точки сочленения.
Нам понадобится понятие дерева блоков и точек сочленения графа.
Напомним об этом.
\par
Два ребра графа будем называть \emph{похожими}, если они совпадают или входят в
общий простой цикл (с разными вершинами).
Ключевой момент состоит в том, что введенное нами отношение похожести есть отношение эквивалентности.
Действительно, достаточно доказать транзитивность.
Пусть ребра $a$, $b$ входят в простой цикл Вася, ребра $b$, $c$ входят в
простой цикл Всеволод.
Пойдем по Всеволоду от концов ребра $c$ в две стороны, пока не наткнемся на
Васю.
Это произойдет не позже, чем мы подойдем к ребру $b$, так что на Васю мы
наткнемся в разных вершинах $x_1$, $x_2$.
Чтобы получить простой цикл, содержащий ребра $a$ и $c$, достаточно взять те
ребра Всеволода, по которым мы прошли (в том числе $c$), и добавить ту часть
Васи от $x_1$  до $x_2$, в которой лежит ребро $a$.
Класс эквивалентности ребер будем называть \emph{блоком}.
Легко видеть, что два блока (как графы) могут иметь не более одной общей
вершины, и эта вершина есть точка сочленения.
Рассмотрим новый граф, вершины которого суть блоки и точки сочленения,
и ребро соответствует тому, что точка сочленения принадлежит блоку.
Нетрудно видеть, что этот граф есть дерево, и его висячей вершиной может быть
только блок.
Он называется \emph{крайним блоком} графа. 
\par
Перейдем к решению задачи.
Индукция по $n$.
Для $n = 2$ утверждение понятно.
Так же оно ясно для $k = 0$.
Пусть для меньших значений $n$ оценка установлена и $k \geq 1$.
Рассмотрим крайний блок, пусть он содержит $r + 1$ вершину.
Удалим его ребра (и все вершины, кроме точки сочленения) из графа,
останется хотя бы $k - 1$ точка сочленения.
Поэтому, в частности, $k - 1 \leq n - r - 2$
(в связном графе с хотя бы двумя вершинами есть хотя бы две не точки
сочленения),
$n - r - k \geq 1$.
Пользуясь индукционным предположением получаем, что количество ребер в графе
не превосходит
$r (r + 1) / 2 + (n - r - k + 1) (n - r - k) / 2 + k - 1$.
Максимизируя это выражение по $r \in [1, n - k - 1]$ получаем,
что максимум достигается при $r = 1$ или $r = n - k + 1$
(это квадратный трехчлен по $r$ с положительным старшим коэффициентом,
так что максимум априори достигается в одном из концов отрезка)
и равен $(n - k) (n - k - 1) / 2 + k$, что и требовалось.

\endproblem
