\ifsolution
\begin{figure}\centering
    \jeolmfigure[width=0.5\textwidth]{solution}
    \caption{к решению задачи \ref{2011/ru/pers-geomt/junior/4:solution}.}
    \label{2011/ru/pers-geomt/junior/4:solution:fig}
\end{figure}%
\fi % \ifsolution

\problem
В выпуклом четырехугольнике $ABCD$ оказалось, что
$AB + CD = \sqrt{2} \cdot AC$
и
$BC + DA = \sqrt{2} \cdot BD$.
Докажите, что $ABCD$~--- параллелограмм.
\solution
\label{2011/ru/pers-geomt/junior/4:solution}%
Рис.~\ref{2011/ru/pers-geomt/junior/4:solution:fig}.
Построим точки $M$ и $N$ так, что
$CM \parallel BD$, $AN \parallel BD$ и $CM = BD = AN$.
Получатся параллелограммы $BCMD$, $BDNA$ и $CANM$, откуда мы
имеем $AB = DN$ и $BC = MD$.
Напишем тождество параллелограмма для $CANM$:
\begin{align*}
    2 AC^2 + 2 BD^2
={}&
    AC^2 + CM^2 + MN^2 + NA^2
=
    AM^2 + CN^2
\leq\\\leq{}&
    (AD + DM)^2 + (CD + DN)^2
=\\={}&
    (AD + BC)^2 + (CD + AB)^2
=
    2 AC^2 + 2 BD^2
.\end{align*}
Отсюда получаем, что в неравенстве равенство, то есть точка $D$ лежит на
$AM$ и на $CN$, то есть $AD = DM = BC$ и $CD = DN = AB$, то и требовалось.
\endproblem
% $problem-source: Czech-Polish-Slovak Match 2010
