\ifsolution
\begin{figure}\centering
    \jeolmfigure[width=0.5\textwidth]{solution}
    \caption{к решению задачи \ref{solution:2011/ru/pers-geomt/senior/2}.}
    \label{fig:solution:2011/ru/pers-geomt/senior/2}
\end{figure}%
\fi % \ifsolution

\problem
Дан тетраэдр $ABCD$.
Сфера, проходящая через вершины $A$, $B$ и $C$, пересекает боковые ребра
$DA$, $DB$ и $DC$ в точках $A_1$, $B_1$ и $C_1$.
Эти точки отразили относительно середин соответствующих ребер и получили точки
$A_2$, $B_2$ и $C_2$.
Докажите, что точки $A$, $B$ и $C$ равноудалены от центра описанной
сферы тетраэдра $D A_2 B_2 C_2$.
\solution
\label{solution:2011/ru/pers-geomt/senior/2}%
Рис.~\ref{fig:solution:2011/ru/pers-geomt/senior/2}.
Заметим, что $D A_1 \cdot DA = D B_1 \cdot DB = D C_1 \cdot DC$, так как все
эти три выражения~--- степень точки $D$ относительно данной сферы.
Тогда мы имеем, что
$AD \cdot A A_2 = BD \cdot B B_2 = CD \cdot C C_2$,
что означает, что точки $A$, $B$ и $C$ имеют одинаковую степень относительно
описанной сферы тетраэдра $D A_2 B_2 C_2$, то есть равноудалены от центра этой
сферы.
\endproblem
