\problem
В вершинах выпуклого многогранника с $n$ вершинами записано по два
положительных числа: синее и красное, причем сумма синих равна сумме красных.
За один ход можно изменить два синих числа в концах любого одного ребра так,
чтобы чтобы они остались положительными и сумма сохранилась.
Докажите, что не более чем за $n - 1$ ход можно добиться, чтобы в каждой
вершине синее число стало равно красному.
\solution
Поскольку граф многогранника связен, достаточно доказать для остовного дерева.
Индукция по $n$.
Пусть висячая вершина $A$ связана ребром с вершиной $B$,
синие и красные числа~--- это $c$ и $k$ в $A$, $c'$ и $k'$ в $B$.
Возможны 3 случая.
\\
\emph{(1)} $c = k$.
Тогда выкинем $A$ и ребро $AB$.
\\
\emph{(2)} $c > k$.
Заменим $c$ и $c'$ соответственно на $k$ и $c' + c - k$, после чего выкинем
$A$ и ребро $AB$.
\\
\emph{(3)} $c < k$.
Временно выкинем $A$ и ребро $AB$ и заменим $k'$ на $k' + k - c$.
Теперь в оставшемся графе суммы синих и красных равны.
Добьемся равенства всех синих и красных не более чем за $n - 2$ шага.
Вернем на место $A$ и ребро $AB$ и восстановим значение $k'$.
Последним ходом заменим на ребре $AB$ синие числа $c$ и $k' + k - c$ на
$k$ и $k'$.
\endproblem
