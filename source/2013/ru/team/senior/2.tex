\problem\problemscore{4}
Существуют ли 2013 ненулевых вещественных чисел
$x_1$, $x_2$, \ldots, $x_{2013}$ таких,
что для бесконечно многих натуральных $n$
верно равенство $x_1^n + x_2^n + \ldots + x_{2013}^n = 0$?
\solution
\emph{Ответ:} нет.
Пусть $|x_1| \geq |x_2| \geq \ldots \geq |x_{2013}|$.
Если $|x_1| > |x_2|$, то начиная с некоторого натурального $n$ будет верно
неравенство
$|x_1|^n > 2012 |x_2|^n \geq |x_2|^n + |x_3|^n + \ldots + |x_{2013}|^n$
и это означает, что сумма $n$-ых степеней не может быть нулем.
Значит $|x_1| = |x_2|$.
Теперь заметим, что среди чисел, равных по модулю $x_1$ должно быть поровну
положительных и отрицательных, иначе аналогично убеждаемся, что начиная с
некоторого $n$ сумма не может быть нулем.
Отбрасывая числа с максимальным модулем (которых, как мы поняли, четное число),
мы продолжим аналогичные рассуждения.
Таким образом, будет доказано, что чисел обязательно четное число, то есть их
не может быть 2013.
\endproblem
% $problem-source: В.\,Быковский
