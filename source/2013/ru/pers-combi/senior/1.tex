\problem
Петя и Вася по очереди красят ребра 77-угольной пирамиды, по одному за ход.
Можно красить еще не окрашенное ребро, у которого все смежные ребра не
окрашены.
Кто не может сделать ход~--- проигрывает.
Начинает Петя.
Кто из них может выигрывать, как бы не играл соперник?
(Ребра \emph{смежные}, если у них есть общая вершина.)
\solution
\emph{Ответ:} Вася.
\par
Разобьем ребра на пары, сопоставив каждому ребру основания боковое ребро,
проходящее через противоположную вершину основания.
В ответ на первый ход Пети Вася красит второе ребро из той же пары.
После этого можно красить только некоторые ребра основания, которые разбиваются
на два одинаковых не связанных между собой набора.
На~ход Пети в~любом из~наборов Вася повторяет его в~другом наборе.
\endproblem
% $problem-source: А.\,Шаповалов
