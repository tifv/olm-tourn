\problem\problemscore{4}
В ряд лежат $11$ монет, любые два соседа отличаются на $1$ грамм.
Известно, что у одной не крайней монеты оба соседа легче её, а у остальных не
крайних~--- один сосед легче, другой~--- тяжелее.
Как найти самую тяжелую монету за $2$ взвешивания на чашечных весах без гирь?

\solution
Обозначим монеты $M_1$, $M_2$, \ldots, $M_{11}$.
Пусть наибольший вес у $M_k$.
Первым взвешиванием сравним $M_1 + M_8$ с $M_4 + M_{11}$.
Если перевесит левая чаша, то $k = 2, 3 \text{ или } 4$,
если левая~--- то $k = 8, 9 \text{ или } 10$,
при равновесии $k = 5, 6 \text{ или } 7$.
Вторым взвешиванием сравниваем две монеты с крайними номерами из тройки
подозрительных номеров.

\endproblem
