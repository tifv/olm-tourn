\problem
Шахматная доска покрыта 32 домино (каждое домино покрывает ровно два поля).
Докажите, что эти домино можно повернуть на 90 или на 180 градусов
(каждое~--- вокруг центра одной из закрываемых им клеток, поворачивать можно
независимо друг от друга и в любую сторону), чтобы по-прежнему вся доска была
покрыта.

\solution
Разобьем доску на клетки $2 \times 2$.
Уложим второй слой домино так: если в клетке лежит целиком горизонтальное
домино, кладем на нее два вертикальных, иначе~--- два горизонтальных.
В результате ни одно верхнее домино не совпадет по положению с нижним.
А теперь вспомним о шахматной раскраске доски и повернем каждое нижнее домино
так, чтобы оно заняло положение верхнего домино, закрывающего ту же белую
клетку.
\endproblem
% $criteria:
%   Если доказано, что существует второе разбиение, где ни одна доминошка не
%   совпадает с первой~--- дать 2 балла за частичное продвижение.
% $problem-source: А.\,Шаповалов
