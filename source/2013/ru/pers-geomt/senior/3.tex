\problem
При каких $a > 1$ существует выпуклый многогранник, отношение площадей любых
двух граней которого больше $a$?
\solution
\emph{Ответ:} при всех $a \in (1, 2)$.
\par
\emph{Лемма.}
Пусть грани выпуклого многогранника имеют площади
$S_1$, $S_2$, $\ldots$, $S_n$.
Тогда выполняется неравенство: $S_n \leq S_1 + S_2 + \ldots + S_{n-1}$.
(Аналог неравенства треугольника для многогранника).
\\\emph{Доказательство леммы:}
спроектируем многогранник на плоскость грани с~площадью $S_n$.
Тогда проекции остальных граней, очевидно, накроют грань $S_n$
(в силу выпуклости многогранника).
Отсюда следует, что $S_n$ не превосходит суммы площадей проекций остальных
граней.
Но площадь проекции каждой грани не превосходит площади самой грани
(так как получается из~неё домножением на косинус угла между гранями),
откуда и следует требуемое неравенство.
\par
Теперь покажем, что при $a \geq 2$ такого многогранника существовать не может.
Пусть площади граней этого многогранника
$S_1 < S_2 < S_3 < \ldots < S_{n-1} < S_n = 1$.
Тогда из условия следует, что $S_{k-1} \leq S_k / a$, то есть
$S_k \leq 1 / a^{n-k}$.
Напишем теперь неравенство из леммы:
\[
    1
=
    S_n \leq S_1 + S_2 + \ldots + S_{n-1}
\leq
    \frac{1}{a^{n-1}} + \frac{1}{a^{n-2}}
    + \ldots +
    \frac{1}{a}
\leq
    \frac{1}{2^{n-1}} + \frac{1}{2^{n-2}}
    + \ldots +
    \frac{1}{2}
<
    1
.\]
Полученное противоречие означает, что при $a \geq 2$ такого
многогранника существовать не может.
\par
Теперь построим пример такого многогранника для $a_0 < 2$.
Выберем $a$ так, что $a_0 < a < 2$.
Сперва заметим, что найдется такое натуральное $n$, что
%$\frac{1}{a} + \frac{1}{a^2} + \ldots + \frac{1}{a^{n-1}} > 1$.
$1 / a + 1 / a^2 + \ldots + 1 / a^{n-1} > 1$.
Действительно, сумма геометрической прогрессии в левой части этого неравенства
равна $(1 / a^{n-1}) \cdot \frac{a^{n-1} - 1}{a - 1}$.
Условие того, что последнее выражение больше 1, равносильно тому, что
$1 - 1 / a^{n-1} > a - 1$.
Последнее неравенство перепишем в виде $2 > a + 1 / a^{n-1}$.
Отсюда уже очевидно, что при $a < 2$ мы сможем подобрать подходящее $n$.
Зафиксируем найденное $n$ и построим выпуклый многоугольник $M$ со сторонами
$1$, $1 / a$, $\ldots$, $1 / a^{n-1}$
(условие, которого мы добились на предыдущем шаге, означает, что это возможно).
%Пусть площадь этого многоугольника равна $S_n$.
Он будет основанием нашей пирамиды.
Выберем внутри многоугольника $M$ произвольную точку $O$.
Пусть расстояния от нее до сторон равны $h_0$, $h_1$, $\ldots$, $h_{n-1}$
соответственно.
Теперь выберем точку $P$ вне плоскости построенного многоугольника, но так, что
бы ее проекция попадала в точку $O$.
(По смыслу она будет расположена <<далеко>> от этой плоскости.)
Пусть расстояние от нее до плоскости равно $h$.
Покажем, что подходящим выбором $h$ мы сможем добиться того, чтобы пирамида с
вершиной в $P$ и основанием $M$ будет искомой.
Для этого посчитаем площадь треугольной грани:
\[
    S_k
=
    \frac{1}{2} \frac{\sqrt{h^2 + h_k^2}}{a^k}
\;.\]
Это означает, что отношение площадей соседних граней имеет вид 
\[
    \frac{S_k}{S_{k+1}}
=
    a \cdot \sqrt{\frac{h^2 + h_k^2}{h^2 + h_{k+1}^2}}.
\] 
Выбирая $h$ достаточно большим, мы сможем добиться, чтобы любое такое
отношение было больше $a_0$.
Действительно, преобразовывая условие $S_k : S_{k+1} > a_0$,
%\[
%    a \cdot \sqrt{\frac{h^2 + h_k^2}{h^2 + h_{k+1}^2}}
%>
%    a_0
%\]
мы получаем неравенство на $h$:
\[
    h^2
>
    \frac{a_0^2h_{k+1}^2-a^2h_k^2}{a^2-a_0^2}
.\]
Подходящим выбором $h$ мы добьемся выполнения таких неравенств для всех~$k$.
Остается лишь выбрать $h$ еще и таким, чтобы площадь наименьшей боковой грани
$S_{n-1}$ была больше площади основания (которое от $h$ не зависит), умноженной
на~$a_0$.
Очевидно, что этого также можно добиться.
\endproblem
% $problem-source: А.\,Шаповалов
