\ifsolution
\begin{figure}\centering
    \jeolmfigure[width=0.5\textwidth]{solution}
    \caption{к задаче \ref{2014/ru/regatta/senior/geomt/3:solution}}
    \label{2014/ru/regatta/senior/geomt/3:solution:fig}
\end{figure}%
\fi % \ifsolution

\problem
Внутри равнобедренного треугольника $ABC$ ($AC = BC$) выбрали такую точку~$N$, 
что $2 \angle ANB = 180^\circ + \angle ACB$.
Прямая, проходящая через точку~$C$ и~параллельная $AN$, пересекает прямую~$BN$
в~точке~$D$.
Биссектрисы углов $CAN$ и~$ABN$ пересекаются в~точке~$P$.
Докажите, что прямые $DP$ и~$AN$ перпендикулярны.
\solution\label{2014/ru/regatta/senior/geomt/3:solution}
Заметим, что из~равенства $2 \angle ANB = 180^\circ + \angle ACB$ следует, что
$\angle CAN=\angle ABN$ и~$\angle NAB=\angle NBC$.
Так как $CD \parallel AN$,
то~$\angle DCA = \angle CAN = \angle ABN = \angle ABD$.
Следовательно, $D$ лежит на~описанной окружности треугольника $ABC$.
\par
Пусть $E$~--- середина дуги~$AB$ описанной окружности треугольника $ABC$
(рис.~\ref{2014/ru/regatta/senior/geomt/3:solution:fig}).
Так как $CD \parallel AN$, то~достаточно доказать, что $\angle CDP = 90^\circ$,
то~есть что $DP$ проходит через $E$, или что $DP$~--- биссектриса угла $ADB$.
Из~вписанности четырехугольника $ABCD$ следует, что
$\angle DAC = \angle DBC = \angle NBC = \angle NAB$.
Тогда $\angle DAP = \angle DAC + \angle CAP = \angle NAB + \angle PAN$.
Это означает, что $AP$~--- биссектриса угла $DAB$.
С~другой стороны, $BP$~--- биссектриса угла $ABD$.
Тогда $P$~--- точка пересечения биссектрис треугольника $DAB$, что означает,
что $DP$~--- биссектриса угла $ADB$.
\endproblem
% $problem-source: Middle European 2013
