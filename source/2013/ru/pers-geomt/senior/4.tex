\ifsolution
\begin{figure}[b!]\centering
    \jeolmfigure[width=0.5\textwidth]{solution}
    \caption{к решению задачи \ref{2013/ru/pers-geomt/senior/4:solution}.}
    \label{2013/ru/pers-geomt/senior/4:solution:fig}
\end{figure}%
\fi % \ifsolution

\problem
Пусть $H$, $I$ и $O$~--- соответственно точка пересечения высот, центры
вписанной и описанной окружностей остроугольного треугольника $ABC$.
Докажите, что $\angle AIH = 90^\circ$ тогда и только тогда, когда
$OI \parallel BC$.
\solution
\label{2013/ru/pers-geomt/senior/4:solution}%
Рис.~\ref{2013/ru/pers-geomt/senior/4:solution:fig}.
Обозначим через $M$ середину стороны $BC$, через $N$~--- середину отрезка $AH$,
через $L$~--- середину дуги $BC$ (не содержащей точку $A$) описанной окружности
треугольника $ABC$, а через $K$~--- середину дуги $BAC$.
Докажем две вспомогательные леммы.
\par
\emph{Лемма 1.} Отрезки $AN$ и $OM$ равны и параллельны.
\\\emph{Доказательство леммы.}
Действительно, отразим $H$ относительно точки~$M$ в~точку~$H'$.
В силу симметрии имеем $H' B \parallel HC \perp AB$
и $H' C \parallel HB \perp AC$.
Значит, $H'$~--- точка на описанной окружности треугольника $ABC$, диаметрально
противоположная точке~$A$.
Следовательно, $H'$, $O$ и~$A$ лежат на одной прямой и $O$ делит $AH'$ пополам.
Значит, $OM$~--- средняя линия треугольника $AHH'$, а следовательно, равна
половине стороны $AH$ и параллельна ей.
\par
\emph{Лемма 2.} Треугольники $LMI$ и $LIK$ подобны.
\\\emph{Доказательство леммы.}
Вспомним лемму о~трезубце, которая утверждает, что $LI = LB$.
Выполнено $\angle LKB = \angle LBM$ ввиду равенства дуг $LB$ и $LC$.
Из~перпендикулярности прямых $LK$ и~$BC$, а также того, что
$\angle KBL = 90^\circ$, как опирающийся на диаметр, имеем два подобных
треугольника $LBM$ и $LKB$.
Значит,
\[
    \frac{LI}{LM} = \frac{LB}{LM}
=
    \frac{LK}{LB} = \frac{LK}{LI}
.\]
Из этого равенства отношений и из того, что угол $KLI$ у треугольников $MLI$
и $KIL$ общий, получаем требуемое.
\par
Из равенства отрезков $AN$ и~$OM$ и их параллельности получаем, что $ANMO$~---
параллелограмм, а следовательно, $OA \parallel MN$.
Заметим также, что
\(
    \angle BAO
=
    \frac{1}{2}(180^\circ - \angle AOB)
=
    90^\circ - \angle ACB = \angle HAC
\).
Следовательно, $AI$~--- биссектриса угла $OAH$.
\par
Проведем теперь цепочку равносильных переходов.
$\angle AIH = 90^\circ$ тогда и только тогда, когда медиана~$IN$
в~треугольнике $AIH$ равна половине стороны~$AH$.
Это, в~свою очередь, равносильно тому, что в~треугольнике $INA$ равны
$\angle AIN$ и $\angle IAN$.
Так как $\angle IAN$ всегда равен $\angle IAO$, то~это равносильно тому,
что равны $\angle AIN$ и $\angle IAO$, что равносильно параллельности
прямых $AO$ и~$NI$.
Вспоминая, что, по доказанному, $AO \parallel MN$, получаем равносильность
условию того, что точка $I$ лежит на прямой $MN$.
Другими словами, это равносильно тому, что $MI \parallel OA$.
Ввиду того, что точки $M$ и $I$ лежат на сторонах треугольника $AOL$, получаем,
что последнее условие параллельности равносильно подобию
треугольников $AOL$ и~$MIL$.
Но треугольник $AOL$ всегда равнобедренный, так что это условие равносильно
тому, что треугольник $LMI$ равнобедренный.
Но этот треугольник по лемме~2 подобен треугольнику $LIK$, то есть
изначальное условие равносильно условию равнобедренности треугольника $LIK$.
В свою очередь, это равносильно тому, что медиана~$OI$ этого треугольника
совпадает с~высотой.
Иными словами, это равносильно тому, что $OI \perp KL$.
Последнее очевидно равносильно условию $BC \parallel OI$, что и требовалось.
%\par\emph{Другое решение; необходимость.}
%Пусть $B_1$ и $C_1$~--- основания соответствующих высот треугольника $ABC$.
%Центр окружности $\omega$, построенной на $AH$ как на диаметре и проходящей
%через $B_1$ и $C_1$, обозначим через $F$.
%Если $\angle AIH = 90^\circ$, то точка $I$ лежит на $\omega$ и является
%серединой дуги $B_1 I C_1$.
%Следовательно, $I$ лежит на линии центров окружности $\omega$ и окружности
%девяти точек треугольника $ABC$.
%Вписанная окружность треугольника $ABC$ и окружность девяти точек касаются в
%точке Фейербаха, которая лежит на линии центров.
%Откуда получаем, что точка Фейербаха совпадает с $F$.
%В частности, $I F = r$.
%То есть радиус окружности $\omega$ равен $r$.
%Поскольку $AH$ равно удвоенному расстоянию от $O$ до $BC$, получаем, что точки
%$O$ и $I$ равноудалены от $BC$, что и требовалось.
\endproblem
% $problem-source: Ф.\,Ивлев, решила Ю.\,Зайцева
