\ifsolution
\begin{figure}\centering
    \jeolmfigure[width=0.5\textwidth]{solution}
    \caption{к решению задачи \ref{2013/ru/team/junior/4:solution}.}
    \label{2013/ru/team/junior/4:solution:fig}
\end{figure}%
\fi % \ifsolution

\problem\problemscore{6}
Дан равнобедренный треугольник $ABC$ с основанием~$AC$.
На продолжении стороны~$AB$ за точку~$B$ отмечена точка~$D$ такая, что
$\angle DCB = \angle BCA$.
На высоте~$BH$ треугольника $ABC$ отмечена точка~$E$ такая, что $DE = DC$.
Докажите, что $\angle BDE = \frac{1}{3} \angle BDC$. 
\solution
\label{2013/ru/team/junior/4:solution}%
Рис.~\ref{2013/ru/team/junior/4:solution:fig}.
Отметим точку~$D'$, симметричную точке~$D$ относительно прямой~$BH$.
Мы получим равнобедренную трапецию $AD'DC$, в которой, по условию,
диагональ~$CD'$ является биссектрисой угла $DCA$.
В силу параллельности оснований трапеции имеем
$\angle CD'D = \angle D'CA = \angle DCD'$, откуда $D'D = DC$.
Теперь продлим ось симметрии трапеции $BH$ до пересечения с $DD'$ в точке $F$.
Тогда $FD = \frac{1}{2} DD' = \frac{1}{2} DE$, откуда мы получаем, что
$\angle DEF = 30^\circ$, так как в треугольнике $FED$ катет оказался равен
половине гипотенузы.
Теперь осталось просто аккуратно посчитать углы.
Для этого введем обозначения: $\angle BCD = x$ и $\angle BDE = y$.
Так как $\angle D'DB$ также равен $x$, а $\angle D'DE$ равен $60^\circ$,
то $x + y = 60^\circ$, или $3 x + 3 y = 180^\circ$.
%Тогда мы получили, что $90^\circ + x + y + 30^\circ = 180^\circ$
%(сумма углов треугольника $EBD$.)
%Отсюда мы имеем, что $x + y = 60^\circ$ или $3 x + 3 y = 180^\circ$.
Теперь рассмотрим треугольник $DBC$: в нем $\angle BCD = x$ и
$\angle DBC = 2 x$ (как внешний угол треугольника $ABC$), а следовательно,
$\angle BDC = 3 y$.
%(из суммы углов треугольника $BDC$).
Мы получили то, что требовалось.
\endproblem
% $problem-source: Georgia TST 2005
