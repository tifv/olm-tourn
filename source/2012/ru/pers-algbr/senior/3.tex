\problem
Последовательность $\{ x_n \}$ задана условиями
$x_0 = 1$, $x_1 = 1$, $x_{n + 1} = 2 x_{n} + x_{n - 1}$ при $n \geq 1$.
Докажите, что если число $x_n$~--- простое, то $n$~--- либо степень числа 2,
либо простое.
\solution
Заметим, что последовательность
$y_n = (\alpha^n + \beta^n)/2$,
где $\alpha = 1 + \sqrt{2}$ и $\beta = 1 - \sqrt{2}$,
подходит условию;
также очевидно, что последовательность, подходящая условию, единственна,
поэтому $x_n = y_n$ для всех $n$.
Допустим, что $y_n$ просто при некотором $n$, и $n$ не простое число и не
является степенью двойки.
Тогда $n$ имеет такой нечетный делитель $k > 1$, что $l = n / k > 1$.
Из условия очевидно, что последовательность $x_n$ строго возрастает, поэтому
$x_n > x_l > 1$.
\begin{align*}
    \frac{\alpha^n + \beta^n}{2}
&{}=
    \frac{(\alpha^l)^k + (\beta^l)^k}{2}
=\\&{}=
    \frac{\alpha^l + \beta^l}{2}
    \cdot
    \left(
        \alpha^{l (k - 1)} \beta^l
        +
        \alpha^{l (k - 2)} \beta^{2 l}
        + \ldots +
        \alpha^l \beta^{l (k - 1)}
    \right)
=
    x_l \cdot X
.\end{align*}
Заметим, что $X$ является симметрическим многочленом от $\alpha$ и $\beta$ с
целыми коэффициентами,
поэтому выражается через $\alpha + \beta = 2$ и $\alpha \beta = -1$
с целыми коэффициентами, значит, является целым.
Отсюда простое $x_n$ делится на $x_l$, что входит в противоречие с
неравенством $x_n > x_l > 1$.
\endproblem
