\problem
A quadrilateral $ABCD$ is inscribed into a circle, given $AB > CD$ and $BC > AD$.
Points $K$ and $M$ are chosen on the rays $AB$ and $CD$ respectively in such a
way that $AK = CM = \frac{1}{2} (AB + CD)$.
Points $L$ and $N$ are chosen on the rays $BC$ and $DA$ respectively in such a
way that $BL = DN = \frac{1}{2} (BC + AD)$.
Prove that the $KLMN$ is a rectangle of the same area as $ABCD$.
\solution 
From given equalities one get that $KB=MD$, and $K$ is lying on the segmet $AB$, and point $D$ is lying on segment $MC$.
Similarly $NA = LC$, $L$ on segment $BC$, and $A$ on segment $ND$.
In triangles $MCL$ and $KAN$ the equalities $NA = CL$, $KA = MC$ and $\angle NAK = \angle LCM$ hold.
Thus, the triangles $MCL$ and $KAN$ are equal and $NK = ML$.
In the same way one get that $\triangle KBL = \triangle MDN$ and $KL = MN$.
Consequently $KLMN$ is a parallelogram.
One easily get that $\angle KLM = 180^\circ - \angle KLB - \angle MLC = 180^\circ - \angle MAD - \angle KNA = 180^\circ - \angle KNM$.
So the oposite angles of parallelogram have the sum equal to $180^\circ$.
This can be true only in case of rectangular.
So it's enough to show that the quadrilaterl $ABCD$ has the same area as rectangular $KLMN$.
For this let $F$ be an intersection point of the segments $ML$ and $AD$.
Then the chain of equalities take place 
\[
    S_{ABCD}
=
    S_{AKLF} + S_{KBL} + S_{LCM} - S_{MFD} = S_{AKLF} + S_{MDN} + S_{MAK} - S_{MFD}
=
    S_{KLMN}.
\]
\endproblem
% $problem-source: Eisso\,J.\,Atzema, proposed by В.\,Дубровский
