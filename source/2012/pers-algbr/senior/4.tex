Найдите наименьшее положительное $C$ такое, что неравенство
\[
    \frac{x}{\sqrt{yz}} \cdot \frac{1}{x+1}
    +
    \frac{y}{\sqrt{zx}} \cdot \frac{1}{y+1}
    +
    \frac{z}{\sqrt{xy}} \cdot \frac{1}{z+1}
\leq
    C
\]
выполнено для любых положительных чисел $x$, $y$, $z$, удовлетворяющих
равенству
\[
    \frac{1}{x+1} + \frac{1}{y+1} + \frac{1}{z+1}
=
    1
.\]

\solution
Сделаем замену:
$a = 1 / (x + 1)$, $b = 1 / (y + 1)$,  $c = 1 / (z + 1)$, откуда
$x = (1 - a) / a$, $y = (1 - b) / b$, $z = (1 - c) / c$,
$a + b + c = 1$,
\begin{align*}
    W
={}&
    \frac{x}{\sqrt{y z}} \cdot \frac{1}{x + 1}
    +
    \frac{y}{\sqrt{z x}} \cdot \frac{1}{y + 1}
    +
    \frac{z}{\sqrt{x y}} \cdot \frac{1}{z + 1}
=\\={}&
    (1 - a) \sqrt{\frac{b c \mathstrut}{(1 - b) (1 - c)}}
    +
    (1 - b) \sqrt{\frac{c a \mathstrut}{(1 - c) (1 - a)}}
    +
    (1 - c) \sqrt{\frac{a b \mathstrut}{(1 - a) (1 - b)}}
.\end{align*}
Возьмем $a = b = \eps, c = 1 - 2 \eps$.
Тогда 
\[
    W
=
    \sqrt{2} (1 - \eps) \sqrt{\frac{1 - 2 \eps}{1 - \eps}}
    +
    \frac{2{\eps}^2}{1 - \eps}
>
    \sqrt{2} (1 - \eps) \frac{1 - 2 \eps}{1 - \eps}
=
    \sqrt{2} (1 - 2 \eps).
\]
Если $0 < D < \sqrt{2}$, то приняв $\eps = (1 - D / \sqrt{2}) / 2$,
получим, что $W > D$.
Поэтому требуемое $C$ больше или равно $\sqrt{2}$.
\par
Докажем, что $W$ всегда не превосходит $\sqrt{2}$.
Без ограничения общности, $c$ является наибольшим среди чисел $a$, $b$, $c$.
Заметим, что следующие неравенства верны:
\begin{gather*}
    (1 - a) \sqrt{\frac{b c}{(1 - b) (1 - c)}}
=
    (1 - a) \sqrt{\frac{b}{1 - c} \cdot \frac{c}{a + c}}
\leq
    (1 - a) \sqrt{\frac{b}{1 - c}}
,\\
    (1 - b) \sqrt{\frac{c a}{(1 - c) (1 - a)}}
=
    (1 - b) \sqrt{\frac{a}{1 - c} \cdot \frac{c}{b + c}}
\leq
    (1 - b) \sqrt{\frac{a}{1 - c}}
,\\
    a \sqrt{\frac{a b}{(1 - a) (1 - b)}}
=
    a \sqrt{\frac{b}{1 - a} \cdot \frac{a}{a + c}}
\leq
    a \sqrt{\frac{b}{1 - c}}
,\\
    b \sqrt{\frac{a b}{(1 - a) (1 - b)}}
=
    b \sqrt{\frac{a}{1 - b} \cdot \frac{b}{b + c}}
\leq
    a \sqrt{\frac{a}{1 - c}}
.\end{gather*}
Если их сложить, получим, что 
\[
    W
\leq
    \sqrt{\frac{a}{1 - c}} + \sqrt{\frac{b}{1 - c}}
.\]
Заметим, что сумма подкоренных выражений равна 1, поэтому по неравенству
между средним арифметическим и средним квадратическим имеем, что
\[
    W
\leq
    2 \sqrt{1 / 2}
=
    \sqrt{2}
.\]
Поэтому $C \leq \sqrt{2}$, а значит $C = \sqrt{2}$.  

