За круглым столом сидели четыре студента.
% "Л\'{у}зин" -> "Лузин"
Филолог сидел против Л\'{у}зина, рядом с историком.
Математик сидел рядом с Лебедевым.
Соседи Лихачёва~--- Соловьёв и физик.
Какая профессия у Лузина?

\solution
\emph{Ответ:} Математик.
Лузин не может быть ни филологом (так как сидел напротив), ни историком
(с которым сидел рядом).
Значит, он либо физик, либо математик.
Если Лузин физик, то Лихачёв сидел между физиком Лузиным и филологом
Соловьёвым.
Но тогда математик не может сидеть рядом с Лебедевым~--- противоречие.
Значит, Лузин~--- математик.
Можно убедиться, что в этом случае все сходится.
Действительно, физик не Лихачёв, не Соловьёв и не Лузин~--- значит, Лебедев.
Рядом с Лихачёвым нет Лузина, значит, Лихачёв напротив Лузина, то есть он~---
филолог.
Тогда Соловьёв~--- историк, и сидят они в таком порядке:
математик Лузин, физик Лебедев, филолог Лихачёв и историк Соловьёв.

