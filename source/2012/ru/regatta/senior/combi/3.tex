\problem
На некоторых полях доски $100 \times 100$ стоят столбики из шашек.
За один ход разрешается переставить любой столбик на столько клеток по
вертикали или горизонтали, сколько в нём шашек;
если столбик попал на непустую клетку, он ставится на верх стоящего там
столбика и объединяется с ним.
Вначале на каждой клетке стоит по одной шашке.
Можно ли за $9999$ ходов собрать их все на одной клетке?
\solution
\emph{Ответ:} нельзя.
Допустим, что можно собрать все шашки на клетке $K$.
Рассмотрим все ходы на клетку $K$.
Каждым ходом число свободных клеток должно увеличиваться на 1, поэтому
освободившаяся клетка больше не занимается.
Следовательно, на $K$ сделано не более $198$ ходов~--- из клеток, находящихся
с $K$ на одной вертикали или горизонтали.
Сумма расстояний от $K$ до этих клеток не превосходит
$2 \cdot (1 + 2 + \ldots + 99) = 9900$,
поэтому на ней может собраться не более $9901$ шашек.
\endproblem
