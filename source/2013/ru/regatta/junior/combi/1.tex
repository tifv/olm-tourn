\problem
Трём братьям надо перевезти с одной квартиры на другую рояль весом 250 кг,
диван весом 100 кг и более 100 коробок по 50 кг.
Был нанят небольшой фургон с шофером на 5 рейсов туда (и 4 обратно), который
может за раз перевезти 500 кг груза и одного пассажира.
Погрузить или выгрузить диван братья могут вдвоём, рояль~--- втроём, с
коробками любой из братьев справляется в одиночку.
Надо перевезти всю мебель и как можно больше коробок.
Какое наибольшее число коробок удастся перевезти?
(Шофер не грузит, другого транспорта и помощников нет, пассажиров вместо груза
везти нельзя).

\solution
\emph{Ответ:} 33 коробки.
\emph{Пример.}
Последовательность рейсов:
\begin{enumerate}
\item
Загрузим рояль и 5 коробок, уедет один брат.
Там он выгрузит коробки и останется.
Фургон с роялем вернутся.
\item
Загрузим диван и 3 коробки, уедет еще один брат.
Там братья выгрузят диван и коробки.
Фургон без пассажиров с роялем вернется.
\item
Грузим 5 коробок.
С фургоном едет последний брат.
Там братья выгрузят всё.
Фургон вернется пустым с одним братом.
\item
Грузят 10 коробок.
Там их выгружают.
Фургон возвращается пустым и без пассажиров.
\item
Грузят 10 коробок.
Там их выгружают.
\end{enumerate}
\emph{Оценка}.
Всего можно перевезти 2500 кг груза.
От момента погрузки рояля до момента его выгрузки фургон сделает не менее 3
рейсов туда, чтобы перевезти трёх братьев.
Значит, рояль съездит <<туда>> не менее 3 раз.
Вычитая $3 \cdot 250 \, \text{кг}$ и $100 \, \text{кг}$ дивана, получаем, что
перевезено не более $1650 \, \text{кг}$ коробок, то есть, не более 33 коробок.
\endproblem
% $criteria: Алгоритм и верный ответ без оценки~--- 2 балла.
% $problem-source: А.\,Шаповалов, Д.\,Шаповалов
