\ifsolution
\begin{figure}\centering
    \jeolmfigure[width=0.5\textwidth]{solution}
    \caption{к задаче \ref{solution:2012/ru/pers-geomt/junior/2}}
    \label{fig:solution:2012/ru/pers-geomt/junior/2}
\end{figure}
\fi % \ifsolution

\problem
В трапеции $ABCD$ с основаниями $AD$ и $BC$ диагонали перпендикулярны.
Точки $K$ и $L$ на боковых сторонах $AB$ и $CD$ соответственно таковы, что
отрезок $KL$ проходит через точку пересечения диагоналей трапеции $ABCD$ и
параллелен её основаниям.
На боковой стороне $AB$ отмечена точка $M$ такая, что $AM = BK$.
Докажите, что $LM = AB$.
\solution
\label{solution:2012/ru/pers-geomt/junior/2}%
Рис.~\ref{fig:solution:2012/ru/pers-geomt/junior/2}.
Обозначим через $O$ точку пересечения диагоналей трапеции.
Покажем, что $KO = OL$.
Действительно, $KO / BC = AK / AB$ из подобия треугольников
$AKO$ и $ABC$.
Далее, $AK / AB = DL / DC$ из теоремы Фалеса.
Наконец, $DL / DC = OL / BC$, на этот раз из подобия
треугольников $DOL$ и $DBC$.
Соединяя все эти равенства, имеем $KO = OL$.
Теперь обозначим середину боковой стороны $AB$ через $T$, по условию она же
будет серединой отрезка $KM$.
Тогда $TO$~--- средняя линия треугольника $MKL$, откуда $ML = 2 TO$.
С другой стороны, отрезок $TO$ является медианой прямоугольного треугольника
$BOA$
(он прямоугольный в силу перпендикулярности диагоналей),
откуда следует что $2 TO = AB$, так как медиана прямоугольного треугольника в
два раза меньше его гипотенузы.
Сопоставляя эти два равенства, получаем требуемое.
\endproblem
