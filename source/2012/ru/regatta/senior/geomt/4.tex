\ifsolution
\begin{figure}\centering
    \jeolmfigure[width=0.5\textwidth]{solution}
    \caption{к решению задачи \ref{2012/ru/regatta/senior/geomt/4:solution}}
    \label{2012/ru/regatta/senior/geomt/4:solution:fig}
\end{figure}%
\fi % \ifsolution

\problem
Точки $K$, $L$, $M$ и $N$ лежат соответственно на сторонах
$AB$, $BC$, $CD$ и $DA$ квадрата $ABCD$.
Оказалось, что $\angle KLA = \angle LAM = \angle AMN = 45^\circ$.
Докажите, что $KL^2 + AM^2 = LA^2 + MN^2$.
\solution
\label{2012/ru/regatta/senior/geomt/4:solution}%
Рис.~\ref{2012/ru/regatta/senior/geomt/4:solution:fig}.
Заметим, что требуемое равенство эквивалентно
$AM^2 - MN^2 = LA^2 - LK^2$,
что, в свою очередь, можно расписать по теореме Пифагора:
\begin{align*}
&
    AD^2 + DM^2 - (DN^2 + DM^2) = AB^2 + BL^2 - (KB^2 + BL^2)
\\&\Leftrightarrow\quad
    AD^2 - DN^2 = AB^2 - KB^2
\quad\Leftrightarrow\quad
    DN = KB
.\end{align*}
Теперь будем доказывать именно это равенство.
Сделаем поворот с центром в точке $A$ на $90^\circ$ так, чтобы точка $B$
перешла в $D$.
Тогда, так как прямая $AB$ перейдет в прямую $AD$, наша цель~--- доказать, то
$K$ перешла в $N$.
Обозначим образ точки $L$ через $L'$, а точки $K$~--- через $K'$.
Поскольку $L$ лежала на $BC$, то теперь $L'$ лежит на $DC$.
Так как $\angle NMA = \angle MAL$, то $AL \parallel NM$.
Аналогично $KL \parallel AM$.
Значит, после поворота имеем
$AL' \perp NM$, $K'L' \perp AM$ и $AK' \perp L'M$.
То есть $K'$ есть точка пересечения высот треугольника $AL'M$.
Следовательно, $K'$ лежит на $AD$ и на $MN$.
То есть $K' = N$, что и требовалось.
\endproblem
