Найдите наименьшее натуральное число, дающее попарно различные остатки при
делении на $2$, $3$, $4$, $5$, $6$, $7$, $8$, $9$, $10$.

\solution
\label{solution:2012/pers-algbr/junior/1}%
Изучим все числа, которые обладают таким свойством.
Обозначим рассматриваемое нами число с данным свойством через $N$.
Если такое число четное, то его остаток от деления на 4~--- это 2.
Тогда остаток от деления на 6~--- это 4, \ldots,
остаток от деления на 10~--- это 8.
Тогда остаток от деления на 3~--- это 1, от деления на 5~--- это 3, и т.\,д.
Таким образом, если $N$ четное, то $N + 2$ делится на все числа от 2 до 10, и
значит $N \geq [2, 3, \ldots, 10] - 2$.
Теперь пусть $N$ нечетно.
Аналогично получаем, что $N$ дает остаток 3 от деления на 4,
остаток 5 от деления на 6 и т.\,д.
Далее $N$ дает остаток 2 от деления на 3
(так как $N$ дает остаток 5 от деления на 6)
и остаток 4 от деления на 5.
Остаток от деления на 9 может быть равен только 8
(так как $N$ не делится на 3 и нечетные остатки заняты),
а для остатка от деления на 7 остаются 2 варианта: 0 и 6.
В первом случае $N + 1$ делится на все числа от 2 до 10, и значит
$N \geq [2, 3, \ldots, 10] - 1$.
Во втором получаем, что $N + 1$ кратно $[2, 3, 4, 5, 6, 8, 9, 10] = 360$ и дает
остаток 1 от деления на 7.
Наименьшее такое число~--- это 1800, и значит $N = 1799$ будет ответом.

