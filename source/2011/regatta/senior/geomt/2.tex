В четырехугольнике $ABCD$ с углами
$\angle A = 60^{\circ}$, $\angle B = 90^{\circ}$, $\angle C = 120^{\circ}$
отметили точку пересечения диагоналей $M$.
Оказалось, что $BM = 1$ и $MD = 2$.
Найдите площадь $ABCD$.
%Norway 2004

%In a quadrilateral $ABCD$ with
%$\angle A = 60^{\circ}$, $\angle B = 90^{\circ}$, $\angle C = 120^{\circ}$,
%the point $M$ of intersection of the diagonals satisfies $BM = 1$ and $MD = 2$.
%Find the area of quadrilateral $ABCD$.

\solution
[{\begin{figure}
\centering
    \jeolmfigure[width=0.5\textwidth]{2-solution}
\caption{к задаче \ref{solution:2011/regatta/senior/geomt/2}.}
\label{fig:solution:2011/regatta/senior/geomt/2}
\end{figure}}]%
\label{solution:2011/regatta/senior/geomt/2}%
\emph{Ответ:} $9 / 2$.
См. рис. \ref{fig:solution:2011/regatta/senior/geomt/2}.
Заметим, что $ABCD$~--- вписанный четырехугольник, и $AC$~--- диаметр его
описанной окружности.
Пусть $O$~--- середина $AC$ и центр описанной окружности $ABCD$.
Тогда $\angle BOD = 2 \angle BAD = 120^{\circ}$.
Пусть $K$~--- середина $BD$.
Тогда $MK = 1 / 2$, $KD = 3 / 2$, $OK = \sqrt{3} / 2$, стало быть
$MK : KO = 1 : \sqrt{3}$ и потому $\angle MOK = 30^{\circ}$.
Поэтому $\angle DOC = 90^{\circ}$, треугольник $ACD$~--- равнобедренный
прямоугольный.
По теореме синусов $3 = BD = AC \sin \angle BAD$, откуда
$AC = 2 \sqrt{3}$, $S(ACD) = (AC)^2 / 4 = 3$,
$S(ABC) = S(ACD) \cdot (BM : MD) = 3 / 2$.

