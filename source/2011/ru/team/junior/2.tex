\problem\problemscore{5}
В треугольнике $ABC$ проведена биссектриса $AL$.
Известно, что $AB = 2007$, $BL = AC$.
Найдите стороны треугольника $ABC$, если известно, что они целые.

\solution
[{\begin{figure}
\centering
    \jeolmfigure[width=0.5\textwidth]{../2-solution}
\caption{к задаче \ref{solution:2011/team/junior/2}.}
\label{fig:solution:2011/team/junior/2}
\end{figure}}]%
\label{solution:2011/team/junior/2}%
\emph{Ответ:} $BC = 10 \cdot 223$, $AC = 6 \cdot 223$.
См.\,рис.\,\ref{fig:solution:2011/team/junior/2}.
Пусть $AC = a$ и $LC = b$.
По условию это целые числа.
Имеем $2007 / a = a / b = (2007 + a) / (a + b) > 1$ 1 в силу неравенства
треугольника.
Отсюда $a^2 = 2007 b = 3^2 \cdot 223 \cdot b$.
Так как  $a < 2007$, то есть два варианта:
$b = 223$ и $b = 4 \cdot 223$.
В этих случаях получаем $a = 3 \cdot 223$ или $6 \cdot 223$ и окончательно
$BC = a + b = 4 \cdot 223$ или $10 \cdot 223$, а $AC = 3 \cdot 223$ или
$6 \cdot 223$ соответственно.
Первый вариант не удовлетворяет неравенству треугольника $BC + AC > AB$.

\endproblem
