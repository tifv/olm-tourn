Компания из нескольких людей называется \emph{связной}, если её нельзя разбить
на две непустые группы так, что люди из разных групп будут не знакомы.
В некоторой связной компании каждый знает ровно четверых, и четверо знакомых
каждого человека образуют связную компанию.
Докажите, что людей этой компании можно поставить по кругу так, чтобы любые два
соседа были знакомы.

\solution
Будем называть компанию графом, людей вершинами, знакомства ребрами, построения
по кругу циклами.
Рассмотрим наибольший цикл в нашем графе
(ясно, что цикл есть: в дереве была бы висячая вершина).
Предположим, что он содержит не все вершины, и $ab$~--- ребро из вершины цикла
$a$, ведущее вне цикла.
Если из $a$ ведет еще одно ребро $ac$ вне цикла, то соседей $a$ можно разбить
на две группы --- $\{b, c\}$ и два соседа $a$ по циклу, --- так, что ребер
между вершинами разных групп нет (иначе бы цикл увеличивался).
Таким образом, из $a$ ведет ребро $ab$ вне цикла, ребра цикла $ad$ и $ae$ и еще
одно ребро внутрь цикла $ac$.
Вершина $b$ не смежна с $d$ и $e$ (иначе увеличим цикл), значит, она смежна с
$c$.
Если $aecd$ --- весь цикл, то проводя те же рассуждения, что для $a$, для
вершины $e$, получим, что $e$ и $d$ смежны и потому есть более длинный цикл
$cedab$ --- противоречие.
Итак, в цикле не менее пяти вершин.
Одна из вершин $b$, $c$ по условию должна быть смежна с одной из вершин
$d$, $e$.
Это не вершина $b$, так что вершина $c$ смежна с одной из них.
Но из вершины $c$ ведут ребра в $a$, $b$ и соседей $c$ по циклу.
Следовательно, один из этих соседей --- $d$ или $e$.
Пусть $e$ --- сосед $c$, тогда $d$ --- не сосед
(так как наш цикл содержит более 4 вершин).
Но тогда вершина $d$ должна быть смежна с $e$ (иначе $\{d\}$ и $\{b,e,c\}$~---
разбиение соседей $a$ на две части, между которыми нет ребер).
Теперь можно увеличить цикл, рассмотрев новый цикл $c \ldots deab$. 
