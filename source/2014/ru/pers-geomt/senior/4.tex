\problem
Множество точек в~пространстве назовем интересным, если для любой плоскости вне
её~находится хотя~бы 100 точек этого множества.
При каком наименьшем $d$ можно утверждать наверняка, что любое интересное
множество точек в~пространстве содержит интересное подмножество не~более чем
с~$d$~точками?
\solution
\emph{Ответ:} $d = 300$.
\par
Приведем пример, показывающий, что интересное множество из~300 точек может
не~содержать меньших интересных подмножеств: на~каждой из~координатных осей
выберем 100 точек, отличных от~начала координат.
\par
Теперь докажем, что если в~интересном множестве~$M$ хотя~бы 301 точка, можно
одну из~них удалить так, что оно останется интересным.
Предположим противное: при удалении любой точки $a \in M$ найдется плоскость
$\alpha$, вне которой не~более 99~точек множества $M \setminus a$.
Значит, вне плоскости $\alpha$ находится ровно 100 (меньше не~может быть
из-за интересности) точек множества~$M$, одна из~них $a$.
Отметим эти 100 точек и~выберем одну из~неотмеченных точек $b$.
Для нее найдется плоскость~$\beta$, вне которой лежит точка~$b$ и~еще
99~точек.
Отметим те~из~этих ста точек, которые еще не~отмечены, выберем неотмеченную
точку~$c$ и~построим для нее аналогичным образом плоскость $\gamma$ и~отметим
точку $c$ и~еще не~отмеченные точки вне $\gamma$.
Заметим, что плоскости $\alpha, \beta, \gamma$ различны (поскольку
$b \in \alpha \setminus \beta$, $c \in (\alpha \cap \beta) \setminus \gamma$).
Кроме того, эти плоскости не~могут иметь общей прямой, поскольку пересечение
плоскостей $\alpha$ и~$\beta$ не~содержится в~плоскости~$\gamma$.
Итак, они имеют не~более одной общей точки.
Однако все неотмеченные точки множества~$M$ лежат в~их~пересечении.
Отсюда видим, что $M$ содержит ровно 301 точку: одну неотмеченную точку~$p$
и~300 отмеченных по~одному разу.
Отсюда видим, что множество~$M$ содержит по~100 отмеченных точек на~каждой
из~прямых $\alpha \cap \beta$, $\beta \cap \gamma$, $\alpha \cap \gamma$
и~точку $p = \alpha \cap \beta \cap \gamma$.
Несложно видеть, что в~этом случае множество $M \setminus p$ интересное~---
противоречие.
\endproblem
% $problem-source: IMC 2014
