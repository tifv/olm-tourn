\problem
100 аборигенов смогли переправиться в~двухместной лодке с~левого берега Лимпопо
на~правый.
Изначально каждый об одном или нескольких из~остальных слышал слух, что они~---
вирусоносители лихорадки Эболы.
С~тем, о~ком он~такое слышал, абориген вместе в~лодку не~садился.
На~левом берегу распространение слухов запрещено, зато достигнув правого
берега, аборигены высаживаются, все обмениваются всеми слухами, и~только потом
лодка возвращается.
О~каком наименьшем числе аборигенов могло совсем не~быть слухов, что они~---
вирусоносители?
\solution
\emph{Ответ:} 1.
\par
Пусть о~каждом есть слух.
Рассмотрим последний рейс~--- на~тот берег плывут аборигены $A$ и~$B$.
Тогда предыдущим рейсом один из~них, пусть $A$, приплыл с~левого берега и~слышал
там слух о~$B$.
Поэтому они не~могут плыть вместе.
Противоречие.
\par
Приведем пример, когда ровно об~одном только аборигене $A$ нет слухов.
Для этого уберем его пока в~сторонку.
Из~оставшихся $99$ выберем двоих~--- $B$ и~$C$, а~осталных пронуменуем $D_{1}, \ldots, D_{97}$.
Пусть $D_{1}, \ldots, D_{97}$ слышали слух друг о~друге по~циклу.
Пусть также $B$ слышал о~$D_1$ и~$C$ о~$D_2$.
Теперь $B$ отвозит всех с~$D_{97}$ по~$D_2$ на~правый берег, а~сам возвращается.
Потом $C$ отвозит $D_1$ и~возвращается за~$B$.
Теперь все кроме $A$ на правом берегу.
Пусть $A$ слышал слух о~$B$ и~о~$C$, и~только о~них.
Тогда за~ним вернется $D_1$.
\endproblem
% $problem-source: А.\,Шаповалов
