\ifsolution
\begin{figure}\centering
    \jeolmfigure[width=0.5\textwidth]{solution}
    \caption{к решению задачи \ref{solution:2012/ru/regatta/junior/geomt/3}}
    \label{fig:solution:2012/ru/regatta/junior/geomt/3}
\end{figure}%
\fi % \ifsolution

\problem
В трапеции $ABCD$ с основаниями $AB$ и $CD$ оказалось, что $AD = DC = CB < AB$.
Точки $E$ и $F$ на сторонах $CD$ и $BC$ соответственно таковы, что
$\angle ADE = \angle AEF$.
Докажите, что $4 CF \leq BC$.
\solution
\label{solution:2012/ru/regatta/junior/geomt/3}%
Рис.~\ref{fig:solution:2012/ru/regatta/junior/geomt/3}.
Так как трапеция равнобедренная, то $\angle C = \angle D$.
Заметим, что
$\angle CEF + \angle FEA + \angle AED = 180^\circ$, а также
$\angle EDA + \angle DAE + \angle AED = 180^\circ$.
Вычитая одно равенство из другого получаем $\angle CEF = \angle DAE$.
Следовательно, треугольники $CEF$ и $DAE$ подобны.
Обозначим длину отрезка $CE$ через $x$, а $ED$~--- через $y$.
Тогда из свойств подобных треугольников $CF / CE = ED / DA$.
То есть
\[
    CF = \frac{ED \cdot CE}{DA} = \frac{x \cdot y}{x + y}
\]
(мы воспользовались тем, что $DA = CD = x + y$).
Утверждение задачи равносильно неравенству
\[
    4 \leq \frac{CB}{CF} = \frac{(x + y)^2}{xy}
.\]
Для доказательства этого неравенства проведём следующую цепочку равносильных
преобразований:
\[
    4 \leq \frac{(x + y)^2}{xy}
\Leftrightarrow
    4 x y \leq x^2 + 2 x y + y^2
\Leftrightarrow
    x^2 - 2 x y + y^2 \geq 0
\Leftrightarrow
    (x - y)^2 \geq 0
.\]
\endproblem
