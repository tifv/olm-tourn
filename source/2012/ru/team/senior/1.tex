\problem\problemscore{3}
На доске написано несколько различных чисел, причем известно, что среди любых
трех чисел, написанных на доске, есть два числа, сумма которых также написана
на доске.
Какое наибольшее количество чисел может быть на доске?
\solution
Во-первых заметим, что можно добавить к числам 0, если его там нет,
и условие сохранится.
Во вторых, заметим, что можно отдельно изучать положительные числа,
написанные на доске, и отдельно для них тоже условие выполняется.
Рассмотрим только положительные числа.
Пусть наибольшее из них число~--- это $A$.
Пусть $B$~--- второе по величине положительное число.
Рассмотрим тройку $(A, B, x)$, где $x$ любое другое имеющееся на доске
положительное число.
Ясно, что тогда на доске не может быть сумма $A + B$ и $A + x$~--- они больше,
чем $A$.
Значит на доске есть число $B + x$, и, так как оно больше, чем $B$~--- это $A$.
Значит любое другое положительное число на доске, кроме $A$ и $B$~---
это $A - B$.
Отсюда ясно, что положительных чисел на доске не больше трех.
Аналогично доказывается, что отрицательных чисел на доске тоже не больше трех.
Значит всего чисел на доске не более семи.
\emph{Пример:} $-3, -2, -1, 0, 1, 2, 3$.
\emph{Ответ:} 7.
\endproblem
