\ifsolution
\begin{figure}\centering
    \jeolmfigure[width=0.5\textwidth]{solution}
    \caption{к задаче \ref{solution:2012/pers-geomt/junior/1}}
    \label{fig:solution:2012/pers-geomt/junior/1}
\end{figure}
\fi % \ifsolution

\problem
В параллелограмме $ABCD$ биссектриса угла $A$ пересекает сторону $BC$ в ее
середине $M$.
Так же известно, что $\angle BDC = 90^\circ$.
Найдите углы параллелограмма $ABCD$.
\solution
\label{solution:2012/pers-geomt/junior/1}%
Рис.~\ref{fig:solution:2012/pers-geomt/junior/1}.
Так как $BM \parallel AD$, то углы $DAM$ и $AMB$ равны как накрест лежащие.
Так как $AM$~--- биссектриса угла $A$, то получаем, что
$\angle BAM = \angle AMB$.
Значит, треугольник $ABM$ равнобедренный.
Следовательно, $AB = BM$.
Так как $ABCD$~--- параллелограмм, то $AB = CD$.
Проведем медиану $DM$ треугольника $BCD$.
Так как это медиана, проведенная к гипотенузе, то она равна половине стороны
$BC$, то есть $DM = MC$.
Имеем $DM = MC = MB = AB = CD$.
Значит, треугольник $CDM$ равносторонний.
Следовательно, угол $C$ равен $60^\circ$.
\emph{Ответ:}
$\angle A = \angle C = 60^\circ$,
$\angle B = \angle D = 120^\circ$.
\endproblem
