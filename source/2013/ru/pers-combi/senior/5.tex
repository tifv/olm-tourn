\problem
В колоде из 52 карт потерян туз пик.
Про любую пару карт одной масти или одного достоинства известно, сколько карт
между ними лежит.
Докажите, что этого достаточно, чтобы узнать пару, составленную из самой
верхней и самой нижней карт колоды.
\solution
\emph{Указание.}
Карты~--- точки на числовой прямой (согласно номерам), между некоторыми
известно расстояние.
Фигура из трех точек с известными расстояниями~--- жесткая.
Значит, наборы по мастям и по достоинствам~--- жесткие.
Набор из 4 карт двух мастей и двух достоинств~--- четырехугольник с жесткими
сторонами.
Он нежесткий на прямой только если это параллелограмм.
Если он жесткий, то будет жестким объединение некоторой масти и достоинства,
и~тогда вся конструкция жесткая.
А если каждая четверка нежесткая, то масти и достоинства образуют шарнирную
конструкцию из параллелограммов, которую можно <<выпрямить>> двумя способами.
При каждом способе номера карт вычисляются и идут с шагом 1, в том числе
координата потерянного туза пик.
Выпрямим так, чтобы туз пик~--- назовем его $Z$~--- оказался между крайними
картами.
Тогда его координата должна совпасть с координатой некоторой карты $G$.
Пару $G,Z$ дополним до параллелограмма $GFZH$.
Так как при выпрямлении $G$ и $Z$ совместились, то $GFZH$~--- ромб.
Но тогда при другом выпрямлении совместятся $F$ и $H$.
Противоречие.
\endproblem
% $problem-source: А.\,Шаповалов
