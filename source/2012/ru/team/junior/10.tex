% nospell begin
\def\ifsolutiondefined{%
    \csname ifsolution:2012/ru/team/junior/10\endcsname}
\expandafter\providecommand
    \csname ifsolution:2012/ru/team/junior/10\endcsname
    {\iffalse}
\def\definesolution{%
    \expandafter\gdef
    \csname ifsolution:2012/ru/team/junior/10\endcsname
    {\iftrue}}
% nospell end

\problem\problemscore{12}
% duplicated in [2012/ru/team/senior/6]
В правильном семиугольнике $ABCDEFG$ стороны равны 1.
Диагонали $AD$ и $CG$ пересекаются в точке $H$.
Докажите, что $FH = \sqrt{2}$.
\solution
\label{solution:2012/ru/team/junior/10}%
\ifsolutiondefined
См.~решение задачи \ref{solution:2012/ru/team/senior/6}.
\else
\definesolution
\begin{figure}\centering
    \jeolmfigure[width=0.5\textwidth]{solution}
    \caption{к решению задачи \ref{solution:2012/ru/team/junior/10}}
    \label{fig:solution:2012/ru/team/junior/10}
\end{figure}
Рис.~\ref{fig:solution:2012/ru/team/junior/10}.
Обозначим угол $x = \pi / 7$.
Заметим, что $ABCD$ и $ABCG$~--- равнобедренные трапеции,
так что $ABCH$~--- параллелограмм, а следовательно и ромб.
Таким образом, $AH = HC = 1$.
Пусть лучи $GA$, $CB$ пересекаются в точке $M$.
Считая вписанные углы в семиугольнике получаем,
что $\angle AMC = 3 x$, $\angle ACM = x$,
так что треугольник $ACM$ равнобедренный, $AC = CM$.
Далее, треугольники $HCM$, $HAF$ равны по двум сторонам и
углу $2 x$ между ними, так что $HF = HM$ и
$\angle FHM = \angle FHA + \angle AHM = \angle CHM + \angle AHM = 5 x$,
так что можно построить правильный семиугольник $FHMLKJI$.
Пусть отрезки $MJ$, $HK$ пересекаются в точке $N$.
Тогда $NM = HM$
(это равенство аналогично $AH = GA$, но в новом семиугольнике),
$GM = CM$ из симметрии,
$\angle NMG = \angle NMH - \angle GMH = \angle GMC - \angle GMH = \angle HMC$.
Значит, треугольники $NGM$, $HCM$ равны по двум сторонам и углу между ними,
так что $NG = HC = 1$ и $\angle NGM = \angle HCM = 2 x = \pi - \angle FGA$.
Таким образом, $G$ есть середина $FN$ и $FN = 2$.
Отношения $1 : FH$ и $FH : 2$ оказываются соответственными отношениями в
двух наших правильных семиугольниках, так что они равны, что и требовалось
доказать.
\par
\emph{Другое решение.}
Как и в первом решении, начнем с наблюдения, что $ABCH$~--- параллелограмм,
пусть $U$~--- общая середина его диагоналей, $V$~--- середина $BF$.
Тогда $FH = 2 UV$ по теореме о средней линии треугольника $BHF$.
Нам понадобится такой общий факт:
если $X_1$, $X_2$, $X_3$, $X_4$ любые 4 точки плоскости,
$M_{ij}$~--- середины соответствующих отрезков $X_iX_j$, то
\[
    4 \cdot {M_{13} M_{24}}^2
=
    {X_1 X_2}^2 + {X_2 X_3}^2 + {X_3 X_4}^2 + {X_4 X_1}^2 - {X_1 X_3}^2 - {X_2 X_4}^2
.\]
Установить его можно, например, рассмотрев параллелограммы Вариньона
$M_{ij} M_{jk} M_{kl} M_{li}$ для перестановок
$(i, j, k, l) = (1, 2, 3, 4), (1, 3, 2, 4), (1, 2, 4, 3)$,
написав для них равенства параллелограмма и вычтя из суммы двух третье.
Или применяя декартовы координаты.
Рассматривая четырехугольник $X_1 X_2 X_3 X_4 = ABCF$ мы видим, что
$X_3 X_4 = X_4 X_2$, $X_1 X_3 = X_1 X_4$,
так что $4 UV^2 = AB^2 + BC^2 = 2$, что и требовалось.
\fi % \ifsolutiondefined
\endproblem
