\ifsolution
\begin{figure}\centering
    \jeolmfigure[width=0.4\textwidth]{solution}
    \caption{к задаче \ref{solution:2013/ru/regatta/senior/geomt/1}.}
    \label{fig:solution:2013/ru/regatta/senior/geomt/1}
\end{figure}
\fi % \ifsolution

\problem
В равнобедренном прямоугольном треугольнике $ABC$ на гипотенузе $AC$ выбрали
точки $M$ и $N$ ($M$ между $A$ и $N$).
Оказалось, что $\angle MBN = 45^\circ$.
Докажите, что из отрезков $AM$, $MN$ и $NC$ можно сложить прямоугольный
треугольник.
\solution
\label{solution:2013/ru/regatta/senior/geomt/1}%
Рис.~\ref{fig:solution:2013/ru/regatta/senior/geomt/1}.
Точка $X$, симметричная $A$ относительно $BM$, совпадает с точкой, симметричной
$C$ относительно $BN$.
Треугольник $XMN$~--- искомый.
\par
\begin{figure}\centering
    \jeolmfigure[width=0.4\textwidth]{another-solution}
    \caption{к задаче \ref{solution:2013/ru/regatta/senior/geomt/1}
        (другое решение).}
    \label{fig:solution:2013/ru/regatta/senior/geomt/1/another}
\end{figure}
\emph{Другое решение.}
Рис.~\ref{fig:solution:2013/ru/regatta/senior/geomt/1/another}.
Построим точку $D$ по ту же сторону от $AC$, что и точка $B$, такую, что
$CD \perp AC$, $CD = AM$.
Тогда, поскольку $\angle BAM = \angle BCD = 45^\circ$, треугольники
$\triangle ABM$ и $\triangle BCD$ равны.
Следовательно,
$\angle DBN = \angle DBC + \angle CBN = \angle MBA + \angle CBN = 45^\circ$.
Таким образом, равны треугольники $\triangle MBN$ и $\triangle DBN$ и
$DN = MN$.
Прямоугольный треугольник $\triangle DNC$ имеет нужные стороны.
\endproblem
% $problem-source: фольклор
