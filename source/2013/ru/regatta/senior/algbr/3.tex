\problem
У квадратного трехчлена $a x^2 + b x + c$ коэффициент $a \neq 0$.
Докажите, что при некотором иррациональном $x$ значение
$a x^2 + b x + c$~--- рационально.

\solution
Если $a < 0$, сменим все знаки.
Обозначим полученный трехчлен через $P(x)$.
Выберем достаточно большое рациональное число $r$, чтобы у уравнения
$P(x) - r$ были два корня: $x_1$ и $x_2$.
Тогда по теореме Виета $P(x) - r = a (x - x_1) (x - x_2)$.
Если хотя бы один из корней~--- иррационален (скажем, $x_1$), то
$P(x_1) = r$~--- рационально, то есть $x_1$~--- искомое.
Пусть оба корня~--- рациональны.
Разберем два случая.
\\
\emph{Случай 1:} $a$~--- иррационально.
Левая часть принимает все положительные значения, в частности, значение 1.
Тогда уравнение $a (x - x_1) (x - x_2) = 1$ имеет корень, скажем $x_0$.
Если бы  $x_0$ был рациональным, то и $a = 1 / (x_0 - x_1) (x_0 - x_2)$ было бы
рациональным.
Значит, $x_0$~--- иррационально.
И тогда $P(x_0) = 1 + r$~--- рационально, то есть $x_0$~--- искомое.
\\
\emph{Случай 2:} $a$~--- рационально.
Выделим в выражении $a (x - x_1) (x - x_2)$ полный квадрат, получим
$a (x - x_3)^2 + b$, где $x_3$ и $b$~--- рациональны.
Если в него подставить иррациональное число $x = x_3 + \sqrt{2}$, получим
рациональное число $2 a + b$.
Тогда и $P(x) = 2 a + b + r$~--- рациональное. 
\endproblem
% $problem-source: С.\,Когаловский, А.\,Шаповалов
