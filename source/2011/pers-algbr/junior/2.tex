\problem
Какое наименьшее количество натуральных делителей может иметь число
$p^2 + 2011$ при простом $p$?

\solution
\emph{Ответ:} $6$.
Заметим, во-первых, что число
$2^2 + 2011 = 2015 = 5 \cdot 13 \cdot 31$
имеет $8$ натуральных делителей.
Пусть $p$ нечетно.
Тогда $p^2 + 2011$ обязательно делится на $4$, но не делится на $8$.
Следовательно, $p^2 + 2011 = 4 m = 2^2 m$, где $m$~--- нечетно.
Отсюда число делителей не меньше $6$, причем ровно $6$ оно будет, если
$m$~--- простое.
Это будет так, например, при $p = 5$; тогда $5^2 + 2011 = 2036 = 4 \cdot 509$.

\endproblem
