\problem
\emph{Радикалом} $r(n)$ натурального числа $n$ назовем произведение всех его
различных простых делителей.
Например, $r(2000) = 10$, $r(2011) = 2011$.
Последовательность натуральных чисел задается первым членом $a_1$ и
соотношением $a_{n + 1} = a_n + r(a_n)$ при $n \geq 1$.
Докажите, что в ней встретится миллион подряд идущих членов, образующих
арифметическую прогрессию.

\solution
Предположим противное.
Обозначим миллион буквой $M$.
С одной стороны, по индукции очевидно получаем $a_n \leq 2^{n - 1} a_1$.
С другой стороны, в последовательности $b_n = r(a_n)$ каждый следующий член
кратен предыдущему, и либо она принимает какое-то значение $M$ раз подряд
(что нам и нужно), либо $b_{n + M} > b_{n}$ для всех $n$.
Во втором случае получаем, что
\[
    b_{2 n M}
=
    b_M \cdot (b_{2 M} / b_M) \cdot (b_{3 M} / b_{2 M})
    \cdot \ldots \cdot
    (b_{2 n M} / b_{(2 n - 1) M})
,\]
где все $2 n$ сомножителей в правой части попарно взаимно просты и строго
больше 1.
Но тогда их произведение, очевидно, больше, чем $(2 n)! > n^n$.
Отсюда 
\[
    2^{2 n M} a_1
>
    a_{2 n M}
\geq
    b_{2 n M}
>
    n^n
,\]
что при больших $n$ (например $n = (4 a_1)^M$), очевидно, не так.
Противоречие.

\endproblem
