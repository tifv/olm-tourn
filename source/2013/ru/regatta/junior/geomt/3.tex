\ifsolution
\begin{figure}\centering
    \jeolmfigure[width=0.5\textwidth]{solution}
    \caption{к решению задачи \ref{solution:2013/ru/regatta/junior/geomt/3}.}
    \label{fig:solution:2013/ru/regatta/junior/geomt/3}
\end{figure}
\fi % \ifsolution

\problem
В неравнобедренную трапецию $ABCD$ вписана окружность с центром $O$.
Точка $M$~--- середина более длинного основания $AB$.
Прямая $MO$ пересекает отрезок $CD$ в точке $F$.
$E$~--- точка касания $CD$ и окружности.
Докажите, что $DE = CF$ тогда и только тогда, когда $AB = 2 CD$.
\solution
\label{solution:2013/ru/regatta/junior/geomt/3}%
Рис.~\ref{fig:solution:2013/ru/regatta/junior/geomt/3}.
Пусть $X$~--- точка касания отрезка $AB$ и окружности, $T$~--- точка
пересечения $AD$ и $BC$.
Будем для определенности считать, что $AX > BX$, тогда
$FE = XM = \frac{AX - BX}{2} = \frac{TA - TB}{2}$.
В последнем равенстве мы воспользовались тем, что расстояние от вершины до
точки касания с вписанной окружности в треугольнике равно разности
полупериметра и соответствующей стороны.
С другой стороны $CE - DE = TD - TC$.
На этот раз мы воспользовались формулой для расстояния от вершины до точки
касания с вневписанной окружностью.
Теперь отметим, что $DE = CF$ тогда и только тогда, когда $FE = CE - DE$, что
эквивалентно равенству $\frac{TA - TB}{2} = TD - TC$.
В силу теоремы Фалеса это равносильно тому, что $DC$ есть средняя линия
треугольника $TAB$.
\endproblem
% $problem-source: Polish NO 1993 
