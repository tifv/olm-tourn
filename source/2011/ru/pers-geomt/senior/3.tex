\ifsolution
\begin{figure}\centering
    \jeolmfigure[width=0.5\textwidth]{solution}
    \caption{к задаче \ref{solution:2011/pers-geomt/senior/3}.}
    \label{fig:solution:2011/pers-geomt/senior/3}
\end{figure}
\fi % \ifsolution

\problem
Точки $A$, $B$, $C$, $D$ лежат на окружности в указанном порядке, причем $AB$ и
$CD$ не параллельны.
Длина дуги $AB$, содержащей точки $C$ и $D$, в два раза больше длины дуги $CD$,
не содержащей точек $A$ и $B$.
Точка $E$ задается условиями $AC = AE$ и $BD = BE$.
Оказалось, что перпендикуляр из точки $E$ на прямую $AB$ проходит через
середину дуги $CD$, не содержащей точек $A$ и $B$.
Найдите $\angle ACB$.
\solution
\label{solution:2011/pers-geomt/senior/3}%
Рис.~\ref{fig:solution:2011/pers-geomt/senior/3}.
Как известно, для точек $X$, $Y$ на плоскости любая прямая, перпендикулярная
$XY$, задается условием
$\{Z \colon ZX^2 - ZY^2 = \const\}$.
Пусть $P$~--- середина $CD$.
Тогда условие $EP \perp AB$ влечет равенство
$P A^2 - P B^2 = E A^2 - E B^2 = A C^2 - B D^2$.
Значит, $A P^2 - A C^2 = B P^2 - B D^2 = B' P^2 - B' C^2$, где точка $B'$
симметрична $B$ относительно серединного перпендикуляра к $CD$
(в силу условия о непараллельности $B' \neq A$).
Отсюда получаем, что прямые $AB'$ и $PC$ перпендикулярны.
Для направленных углов между прямыми имеем 
\begin{gather*}
    \pi / 2
=
    \angle (AB', PC)
=
    \angle (AB', B'C) + \angle (B'C, CP)
=\\=
    \angle (AB, BC) + \angle (DP, DB)
=
    \angle (AB, BC) + \angle (AP, AB)
=
    \angle (AP, BC)
.\end{gather*}
Таким образом, прямые $AP$ и $BC$ перпендикулярны.
Это значит, что разность угловых мер дуг $AB$ и $PC$
(не содержащих точек $P$ и $A$ соответственно) равна $\pi$.
Если обозначит дугу $PC$ за $x$, то получаем, что дуга $AB$
(не содержащая $P$) равна $2 \pi - 4 x$, откуда $5 x = \pi$, $x = \pi / 5$,
дуга $AB$ равна $6 \pi / 5$, угол $\angle ACB$ равен $3 \pi / 5$.
\emph{Ответ:} $3 \pi / 5$.
\endproblem
% $problem-source: Czech-Polish-Slovak Match 2011
