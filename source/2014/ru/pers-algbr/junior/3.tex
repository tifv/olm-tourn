\problem
Найдите все совершенные числа, в~разложение которых на~простые множители каждое
простое входит в~нечетной степени.
(Напомним, что натуральное число называется \emph{совершенным,} если оно равно
сумме всех своих натуральных делителей, меньших самого числа~---
например, $28 = 1 + 2 + 4 + 7 + 14$.)
\solution
Ответ: $n = 6$.
\par
Пусть $p$~--- наименьший простой делитель нашего числа $n$, $2 k + 1$~---
степень его вхождения в~разложение $n$ на~простые множители.
Разобьем все делители $n$ на~пары
$(x, px), (p^2 x, p^3 x), \ldots, (p^{2k}x, p^{2 k + 1}x)$,
где $x$~--- делитель, не~кратный $p$.
Сумма в~каждой паре кратна $p + 1$, а~значит, и~общая сумма делителей~---
равная $2n$~--- кратна $p + 1$.
Это возможно только при $p = 2$ (иначе $p + 1$ взаимно просто с~$n$ в~силу
минимальности $p$), так что $n$ делится на~$p = 2$ и~на~$p + 1 = 3$, то~есть
$n = 6 k$.
Если $k > 1$, то~сумма делителей $n$ не~меньше, чем
$1 + k + 2 k + 3 k + 6 k = 12 k + 1 > 2 n$~---- противоречие.
Значит, $k = 1$ и~$n = 6$.
Ясно, что 6 подходит.
\endproblem
% $problem-source: IMC 2014
