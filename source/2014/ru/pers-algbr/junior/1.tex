\problem
Электричка Москва--Петушки проходит начальный путь~--- от~Курского вокзала
до~Дрезны~--- втрое дольше, чем от~Леоново до~Петушков.
При этом от~Дрезны до~Петушков она идёт вдвое быстрее, чем от~Курского вокзала
до~Леоново.
Во~сколько раз время пути от~Курского вокзала до~Петушков больше, чем от~Дрезны
до~Леоново?
\solution
\emph{Ответ:} в~пять раз.
\par
Обозначим время пути от~Москвы до~Петушков через 1
(одна Метафизическая Единица Времени).
Тогда, если время от~Москвы до~Дрезны есть $x$ МЕВ, от~Леоново до~Петушков~---
$y$ МЕВ, имеем $x = 3 y$, $1 - y = 2 (1 - x) = 2 (1 - 3 y)$, отсюда
$5 y = 1$, $y = 1 / 5$, $x = 3 / 5$, $1 - x - y = 1 / 5$~--- время от~Дрезны
до~Леоново (в~частности, мы~видим, что Дрезна по~дороге от~Москвы к~Петушкам
встречается раньше Леоново.)
\endproblem
% $problem-source: South Africa 2014
% http://www.petuschki.de/englisch/erechts.php
