\problem\problemscore{6}
\emph{Узлом} назовем точку, обе координаты которой~--- целые числа.
Внутри треугольника $ABC$ с вершинами в узлах расположено ровно $n > 0$ узлов.
Какое наибольшее число узлов может находиться на стороне $BC$?
\solution
\emph{Ответ:} $2 n + 1$.
\par
\emph{Оценка.}
Пусть на стороне $BC$ находится $m$ узлов.
Тогда они делят сторону на $m + 1$ равных отрезков
(если бы отрезки были бы не равные, то узлов на стороне $BC$ было бы больше чем
$m$).
Обозначим через $\overline{e} = \overline{BC} / (m + 1)$, то есть
$BC = (m + 1) \cdot |\overline{e}|$.
Рассмотрим среди узлов внутри треугольника узлы с наименьшим расстоянием до
$BC$.
Проведем через них прямую $l$ параллельно $BC$.
Пусть прямая $l$ пересекает $AB$ в точке $B'$, а $AC$~--- в точке $C'$.
Ближайший узел к $B'$ среди расположенных внутри треугольника $ABC$ и на прямой
$l$ обозначим через $B_1$, а ближайший узел к $C'$ среди расположенных внутри
треугольника $ABC$ и на прямой $l$ обозначим через $C_1$.
Тогда $B' B_1 \leq |\overline{e}|$, $B_1 C_1 \leq (n - 1) \cdot |\overline{e}|$
(так как если мы отложим от $B_1$ вектор $(n - 1) \cdot \overline{e}$,
то мы получим на этом векторе уже $n$ узлов, следовательно, точка $C_1$ должна
лежать внутри или на конце этого вектора)
и $C_1 C' \leq |\overline{e}|$.
Следовательно $B'C' \leq (n + 1) \cdot |\overline{e}|$.
С другой стороны, прямая $l$ должна лежать между средней линией
(может совпадать с ней) и стороной $BC$, так как если она лежит
<<выше>> средней линии, то можно отразить $A$ относительно $B_1$ и получить
узел, лежащий внутри треугольника $ABC$ и находящийся ближе к $BC$.
Следовательно $B'C' \geq BC / 2 = (m + 1) \cdot |\overline{e}| / 2$.
Совмещая полученные неравенства, получаем
\(
    (m + 1) \cdot |\overline{e}| / 2
\leq
    B' C'
\leq
    (n + 1) \cdot |\overline{e}|
\).
Значит $m \leq 2 n + 1$.
\par
\emph{Пример:} $A(0; 2)$; $B(0; 0)$; $C(2 n + 2; 0)$.
На стороне $BC$ ровно $2 n + 1$ узел, а внутри треугольника ровно $n$ узлов.
\endproblem
