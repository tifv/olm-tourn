\problem
Произведение всех натуральных делителей натурального числа~$n$ (включая $n$)
оканчивается на~120 нулей.
На~сколько нулей может оканчиваться число~$n$?
(Перечислите все варианты и~докажите, что других нет.)
\solution
\emph{Ответ:} $1$, $2$, $3$, $4$, $5$.
\par
Квадрат произведения всех делителей $n$ (он~оканчивается на~240 нулей)
равен $n^k$, где $k$~--- количество делителей $n$.
Это можно понять, разбивая делители на~пары $d \cdot (n / d)$.
Таким образом, если $n$ оканчивается на~$s$ нулей, то~$k s = 240$.
Заметим, что $k$ кратно $s+1$ (скажем, если $n$ кратно $2^s$, но~не~кратно
$2^{s+1}$, то~все делители разбиваются на~группы $x, 2x, \ldots, 2^s x$,
где $x$ нечетно, аналогично, если $n$ кратно $5^s$, но~не~$5^{s+1}$.)
Кроме того, $k \geq (s+1)^2$, поскольку каждый из~сомножителей 2, 5 входит
в~разложение $n$ хотя~бы в~$s$-ой степени.
Таким образом, $240$ делится на~$s (s + 1)$, но~$240 \geq s (s + 1)^2$.
Это возможно при $s = 1, 2, 3, 4, 5$.
\par
\emph{Примеры:} $2 \cdot 5^{119}$, $2^2 \cdot 5^{39}$, $2^3 \cdot 5^{19}$,
$2^4 \cdot 5^{11}$, $2^5 \cdot 5^{7}$. 
\endproblem
% $problem-source: по мотивам Кенгуру 2013
