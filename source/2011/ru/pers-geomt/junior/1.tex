\ifsolution
\begin{figure}\centering
    \jeolmfigure[width=0.5\textwidth]{solution}
    \caption{к решению задачи \ref{2011/ru/pers-geomt/junior/1:solution}.}
    \label{2011/ru/pers-geomt/junior/1:solution:fig}
\end{figure}%
\fi % \ifsolution

\problem
В треугольнике $ABC$ проведена медиана $A A_1$ и на ней отмечена точка $M$~---
точка пересечения медиан.
Точка $K$ на стороне $AB$ такова, что $MK \parallel AC$.
Оказалось, что $AM = CK$.
Найдите угол $ACB$.
\solution
\label{2011/ru/pers-geomt/junior/1:solution}%
\emph{Ответ:} $\angle ACB = 90^\circ$.
\par
Рис.~\ref{2011/ru/pers-geomt/junior/1:solution:fig}.
Соединим точку $C$ с точкой $M$ и получим трапецию $AKMC$.
Эта трапеция~--- равнобедренная, так как по условию в ней равны диагонали.
Отсюда получаем, что $\angle ACM = \angle BAC$.
Продлим $CM$ до пересечения с $AB$ в точке $L$.
Так как $M$~--- точка пересечения медиан, то $CL$~--- медиана, и, значит,
$AL = CL = LB$, то есть медиана $CL$ равна половине стороны $AB$, что и
означает, что $\angle ACB = 90^\circ$.
\endproblem
