\problem
Есть $100$ одинаковых с виду монет.
Известно, что среди них ровно $4$ фальшивых, которые весят одинаково, но легче
настоящих.
Как найти за $2$ взвешивания на чашечных весах без гирь хотя бы $13$ настоящих?
\solution
Кладем на чаши по $29$ монет.
При неравенстве в тяжелой группе не более одной фальшивой монеты.
Кладем из неё по $14$ монет на чаши, в более тяжелой~--- все настоящие, при
равенстве: все $28$~--- настоящие.
При равенстве в первый раз среди оставшихся $42$ монет четное число фальшивых.
Добавим к одной чаше $13$ монет с другой и сравниваем с $42$ монетами.
Теперь на весах не менее $2$ фальшивых.
При равенстве на обеих по $2$ фальшивые, значит, оставшиеся $16$~--- настоящие.
Если легче $42$ монеты, то там либо $4$ фальшивые, либо две.
В обоих случаях переложенные $13$~--- настоящие.
Если $42$ монеты тяжелее, то это $42$ настоящие.
\endproblem
