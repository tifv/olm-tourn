\ifsolution
\begin{figure}\centering
    \jeolmfigure[width=0.5\textwidth]{solution}
    \caption{к решению задачи \ref{solution:2013/ru/pers-geomt/junior/4}.}
    \label{fig:solution:2013/ru/pers-geomt/junior/4}
\end{figure}
\fi % \ifsolution

\problem
В остроугольном неравнобедренном треугольнике $ABC$ точки $O$ и $H$~--- это
соответственно центр описанной окружности и точка пересечения высот.
Окружность $\omega_A$ симметрична описанной около треугольника $AOH$ окружности
относительно прямой $AO$.
Аналогично определяются окружности $\omega_B$ и $\omega_C$.
Докажите, что окружности $\omega_A$, $\omega_B$ и $\omega_C$ пересекаются в
одной точке, лежащей на описанной окружности треугольника $ABC$.
\solution
\label{solution:2013/ru/pers-geomt/junior/4}
Рис.~\ref{fig:solution:2013/ru/pers-geomt/junior/4}.
Обозначим через $X$ точку пересечения описанной окружности треугольника с
окружностью $\omega_A$.
Через $H_A$ и $H_B$ обозначим точки, симметричные $H$ относительно середин $OA$
и $OB$ соответственно.
Под \emph{направленным углом} между прямыми $\ell_1$ и $\ell_2$ будем понимать
угол на который надо повернуть прямую $\ell_1$, чтобы она стала параллельна
$\ell_2$.
Обозначать его будем $\angle (\ell_1, \ell_2)$.
Тогда, пользуясь простейшими свойствами направленных углов, получаем
$\angle(BX, OX) = \angle(BX, AX) + \angle(AX, OX)$.
Далее $\angle(BX, AX) = \angle (BC, AC) = \angle (AH, BH)$,
поскольку точки $A$, $B$, $C$, $X$ лежат на одной окружности и угол между
высотами равен углу между сторонами.
Кроме того, $\angle(AX, OX) = \angle(A H_A, O H_A) = \angle (OH, AH)$,
поскольку $A$, $O$, $X$ и $H_A$ лежат на $\omega_A$.
Следовательно,
\(
    \angle(BX, OX)
=
    \angle (AH, BH) + \angle (OH,AH)
=
    \angle (OH, BH)
=
    \angle (B H_B, O H_B)
\).
Отсюда следует, что точка $X$ лежит на $\omega_B$.
Аналогично проверяется, что точка $X$ лежит на $\omega_C$.
\par
\emph{Другое решение.} 
Будем считать, что описанная окружность треугольника $ABC$~--- это единичная
окружность с центром в нуле на комплексной плоскости.
Названия всех точек будем отождествлять с соответствующими комплексными
числами. Тогда $H = A + B + C$, точка $H_A$ симметричная $H$ относительно
середины $OA$ вычисляется по формуле $H_A = - B - C$.
Пусть $X$~--- это точка пересечения окружности $\omega_A$ и $\omega_B$.
То, что $X$ лежит на $\omega_A$ и $\omega_B$ означает, что
\[
    \frac{X - O}{X - A} : \frac{- B - C - O}{- B - C - A}
        \in {\mathbb R}
\text{\quad и\quad}
    \frac{X - O}{X - B} : \frac{- A - C - O}{- C - A - B}
        \in {\mathbb R}
.\]
Поделим одно двойное отношение на другое и получим, что
\[
    \frac{X - B}{X - A} : \frac{- C - B}{- C - A}
        \in {\mathbb R}
.\]
Следовательно, точки $X$, $A$, $B$ и $-C$ лежат на одной окружности, то есть
$X$ лежит на описанной окружности треугольника $ABC$ и отлична от $A$, $B$ и
$C$.
Из этого следует утверждение задачи.
\endproblem
% $problem-source: Ф.\,Бахарев по мотивам Iran TST 2013
