\ifsolution
\begin{figure}\centering
    \jeolmfigure[width=0.5\textwidth]{solution}
    \caption{к задаче \ref{solution:2012/ru/pers-geomt/senior/1}}
    \label{fig:solution:2012/ru/pers-geomt/senior/1}
\end{figure}
\fi % \ifsolution

\problem
Дан прямоугольный треугольник $ABC$ с прямым углом $C$.
На гипотенузе $AB$ отмечена её середина $M$.
На стороне $CB$ выбрана точка $Q$ такая, что $BQ : QC = 2 : 1$.
Докажите, что $\angle QAB = \angle QMC$.
\solution
\label{solution:2012/ru/pers-geomt/senior/1}%
Рис.~\ref{fig:solution:2012/ru/pers-geomt/senior/1}.
Пусть $A'$~--- точка, симметричная точке $A$ относительно $C$.
Тогда $BC$~--- медиана в треугольнике $AA'B$.
Так как точка пересечения медиан делит каждую из медиан в отношении $2 : 1$,
то $Q$ является точкой пересечения медиан треугольника $AA'B$.
Значит, медиана $A'M$ проходит через $Q$.
Так как медиана $BC$ является одновременно и высотой, то треугольник $AA'B$
равнобедренный.
Следовательно, $\angle QAB = \angle QA'B$.
$CM$~--- средняя линия в треугольнике $AA'B$, а значит, параллельна $A'C$.
Следовательно, угол $QA'B$ равен углу $QMC$ как накрест лежащий.
\endproblem
