\textsf{(5)}
Ученики школы посещают кружки.
Докажите, что можно несколько школьников принять в пионеры так, чтобы в каждом
кружке был хотя бы один пионер и для любого пионера нашелся кружок, в котором он
был бы единственным пионером.

\solution
\emph{Решение:}
Примем сначала в пионеры всех учеников этой школы.
Пусть нарушается одно из условий задачи, ясно, что это второе.
Тогда есть пионер такой, что в каждом кружке, в котором он занимается есть еще
один пионер.
Тогда разжалуем такого школьника из пионеров, заметим, что при этом кружков без
пионеров не появится.
Будем продолжать, пока это возможно, ясно, что в тот момент когда больше не
найдется пионера, на котором нарушается второе условие, мы получим требуемый
способ, ибо при наших операциях первое условие не нарушается.

