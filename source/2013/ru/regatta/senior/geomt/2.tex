\problem
Пусть $Q$~--- точка внутри выпуклого многогранника $M$.
Прямая $\ell$, проходящая через $Q$, пересекает поверхность многогранника в
точках $A$ и $B$.
Докажите, что для бесконечного множества направлений прямой $\ell$ верно, что
$AQ = BQ$.

\solution
Пусть $M'$~--- многогранник, симметричный $M$ относительно $Q$.
Многогранники $M$ и $M'$ пересекаются, ибо оба содержат точку $Q$.
При этом ни один из них не содержится в другом, поскольку объемы $M$ и $M'$
равны.
Следовательно, пересекаются поверхности многогранников $M$ и $M'$, и
пересечение поверхностей содержит отрезок.
Для любой точки $A$, лежащей на обеих поверхностях, прямая $\ell = AQ$ подходит
в условие задачи. 
\endproblem
% $problem-source: Putnam 1977 B4 Problem  
