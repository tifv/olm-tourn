\problem\problemscore{3}
В семье, кроме папы и мамы, двое детей.
Все зовут друг друга по имени.
Принято, обращаясь к своему ребенку или своему родителю, перечислять всех
упомянутых от младшего к старшему,
а в разговорах между детьми или между родителями~--- наоборот.
Лёша сказал Свете: <<Мы идем в театр: Настя, я и Володя>>.
Как зовут папу и маму?
\solution
\emph{Ответ:} Володя и Света.
\par
Допустим, папа~--- Лёша.
Тогда Володя младше, значит, Лёша перечислял от старшего к младшему.
Поэтому Настя старше Лёши, значит, Настя~--- мама.
Но перечислять от старшего к младшему Лёша мог только обращаясь к маме.
Противоречие. 
Значит, Лёша~--- сын, а папа~--- Володя.
Тогда Лёша перечислял от младшего к старшему, значит, обращался к маме.
Итак, мама~--- Света.
\endproblem
% $problem-source: А.\,Шаповалов
