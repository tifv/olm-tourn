% nospell begin
\def\ifsolutiondefined{%
    \csname ifsolution:2012/ru/pers-algbr/junior/1\endcsname}%
\expandafter\providecommand
    \csname ifsolution:2012/ru/pers-algbr/junior/1\endcsname
    {\iffalse}%
\def\definesolution{%
    \expandafter\gdef
    \csname ifsolution:2012/ru/pers-algbr/junior/1\endcsname
    {\iftrue}}%
% nospell end

\problem
% duplicated in [2012/ru/regatta/senior/algbr/2]
Найдите наименьшее натуральное число, дающее попарно различные остатки при
делении на $2$, $3$, $4$, $5$, $6$, $7$, $8$, $9$, $10$.
\solution
\label{solution:2012/ru/pers-algbr/junior/1}%
\emph{Ответ:} $1799$.
\ifsolutiondefined
См.~решение задачи \ref{solution:2012/ru/regatta/senior/algbr/2}.
\else
\definesolution
\begingroup\let\div\diamond
Изучим все числа, которые обладают таким свойством.
Обозначим рассматриваемое нами число с данным свойством через $N$, а остаток
при делении $N$ на $x$ через $N \div x$.
Пусть сначала $N$ четно.
Тогда $N \div 2 = 0$, $N \div 4 = 2$, $N \div 6 = 4$, $N \div 8 = 6$,
$N \div 10 = 8$.
Далее, получаем $N \div 3 = 1$, $N \div 5 = 3$, $N \div 7 = 5$, $N \div 9 = 7$.
Таким образом, если $N$ четное, то $N + 2$ делится на все числа от 2 до 10,
а значит $N \geq [2, 3, \ldots, 10] - 2 = 2518$.
Пусть теперь $N$ нечетно.
Тогда $N \div 2 = 1$, $N \div 4 = 3$, $N \div 6 = 5$, $N \div 8 = 7$,
$N \div 10 = 9$.
Далее, $N \div 3 = 2$ и $N \div 5 = 4$
(это следует соответственно из $N \div 6 = 5$ и $N \div 10 = 9$).
Остаток $N \div 9$ может быть равен только 8, так как $N \div 3 = 2$, а остатки
2 и 5 уже заняты.
Для остатка $N \div 7$ остаются 2 варианта: 0 и 6.
В первом случае $N + 1$ делится на все числа от 2 до 10, а значит
$N \geq [2, 3, \ldots, 10] - 1 = 2519$.
Во втором получаем, что $N + 1$ кратно $[2, 3, 4, 5, 6, 8, 9, 10] = 360$ и дает
остаток 1 от деления на 7.
Наименьшее такое число~--- это 1800.
\endgroup
\fi % \ifsolutiondefined
\endproblem
