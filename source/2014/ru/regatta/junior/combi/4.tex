\problem
Назовем девятизначное число \emph{хорошим}, если в~нем можно переставить одну
цифру на~\emph{другое} место и~получить девятизначное число, в~котором цифры
идут строго по~возрастанию.
Сколько всего хороших чисел?
\solution
\emph{Ответ:} $64$.
\par
В~хорошем 9-значном числе нет цифры 0, иначе после перестановки 0 встал~бы
на~1-е место и~число стало~бы 8-значным.
Значит, после перестановки число стало равным $M = 123456789$.
Тогда каждое хорошее число можно получить из~$M$ перестановкой одной цифры.
При перестановке цифры на~соседнее место фактически меняются местами 2~соседние
цифры.
Поскольку есть всего 8~пар соседних цифр, то~так можно получить 8 хороших
чисел.
Осталось рассмотреть перестановки на~несоседние места.
Крайние цифры 1 и~9 можно переставить на~7 несоседних мест, а~каждую из~7
остальных цифр~--- на~6 несоседних мест.
Итого, хороших чисел $8 + 2 \cdot 7 + 7 \cdot 6 = 64$.
\endproblem
% $problem-source: А.\,Шаповалов
