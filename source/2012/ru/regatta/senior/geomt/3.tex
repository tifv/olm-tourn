\problem
В тетраэдре $SABC$ радиусы описанных окружностей граней $SAB$, $SBC$ и $SAC$
равны $108$.
Радиус вписанной сферы этого тетраэдра равен $35$, а расстояние от вершины $S$
до её центра равно $125$.
Чему равен радиус описанной сферы этого тетраэдра?
\solution
\emph{Ответ:} $112{,}5$.
\par
Обозначим центр описанной сферы через $O$, а вписанной через $I$.
Опустим перпендикуляры из $O$ на грани тетраэдра.
Тогда их основаниями будут являться центры описанных окружностей этих граней.
Обозначим центр описанной окружности треугольника $SBC$ через $O_A$.
Тогда так как $O_AO \perp SBC$, то по теореме Пифагора имеем
$SO^2 = SO_A^2 + O_AO^2$.
Значит, $O_AO = \sqrt{SO^2 - SO_A^2}$.
Аналогично для других центров.
Значит, $O$ равноудалена от граней $SAB$, $SBC$ и $SAC$.
То есть существует сфера с центром в $O$ и касающаяся плоскостей
$SAB$, $SBC$ и $SCA$.
Но тогда эта сфера гомотетична вписанной сфере тетраэдра с центром в точке $S$.
Следовательно, что треугольник $SOO_A$ подобен треугольнику $SII_A$,
где $I_A$~--- точка касания вписанной сферы с гранью $SBC$.
Находим из подобия
\[
    SO
=
    SO_A \cdot SI / SI_A
=
    \dfrac{108 \cdot 125}{\sqrt{125^2 - 35^2}}
=
    \dfrac{27 \cdot 4 \cdot 5 \cdot 25}{5 \cdot 8 \cdot 3}
=
    112{,}5
.\]
\endproblem
