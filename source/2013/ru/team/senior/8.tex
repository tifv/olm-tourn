\problem\problemscore{8}
Дан граф, степени всех вершин которого не превосходят 7.
Оказалось, что его вершины нельзя покрасить правильным образом в 6 цветов.
Докажите, что в нем есть три попарно соединенные вершины.
\solution
Разобьем вершины на две группы так, чтобы сумма ребер внутри обеих групп была
минимальной.
Заметим, что максимальная степень внутри группы не больше 3-х, иначе перекинем
эту вершину в другую группу, и тогда количество ребер внутри групп уменьшится. 
\par
Так как граф не красится в 6 цветов, то индуцированный граф на одну из групп не
красится в 3 цвета. 
Рассмотрим эту группу, в ней максимальная степень 3, а следовательно, по
теореме Брукса в ней есть 4 попарно соединенные вершины.
\par
\emph{План доказательства без теоремы Брукса:}
Осталось доказать, что в графе с максимальной степенью три, который не красится
в три цвета, есть треугольник.
Пусть не так, рассмотрим минимальный контрпример, очевидно, в нем все степени
ровно 3.
\par
Так как граф не красится в три цвета, то и в два тоже.
Значит, в нем есть нечетный цикл, выберем из них самый короткий.
Пусть это $a_1 \cdots a_n$.
Ясно, что в нем нет диагоналей, иначе он был бы не самый короткий.
Пусть вершина $a_i$ помимо вершин $a_{i-1}$ и $a_{i+1}$ связана с вершиной
$b_i$. Удалим все вершины цикла, тогда граф красится в три цвета, и мы
попробуем его докрасить.
Если есть две вершины $b_i$ различного цвета, то несложно покрасить цикл
$a_1\cdots a_n$.
\par
Значит, все $b_i$ одного цвета.
Если какие-то две $b_i$ совпадают, то мы можем перекрасить эту вершину
(так как после удаление всех $a_j$ у нее осталась степень не больше 1).
Значит, они все различные, причем их степени не более $2$-х, поэтому попробуем
добавить ребро $b_i b_j$ так, чтобы не появился треугольник.
Если у нас получится, то существует раскраска с различными цветами $b_i$, а
далее мы сможем докрасить цикл $a_1 \cdots a_n$.
То есть при добавлении любого ребра $b_i b_j$ появляется треугольник, но это
невозможно, так как $n > 4$ (доказывается исследованием соседей $b_i$).
\endproblem
% $problem-source: Г.\,Ненашев
