% nospell begin
\def\ifsolutiondefined{%
    \csname ifsolution:2014/ru/team/junior/3\endcsname}%
\expandafter\providecommand
    \csname ifsolution:2014/ru/team/junior/3\endcsname
    {\iffalse}%
\def\definesolution{%
    \expandafter\gdef
    \csname ifsolution:2014/ru/team/junior/3\endcsname
    {\iftrue}}%
% nospell end

\problem\problemscore{5}
% duplicate of [2014/ru/team/junior/3]
В~строке в~некотором порядке записаны числа $1, 2, \ldots, n$.
Пару чисел назовем \emph{лункой}, если эти числа стоят рядом, либо между ними
есть только числа, меньшие каждого из~них.
Каково наибольшее количество лунок?
(Одно число может входить в~несколько лунок.)
\solution
\label{2014/ru/team/senior/2:solution}%
\emph{Ответ:} $2 n - 3$.
\ifsolutiondefined
См.~решение задачи \ref{2014/ru/team/junior/3:solution}.
\else
\definesolution
В~каждой лунке проведем стрелку от~меньшего числа к~большему.
Заметим, что из~каждого числа~$k$ в~одну сторону не~могут выходить 2~стрелки:
число на~конце более короткой окажется меньше $k$.
Итак, из~чисел $1, 2, \ldots, n - 2$ выходит не~более чем по~2 стрелки,
а~из~числа $n - 1$~--- не~более одной, итого~--- не~более $2 n - 3$.
Это достигается: поставим 1 в~середину, и~будем добавлять по~порядку числа
$2, 3, \ldots, n$ по~краям, строго чередуя края.
\fi % \ifsolutiondefined
\endproblem
% $problem-source: А.\,Лебедев, А.\,Шаповалов
