\problem
В клетках квадрата $10 \times 10$ расставили все натуральные числа от $1$ до
$100$, и вычислили произведения в каждой строке и в каждом столбце.
Может ли самое большое из произведений в строке делится на каждое из
произведений в столбце?

\solution
\emph{Ответ:} нет.
Рассмотрим $11$ простых чисел от $\sqrt{100}$ до $100$, например:
$11$, $13$, $17$, $19$, $23$, $29$, $31$, $37$, $41$, $43$, $47$.
Наше <<большое>> произведение должно делится на каждое из них, то есть каждое из
них присутствует в соответствующей строке.
Так как клеточек в строке всего $10$, то в число в одной из клеточек делится на
два из этих простых.
Но это невозможно, так как такое число будет больше $100$.

\endproblem
