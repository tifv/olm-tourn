\ifsolution
\begin{figure}\centering
    \jeolmfigure[width=0.5\textwidth]{solution}
    \caption{к задаче \ref{2014/ru/regatta/junior/geomt/1:solution}}
    \label{2014/ru/regatta/junior/geomt/1:solution:fig}
\end{figure}%
\fi % \ifsolution

\problem
В~треугольник $ABC$ вписана окружность с~центром~$O$.
Точка~$L$ лежит на~продолжении стороны~$AB$ за~вершину~$A$.
Проведенная из~$L$ касательная к~окружности пересекает сторону~$AC$ в~точке~$K$.
Найдите $\angle KOL$, если $\angle BAC = 50^\circ$.
\solution
\label{2014/ru/regatta/junior/geomt/1:solution}%
\emph{Ответ:} $65^\circ$.
\par
Рис.~\ref{2014/ru/regatta/junior/geomt/1:solution:fig}.
Пусть $\angle KLO = x$.
Тогда $\angle KLA = 2x$ и $\angle LKA = 50^{\circ} - 2x$.
$KO$ является биссектрисой угла, смежного с~$\angle LKA$, а~значит
$\angle AKO = 90^{\circ} - (50^{\circ} - 2x) / 2 = 65^{\circ} + x$.
Тогда
\(
    \angle KOL
=
    180^{\circ} - x - (50^{\circ} - 2x) - (65^{\circ} + x)
=
    65^{\circ}
\).
\endproblem
% $problem-source: перелицовка задачи Соросовской олимпиады 1995
