\section*{Правила Турнира}

% $build$matter:
% - - .[no-header]

Стоит отметить несколько принципов, которые легли в основу правил Турнира.

\begin{itemize}
\item
Спортивная составляющая~--- не на первом месте в Турнире.
Для нас главное~--- это плодотворные занятия математикой в психологически
комфортных условиях.
\item
При подсчете рейтинга все соревнования должны давать одинаковый вклад: если за
какое-то соревнование можно получить определенное количество баллов, то это же
количество баллов (теоретически) можно получить за любое из оставшихся
соревнований.
\item
Успех команды зависит от всех её членов.
Поэтому все участники во всех мероприятиях должны участвовать в той лиге, в
которой играет команда.
Это не очень удобно, но без этого подведение командных итогов получается
несправедливым.
В частности, варианты личных олимпиад в 8 и в 9 классе совпадают; аналогичное
верно для вариантов 10 и 11 класса.
\end{itemize}

Каждая команда состоит из четырех человек.

\subsection*{Правила подсчета рейтинга команды}

\paragraph{Рейтинг команды на турнире}
равен сумме рейтингов этой команды за каждое из пяти зачетных соревнований в
отдельности, деленной на 3 и округленной до ближайшего целого.
Для равномерного влияния каждого мероприятия устанавливается, что рейтинг по
каждому мероприятию~--- целое неотрицательное число от 0 до 60 включительно,
причем достижимые значения расположены почти непрерывно.
Таким образом, наибольший возможный рейтинг команды на Турнире равен 100,
наименьший~--- 0.
\paragraph{Рейтинг команды на регате}
определяется очень просто: это количество баллов, набранных командой на регате.
Регата состоит из 4 туров, в каждом из них по три задачи.
Каждая задача первого тура будет стоить 4 балла, второго тура~--- 5 баллов,
третьего тура~--- 6 баллов, четвертого~--- 5 баллов.
Итого за регату можно набрать 60 баллов.
\paragraph{Рейтинг команды на личной олимпиаде по алгебре и теории чисел}
равен сумме баллов всех членов команды, умноженной на $7/13$.
Если получившееся число нецелое (а с вероятностью около $12/13$ так и будет), то оно
округляется до ближайшего целого.
Каждая из 4 задач стоит 7 баллов, итого максимальная возможная сумма баллов
команды равна 112, после умножения~--- $60\frac{4}{13}$, а после
округления~--- 60.
\paragraph{Рейтинг команды на личной олимпиаде по комбинаторике и логике}
равен сумме баллов  всех членов команды, умноженной на 3.
Полное решение каждой из 5 задач стоит 1 балл, неполное~--- ноль баллов.
\paragraph{Рейтинг команды на личной олимпиаде по геометрии}
вычисляется так же, как рейтинг команды на личной олимпиаде по алгебре и теории чисел.
\paragraph{Рейтинг команды на командной устной олимпиаде}
равен количеству баллов, набранных командой на олимпиаде (см. правила устной
командной олимпиады на Турнире).
\paragraph{Первое, второе и третье места}
на Турнире присуждаются командам согласно рейтингу.
Четвертое место (похвальная грамота) присуждается командам на усмотрение
оргкомитета, методической комиссии и жюри турнира.

\subsection*{Правила соревнований}

\subsubsection*{Правила математической регаты}

Математическая регата~---
командное соревнование участников, которое проводится в четыре тура.
Каждый тур представляет собой коллективное письменное решение трех задач.
Условия задач выдаются на разноцветных листах формата A4 в начале тура.
Любая задача оформляется и сдается жюри на листе, на котором выдано условие,
причем каждая команда имеет право сдать только по одному варианту решения
каждой из задач.
Проведением регаты руководит \emph{Координатор}.
Он организует раздачу заданий и сбор листов с решениями; проводит разбор
решений задач и обеспечивает своевременное появление информации об итогах
проверки.
Проверка решений осуществляется жюри после окончания каждого тура.
Жюри состоит из трех комиссий, специализирующихся на проверке задач
\texttt{a}, \texttt{g} и \texttt{c} каждого тура соответственно.
Параллельно с ходом проверки, Координатор осуществляет для учащихся разбор
решений задач, после чего школьники получают информацию об итогах проверки.
После объявления итогов тура команды, не согласные с тем, как оценены их
решения, имеют право подать заявки на \emph{апелляции}.
В случае получения такой заявки, комиссия, проверявшая решение, осуществляет
повторную проверку и, после нее, может изменить свою оценку.
Если оценка не изменена, то сам процесс апелляции эта же комиссия осуществляет
после окончания всех туров регаты, но до окончательного подведения итогов.
В результате апелляции оценка решения может быть как повышена, так и понижена,
или же оставлена без изменения.
В спорных случаях окончательное решение об итогах проверки принимает
председатель жюри.

\subsubsection*{Правила личной письменной олимпиады}

Личная письменная олимпиада~---
личное соревнование участников Турнира.
В начале соревнования каждому участнику выдаются 4 задачи (\emph{вариант}).
Внутри одной лиги варианты, выданные всем участникам, одинаковы.
Участники решают и письменно оформляют решения задач.
По окончании времени, отведенного на решение и оформление задач (время
объявляется в начале олимпиады и одинаково внутри одной лиги), или ранее
работы сдаются дежурному по аудитории, после чего проверяются жюри.
Жюри оценивает написанное решение каждой задачи целым числом баллов от 0 до 7.
\emph{Результатом участника}
является сумма баллов за все задачи в варианте.

К личными письменным олимпиадам на Турнире относятся олимпиады
<<Алгебра и теория чисел>> и <<Геометрия>>.

\subsubsection*{Правила личной устной олимпиады}

Личная устная олимпиада~---
личное соревнование участников Турнира.
В начале соревнования участникам выдаются 5 задач (\emph{вариант}).
Внутри одной лиги варианты, выданные всем участникам, одинаковы.
Участники решают и рассказывают решения задач членам жюри в заранее
определенном помещении.
Каждый участник имеет право рассказывать решение каждой задачи не более трех
раз.
Есть два типа оценки:
<<$+$>>
(если решение верное)
и <<$-$>>
(если в решении есть недочеты, которые участник, рассказывавший решение, не
смог устранить после их обнаружения жюри и указания на них рассказчику).
Следовательно, по окончании олимпиады оценка за одну задачу может быть одного
из семи видов:
<<$\ $>>,
<<$-$>>,
<<$-\ -$>>,
<<$-\ -\ -$>>,
<<$-\ -\ +$>>,
<<$-\ +$>>,
<<$+$>>.
В последних трех случаях задача \emph{зачтена}.
\emph{Результатом участника} является количество зачтённых задач, независимо от
количества использованных попыток.

К личной устной олимпиаде на Турнире относится олимпиада
<<Комбинаторика и логика>>.

\subsubsection*{Правила командной устной олимпиады}

Командная устная олимпиада~--- командное соревнование участников Турнира.
В начале соревнования командам выдаются 7-10 задач (\emph{вариант}).
Внутри одной лиги варианты, выданные всем командам, одинаковы.
Возле номера каждой задачи указана её стоимость (в баллах).
Суммарная стоимость всех задач составляет 60 баллов.
Участники решают и рассказывают решения задач членам жюри в заранее
определенном помещении.
Каждый член команды имеет право рассказывать решения не более трех задач из
варианта; команда имеет право рассказывать решение каждой задачи сколько угодно
раз.
Каждая попытка заканчивается выставлением жюри оценки решения
(<<$+$>> или <<$-$>>).
Если задача рассказана верно с первой попытки, команда получает полную
стоимость этой задачи.
За каждую неудачную попытку стоимость данной задачи для данной команды
уменьшается на 1.
Как только стоимость задачи становится равной нулю, команда теряет право
рассказывать решение этой задачи (в целях всеобщей экономии сил и времени).
При решении разрешается пользоваться любой литературой, вычислительной
техникой; разрешается общение участников, состоящих в одной команде.
Запрещается общение с людьми, не состоящими в команде, кроме руководителя
команды, жюри и оргкомитета (все вопросы, не относящиеся к олимпиаде, нужно
решать через руководителя команды или оргкомитет).
Требуется, чтобы в любой момент хотя бы один представитель команды находился в
помещении, которое до начала команды фиксируется как \emph{штабная комната}
данной команды.
По окончании времени, отведенного на решение (время объявляется в начале
олимпиады и одинаково внутри одной лиги), жюри прекращает выслушивание решений,
кроме тех решений, о желании рассказать которые команда заявляет до или в
момент окончания времени.
\emph{Результатом команды} является количество набранных баллов.

