\problem
На смежные стороны $a$ и $b$ параллелограмма $ABCD$ опущены соответственно
высоты $h_a$ и $h_b$.
Известно, что $a + h_a = b + h_b$.
Рассмотрим отрезки $AB$, $AC$, $AD$, $BC$, $BD$, $CD$.
Какое наибольшее количество различных может быть среди них?

%Let $h_a$ and $h_b$ be the altitudes on the non-parallel sides $a$ and $b$ of a
%parallelogram $ABCD$, respectively.
%Suppose that $a + h_a = b + h_b$.
%How many different lengths can there be among segments
%$AB$, $AC$, $AD$, $BC$, $BD$, $CD$?

\solution
\label{solution:2011/regatta/junior/geomt/4}%
\emph{Ответ:} Три отрезка.
Заметим, что $a h_a = b h_b$, откуда и из условия задачи
(согласно теореме Виета) имеем либо $a = b$, и тогда это ромб, либо $a = h_b$,
и тогда это прямоугольник.
В первом случае совпадут две стороны, а во втором~--- две диагонали.

\endproblem
