\ifsolution
\begin{figure}\centering
    \jeolmfigure[width=0.5\textwidth]{solution-sequence}
\\[1ex]
    \jeolmfigure[width=0.5\textwidth]{solution-angles}
    \caption{к задаче \ref{2014/ru/regatta/junior/geomt/2:solution}}
    \label{2014/ru/regatta/junior/geomt/2:solution:fig}
\end{figure}%
\fi % \ifsolution

\problem
Из~точки $A_0$ под углом в~$7^\circ$ проведены черный и~красный лучи, после
чего построена ломаная $A_{0} A_{1} \ldots A_{20}$
(возможно, самопересекающаяся, но~все вершины различны), у~которой все звенья
имеют длину~1, все четные вершины лежат на~черном луче, а~нечетные~---
на~красном.
Вершина с~каким номером наиболее удалена от~вершины $A_0$?
\solution
\label{2014/ru/regatta/junior/geomt/2:solution}%
\emph{Ответ:} $A_{13}$.
\par
Рис.~\ref{2014/ru/regatta/junior/geomt/2:solution:fig}.
Заметим, что если $180^\circ - \angle A_{i} A_{i-1} A_0 < 90^\circ$,
то~$A_{i+1}$ находится дальше от~$A_0$, чем $A_{i-1}$, при этом треугольник
$A_{i-1} A_{i} A_{i+1}$ равнобедренный и
\[
    180^\circ - \angle A_{i+1} A_i A_0
=
    \angle A_i A_{i+1} A_0 + 7^\circ
=
    (180^\circ-\angle A_i A_{i-1} A_0) + 7^\circ
.\]
Следовательно, величина $180^\circ-\angle A_{i} A_{i-1} A_0$ за~одну итерацию
увеличивается на $7^\circ$ (а при $i = 2$ она равна $14^\circ$.)
Увеличиваться она может до тех пор, пока не станет больше $90^\circ$~---
после этого точки начнут приближаться к $A_0$.
Это произойдет при $i = 13$ ($7^\circ \cdot 13 = 91^\circ$).
Осталось заметить, что $A_0 A_{12} < A_0 A_{13}$.
Действительно, это равносильно тому, что 
\(
    84^\circ = \angle A_0 A_{13} A_{12}
<
    \angle A_0 A_{12} A_{13} = 89^\circ
\).
\endproblem
% $problem-source: старый конкурс Кенгуру
