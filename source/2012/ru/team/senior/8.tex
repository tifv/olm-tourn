\problem\problemscore{7}
Пусть $\gamma_1$ и $\gamma_2$~--- две окружности, касающиеся в точке $T$.
Через точку $T$ проведены прямые $a$ и $b$, которые пересекают окружность $\gamma_1$ вторично в точках $A$ и $B$
соответственно, а окружность $\gamma_2$~--- в точках $A_1$ и $B_1$
соответственно.
Пусть $X$~--- произвольная точка плоскости, не лежащая на данных прямых $a$, $b$ и окружностях $\gamma_1$, $\gamma_2$.
Окружности, описанные вокруг треугольников $ATX$ и $BTX$, пересекают окружность $\gamma_2$ в точках $A_2$ и $B_2$ соответственно.
Докажите, что прямые $TX$, $A_1B_2$ и $A_2B_1$ пересекаются в одной точке.
\solution
[{\begin{figure}
\centering
    \jeolmfigure[width=0.5\textwidth]{solution}
\caption{к задаче \ref{solution:2012/team/senior/8}}
\label{fig:solution:2012/team/senior/8}
\end{figure}}]%
\label{solution:2012/team/senior/8}%
Рис.~\ref{fig:solution:2012/team/senior/8}.
Пусть прямая $A_2 B_1$ вторично пересекает окружность $ATX$ в точке $Y$.
Тогда для направленных углов между прямыми имеем 
\begin{align*}
    \angle (AY, a)
={}&
    \angle (AY, AT)
=
    \angle (Y A_2, A_2 T)
=\\={}&
    \angle (B_1 A_2, A_2 T)
=
    \angle (B_1 A_1, A_1 T)
=
    \angle (B_1 A_1, a)
,\end{align*}
то есть прямые $AY$ и $A_1 B_1$ параллельны.
Но прямые $AB$ и $A_1 B_1$ параллельны из гомотетичности окружностей с центром
в точке $T$, так что $Y$ лежит на $AB$.
Аналогично, вторая точка $Z$ пересечения прямой $A_1 B_2$ с окружностью $BTX$
лежит на $AB$.
Тогда из параллельности $ABYZ \parallel A_1 B_1$ и вписанности четырехугольника
$A_2 B_1 A_1B_2$ получаем
\[
    \angle (ZY, ZB_2)
=
    \angle (B_1A_1,A_1B_2)
=
    \angle (B_1A_2,A_2B_2)
=
    \angle (A_2Y,A_2B_2)
,\]
то есть точки $A_2$, $B_2$, $Y$, $Z$ лежат на одной окружности.
Но тогда прямые $TX$, $A_1B_2$, $A_2B_1$ суть радикальные оси этой окружности и
окружностей $ATX$, $BTX$, откуда и следует и утверждение задачи.
\endproblem
