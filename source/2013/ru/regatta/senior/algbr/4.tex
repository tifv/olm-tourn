\problem
В основании прямой призмы лежит прямоугольный треугольник.
Все ребра этой призмы имеют натуральные длины, а площади некоторых двух её
граней равны 13 и 30.
Найдите стороны основания этой призмы.
\solution
\emph{Ответ:} $5$, $12$, $13$.
Докажем, что из двух указанных граней хотя бы одна является основанием.
Пусть это не так.
Если высота равна 1, а то у треугольника в основании стороны равны 13 и 30, что
не может быть у прямоугольного треугольника с натуральными сторонами.
Если же высота призмы больше единицы, то площади боковых граней делятся на эту
высоту и не могут быть взаимно просты.
Значит либо 13, либо 30~--- это площадь основания.
Если площадь основания 13, то произведение катетов равно 26, то есть это либо 1
и 26, либо 2 и 13, в обоих случая гипотенуза не получается равна натуральному
числу.
Значит площадь основания 30, и площадь одной из боковых граней~--- 13.
Это может быть только если высота равна 1, а сторона основания 13
(у прямоугольного треугольника с натуральными сторонами сторона не может быть
равна 1).
Произведение катетов равно 60, значит среди них нет стороны равной 13, то есть
13~--- это гипотенуза.
Тогда несложно убедиться, что катеты~--- это 12 и 5.
\endproblem
% $problem-source: KöMaL journal
