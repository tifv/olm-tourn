Решите в целых числах уравнение $x - y = x^2 + x y + y^2$.

\solution
\emph{Ответ:} $(0, 0)$, $(1, 0)$, $(0, -1)$, $(2, -1)$, $(1, -2)$ и $(2, -2)$.
Перепишем это как квадратное уравнение относительно $x$:
$x^2 + (y - 1) x + y^2 + y = 0$.
Дискриминант равен $1 - 3 y (y + 2)$.
Ясно, что при $y \geq 1$ и $y \leq -3$ это число отрицательно, то есть
решений нет.
При $y = 0$ получаем $x = 0$ или $x = 1$.
При $y = -1$ получаем $x = 0$ или $x = 2$.
Наконец, при $y = -2$ получаем $x = 1$ или $x = 2$.

