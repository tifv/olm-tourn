\problem
77 жителей острова лжецов и рыцарей стали в круг.
Всем известно, что их веса различны.
На вопрос <<У тебя есть сосед-лжец легче тебя?>> все ответили <<Да>>.
После перерыва они стали в круг в другом порядке.
Докажите, что на вопрос <<У тебя есть сосед-рыцарь легче тебя?>> как минимум
двое ответят <<Да>>.
\solution
Самый легкий житель в первом круге солгал, поэтому он лжец.
Но тогда он так же солжет, сказав <<Да>> и во второй раз.
Далее, в первом круге нет соседей-лжецов, иначе более тяжелый из них ответил бы
<<Нет>> на первый вопрос.
Значит, в первом круге одинокие лжецы чередуются с группами из подряд идущих
рыцарей.
Поэтому рыцарей не меньше, чем лжецов, а ввиду нечетности количества
жителей~--- больше.
Тогда во втором круге есть группа из не менее чем двух рыцарей.
Более тяжелый из них на второй вопрос ответит <<Да>>.
\endproblem
