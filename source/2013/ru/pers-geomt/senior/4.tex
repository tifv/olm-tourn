\problem
Пусть $H$, $I$ и $O$~--- соответственно точка пересечения высот, центры
вписанной и описанной окружностей остроугольного треугольника $ABC$.
Докажите, что $\angle AIH = 90^\circ$ тогда и только тогда, когда
$OI \parallel BC$.

\solution
[{\begin{figure}[b!]\centering
    \jeolmfigure[width=0.5\textwidth]{solution}
\caption{к задаче \ref{solution:2013/ru/pers-geomt/senior/4}.}
\label{fig:solution:2013/ru/pers-geomt/senior/4}
\end{figure}}]%
\label{solution:2013/ru/pers-geomt/senior/4}
Рис. \ref{fig:solution:2013/ru/pers-geomt/senior/4}.
Пусть $B_1$ и $C_1$~--- основания соответствующих высот треугольника $ABC$.
Центр окружности $\omega$, построенной на $AH$ как на диаметре и проходящей
через $B_1$ и $C_1$, обозначим через $F$.
Если $\angle AIH = 90^\circ$, то точка $I$ лежит на $\omega$ и является
серединой дуги $B_1 I C_1$.
Следовательно, $I$ лежит на линии центров окружности $\omega$ и окружности
девяти точек треугольника $ABC$.
Вписанная окружность треугольника $ABC$ и окружность девяти точек касаются в
точке Фейербаха, которая лежит на линии центров.
Откуда получаем, что точка Фейербаха совпадает с $F$.
В частности, $I F = r$.
То есть радиус окружности $\omega$ равен $r$.
Поскольку $AH$ равно удвоенному расстоянию от $O$ до $BC$, получаем, что точки
$O$ и $I$ равноудалены от $BC$, что и требовалось.
\\
\emph{А как же доказать в обратную сторону???}
\endproblem
% $problem-source: Ф.\,Ивлев
