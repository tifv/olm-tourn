\problem
В~компании из~$n$ человек нет тех, которые были~бы не~знакомы ни~с~кем, и~тех,
которые были~бы знакомы со~всеми остальными.
Докажите, что каких-то четверых из~них можно посадить за~круглый стол так,
чтобы каждый был знаком ровно с~одним своим соседом.
\solution
Пусть человек $A$ имеет ровно одного знакомого~--- человека $B$.
$B$ не~знаком хотя~бы с~кем-то, пусть с~$C$.
$C$ знаком хотя~бы с~кем-то, пусть с~$D$.
Заметим, что $D$ и~$A$ не~знакомы, так как $A$ знаком только с~$B$.
Тогда, посадив подряд $A$, $B$, $C$, $D$, получим требуемое.
\par
Заменив в~рассуждении выше знакомство на~незнакомство и~наоборот, показываем,
что достаточно иметь человека с~ровно одним незнакомым.
\par
Пусть нет ни~такого, ни~такого.
Удалив теперь любого человека, получим компанию из~$n - 1$ человека,
удовлетворяющую условию задачи.
Ни~одна компания из~трех человек условию задачи не~удовлетворяет, поэтому этот
процесс рано или поздно приведет к~желаемой четверке.
\endproblem
% $problem-source: Kürschák 2014
