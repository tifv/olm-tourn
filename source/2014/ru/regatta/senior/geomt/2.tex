\ifsolution
\begin{figure}\centering
    \jeolmfigure[width=0.5\textwidth]{solution}
    \caption{к задаче \ref{2014/ru/regatta/senior/geomt/2:solution}}
    \label{2014/ru/regatta/senior/geomt/2:solution:fig}
\end{figure}%
\fi % \ifsolution

\problem
Вписанная окружность треугольника $ABC$ касается стороны $AC$ в~точке~$D$.
Оказалось, что $\angle{BDC}$ равен $60^{\circ}$.
Докажите, что вписанные окружности треугольников 
$ABD$ и~$CBD$ касаются стороны~$BD$ в~одной точке и~найдите отношение радиусов
этих окружностей.
\solution
\label{2014/ru/regatta/senior/geomt/2:solution}%
\emph{Ответ:} 3.
\par
Пусть вписанная окружность треугольника $BCD$ касается стороны $BD$
в~точке~$M$, а~вписанная окружность треугольника $BAD$ касается стороны~$BD$
в~точке~$N$
(рис.~\ref{2014/ru/regatta/senior/geomt/2:solution:fig}).
Тогда $DM = (DB + DC - BC) / 2$ и~$DN = (DA + DB - AB) / 2$.
Так как $DC = (AC + BC - AB) / 2$ и~$DA = (AC + AB - BC) / 2$,
то~$DC - BC = DA - AB$ и~$DM = DN$, что означает, что наши окружности касаются.
\par
Обозначим центр вписанной окружности треугольника $ABD$ через~$O_1$, а~центр
вписанной окружности $BCD$~--- через $O_2$.
Тогда $\angle O_1 D O_2 = 90^\circ$ и~$DM$~--- высота треугольника $O_1 D O_2$.
Заметив, что $\angle M D O_2 = \angle M O_1 D = 30^\circ$
и~$\angle M D O_1 = \angle M O_2 D = 60^\circ$, находим, что отношение радиусов
равно $M O_1 / M O_2 = 3$.
\endproblem
% $problem-source: М.\,Волчкевич
