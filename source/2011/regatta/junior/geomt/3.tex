\problem
В равнобедренном треугольнике $ABC$ с основанием $BC$ проведена биссектриса
$BD$.
Оказалось, что $BD + DA = BC$.
Найдите углы треугольника $ABC$.

\solution
[{\begin{figure}
\centering
    \jeolmfigure[width=0.5\textwidth]{3-solution}
\caption{к задаче \ref{solution:2011/regatta/junior/geomt/3}.}
\label{fig:solution:2011/regatta/junior/geomt/3}
\end{figure}}]%
\label{solution:2011/regatta/junior/geomt/3}%
Пусть $\angle ABC = \angle ACB = 2 \alpha$.
Отметим на стороне $BC$ точку $X$~--- пересечение описанной окружности
треугольника $ABD$ с $BC$
(см.\,рис.\,\ref{fig:solution:2011/regatta/junior/geomt/3}).
Тогда
$\angle DXC = \angle BAC = 180^\circ - 4 \alpha$,
откуда $\angle XDC = 2 \alpha = \angle DCX$.
Значит $XC = XD$.
Кроме того, $XD = AD$ так как на эти хорды опираются равные углы
(так как $BD$~--- биссектриса).
Значит, из условия мы получаем, что $BD = BX$, и, значит в треугольнике $BDX$
угол при вершине равен $\alpha$, а углы про основании~--- $4 \alpha$.
Отсюда $\alpha = 20^\circ$.
\emph{Ответ:}
$40^\circ$, $40^\circ$ и $100^\circ$.

\endproblem
