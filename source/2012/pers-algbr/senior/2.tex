Натуральное число $k > 2$ и вещественные числа $a$, $b$ таковы, что
многочлен $x^k + a x + 1$ делится на многочлен $x^2 + b x + 1$.
Докажите, что $a (a - b) = 0$.

\solution
Обозначим $P(x) = x^2 + b x + 1$.
Пусть $a \neq 0$, иначе очевидно, что $a (a - b) = 0$.
По теореме Виета корни (возможно, комплексные) многочлена $P(x)$ в произведении
дают единицу.
Обозначим их через $u$ и $1 / u$.
Тогда из делимости первого многочлена на второй следует, что
$u^k + a u + 1 = 1 / u^k + a / u + 1 = 0$.
Отсюда
$-a = (u^k + 1) / u = (u^k + 1) / u^{k - 1}$,
то есть $u^{k - 2} = 1$,
$u^2 + b u + 1 = 0 = u^k + a u + 1 = u^2 + a u + 1$, $a = b$.
\par
\emph{Другое решение.}
Обозначим $P(x) = x^2 + b x + 1$.
Пусть $a \neq 0$, иначе очевидно, что $a (a - b) = 0$.
Если $x^k + a x + 1$ делится на $P(x)$, то $x^k + a x^{k - 1} + 1$ тоже делится
на $P(x)$.
Тогда их разность $a x^{k - 1} - a x$ делится на $P(x)$, а значит и
$x^{k - 2} - 1$ делится на $P(x)$.
Остаток при делении $x^k + a x^{k - 1} + 1$ на $x^{k - 2} - 1$ равен
$x^2 + a x + 1$, и он тоже будет делится на $P(x)$.
Из $x^2 + b x + 1 \mid x^2 + a x + 1$ следует $a = b$.

