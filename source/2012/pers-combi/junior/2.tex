На доске написаны числа 1, 2, 3, \ldots, 2012.
Петя стирает их по одному.
Докажите, что он может делать это в таком порядке, чтобы сумма нестертых чисел
всегда была составным числом.

\solution
Заметим, что сумма $1 + 2 + \ldots + n = n (n + 1) / 2$~--- составное число при
$n > 2$.
Поэтому будем вычеркивать каждый раз самое большое число, пока не останутся
1, 2, 3 и 4.
Далее вычеркиваем по порядку 1, 3, 2, оставляя соответственно суммы 9, 6, 4.

