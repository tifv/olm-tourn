В прямоугольнике $ABCD$ точка $P$~--- середина стороны $AB$, а точка
$Q$~--- основание перпендикуляра, опущенного из точки $C$ на $PD$.
Докажите, что $BQ = BC$.

\solution
[{\begin{figure}
\centering
    \jeolmfigure[width=0.5\textwidth]{1-solution}
\caption{к задаче \ref{solution:2011/pers-geomt/senior/1}.}
\label{fig:solution:2011/pers-geomt/senior/1}
\end{figure}}]%
\label{solution:2011/pers-geomt/senior/1}%
См.\,рис.\,\ref{fig:solution:2011/pers-geomt/senior/1}.
Заметим, что четырехугольник $PBCQ$ вписанный
(так у него противоположные углы прямые),
а значит $\angle BCQ = \angle DPA$.
С другой стороны $\angle DPA = \angle CPB$ (из симметрии) и
$\angle CPB = \angle BQC$ (снова из вписанности четырехугольника $BCQP$).
Тем самым мы получили, что $\angle BCQ = \angle BQC$, что и означает, что
$BQ = BC$.

