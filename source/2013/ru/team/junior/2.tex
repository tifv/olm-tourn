\problem\problemscore{4}
Перед началом шахматного турнира участники были занумерованы.
Каждый с каждым сыграл по разу.
Во всех партиях, которые не закончились вничью, у победителя номер был меньше,
чем у побеждённого.
Петя победил Васю, но набрал по результатам турнира меньше очков, чем Вася.
Каково наименьшее возможное число участников турнира?
(В турнире давали $1$ очко за~победу, $0{,}5$ за~ничью, $0$ за~поражение.)
\solution
\emph{Ответ:} 5 участников.
\par
\emph{Оценка.}
Пусть $x$ человек победили Петю, Вася победил $y$ человек.
У этих $x$ номера меньше, чем у Пети, и тем более, меньше, чем у Васи.
А у этих $y$ человек номера больше, чем у Васи.
Значит, все эти люди различны.
Число очков $y$ игрока тем больше, чем больше у него разность между числом
побед и числом поражений.
У Пети разность $a \geq 1 - x$, у Васи разность $b \leq y - 1$, и по условию
$a < b$.
Значит, $1 - x < y - 1$, откуда $x + y > 2$.
А всего у нас не меньше $x + y + 2 > 4$ игроков, то есть, не менее 5.
\par
\emph{Пример:} (на пять игроков)
Петя победил Васю, Вася победил троих других, а остальные партии закончились
вничью.
Тогда Петя набрал $2{,}5$ очка, Вася набрал 3 очка~--- больше, чем у Пети.
\endproblem
% $problem-source: А.\,Шаповалов
