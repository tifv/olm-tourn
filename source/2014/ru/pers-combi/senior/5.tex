\problem
На~клетках шахматной доски лежит по~алмазу так, что в~каждой паре соседних
(по~стороне) клеток веса алмазов отличаются меньше, чем на~1 карат.
Докажите, что эти алмазы можно разложить по~одному в~клетки прямоугольника
$2 \times 32$ так, чтобы по-прежнему в~каждой паре соседних клеток веса алмазов
отличались меньше, чем на~1~карат.
\solution
Упорядочим массы алмазов $m_1 \geqslant m_2 \geqslant \ldots \geqslant m_{64}$.
\par
\emph{Лемма.}
Для всякого $i < 63$ выполнены неравенства $m_i - m_{i+2} < 1$, а~значит
и~$m_{i} - m_{i+1} < 1$.
\\\emph{Доказательство леммы.}
В~доске можно провести два непересекающихся клетчатых пути от~алмаза
с~массой~$m_i$ до~алмазов с~массой $m_{i+1}$ и~$m_{i+2}$.
Пусть первый алмаз от~$i$-го на~первом из~путей с~номером больше~$i$, имеет
номер~$k$.
На~втором пути аналогичный алмаз имеет номер~$l$.
Легко видеть, что $0 \leq m_i - m_l < 1$ и~$0 \leq m_i - m_k < 1$, так как
предыдущий алмаз имеет массу не~меньше $m_i$ и~за~один шаг по~пути она меняется
меньше, чем на~единицу.
Тогда эти неравенства тем более верны для $i + 1$ и~$i + 2$.
Лемма доказана.
\par
Будем считать, что в~прямоугольнике $2 \times 32$ столбцы имеют высоту в~две
клетки.
Тогда в~$j$-ом столбце расположим сверху алмаз массой $m_{2j-1}$, а~снизу~---
алмаз массой $m_{2j}$.
Номера любых соседних камней отличаются не~более чем на~$2$, а~по~лемме
их~массы отличаются менее чем на~один.
Значит, получена требуемая расстановка.
\endproblem
% $problem-source: А.\,Шаповалов
