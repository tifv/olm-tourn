\problem
Все вершины правильного многоугольника лежат на~поверхности куба, но~его
плоскость не~совпадает ни~с~одной из~плоскостей граней.
Какое наибольшее количество вершин может быть у~этого многоугольника?
\solution
\emph{Ответ:} $12$.
\par
\emph{Оценка.}
Ясно, что на~каждой грани лежит не~более $2$ вершин многоугольника, поэтому
вершин не~более $12$.
\par
\emph{Пример.}
Проведем через середины $6$~ребер сечение в~форме правильного шестиугольника.
Впишем в~этот шестиугольник правильный $12$-угольник так, чтобы стороны
$12$-угольника через одну лежали на~сторонах шестиугольника.
\endproblem
% $problem-source: А.\,Шаповалов% задача старого тургора
