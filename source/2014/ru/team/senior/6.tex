\ifsolution
\begin{figure}\centering
    \jeolmfigure[height=0.5\textwidth]{solution}
    \caption{к задаче \ref{2014/ru/team/senior/6:solution}}
    \label{2014/ru/team/senior/6:solution:fig}
\end{figure}%
\fi % \ifsolution

\problem\problemscore{7}
Через центр правильного треугольника $ABC$ провели произвольную прямую~$l$,
пересекающую стороны $AB$ и~$BC$ в~точках $D$ и~$E$.
Построили точку~$F$ такую, что $AE = FE$ и~$CD = FD$.
Докажите, что расстояние от~точки~$F$ до~прямой~$l$ не~зависит от~выбора этой
прямой.
\solution
\label{2014/ru/team/senior/6:solution}%
Рис.~\ref{2014/ru/team/senior/6:solution:fig}.
Рассмотрим правильный тетраэдр $ABCS$ с~основанием $ABC$.
Заметим, что $SE = AE = FE$ и~$SD = CD = FD$.
Поэтому треугольники $FDE$ и~$SDE$ равны.
Обозначим через $O$ центр правильного треугольника $ABC$.
Поскольку прямая~$SO$ перпендикулярна плоскости $ABC$, то~отрезок~$SO$ является
высотой в~треугольнике $SDE$.
Значит, расстояние от~точки~$F$ до~прямой~$l$ равно $SO$ и~не~зависит
от~прямой~$l$.
\endproblem
% $problem-source: М.\,Волчкевич
