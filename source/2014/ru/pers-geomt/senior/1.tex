\problem
Замкнутая шестизвенная ломаная в~пространстве такова, что каждое её~ребро
параллельно одной из~координатных осей прямоугольной системы координат.
Докажите, что её~вершины лежат на~одной сфере или в~одной плоскости.
\solution
Предположим, что среди направлений ребер $A_1 A_2, A_2 A_3, \ldots, A_5 A_6$
нашей ломаной $A_1 \ldots A_6$ не~встречается направление оси $Oz$.
Тогда все эти вершины лежат в~плоскости, параллельной $Oxy$ и~проходящей через
$A_1$.
Аналогично разбираются случаи, когда одно из~направлений координатных осей
встречается среди направлений ребер не~более одного раза.
Осталось разобрать случай, когда каждое встречается ровно два раза.
Рассматривая абсциссы вершин нашей ломаной мы~видим, что при изменении
$i = 1, 2, 3, 4, 5, 6, 1$ абсциссы $A_i$ меняются дважды, то~есть 
принимают всего два значения $a$ и~$b$.
Рассмотрим точку, абсцисса которой равна $(a + b) / 2$, ордината и~аппликата
определяются аналогично.
Легко видеть, что она равноудалена от~концов любого ребра ломаной, и, стало
быть, от~всех её~вершин.
\endproblem
% $problem-source: Iran 2014
