\problem
Для скольких натуральных $N$ от $1$ до $2013$ уравнение $x^{[x]} = N$ имеет
решение в положительных вещественных $x$?
($[x]$~--- это наибольшее целое число, не превосходящее $x$.)
\solution
\emph{Ответ:} $412$ чисел.
\par
Заметим, что подходящие числа $N$ для $x$ таких, что $[x] = n$~--- это числа от
$n^n$ до $(n+1)^n - 1$, то есть в точности такие числа, что
$[\sqrt[n]{N}] = n$.
Такие числа (среди чисел от $1$ до $2013$)~--- это
число $1$,
числа от $2^2$ до $3^2 - 1$ (их ровно $5$),
числа от $3^3$ до $4^3 - 1$ (их ровно $37$),
числа от $4^4$ до $5^4 - 1$ (их ровно $369$).
Итого мы получаем всего $1 + 5 + 37 + 369 = 412$ чисел.
\endproblem
% $problem-source: фольклор
