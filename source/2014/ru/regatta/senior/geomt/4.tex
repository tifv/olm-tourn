\ifsolution
\begin{figure}\centering
    \jeolmfigure[width=0.5\textwidth]{solution}
    \caption{к задаче \ref{2014/ru/regatta/senior/geomt/4:solution}}
    \label{2014/ru/regatta/senior/geomt/4:solution:fig}
\end{figure}%
\fi % \ifsolution

\problem
На~сторонах $BC$, $CA$ и~$AB$ остроугольного треугольника $ABC$ выбраны
точки $A_1$, $B_1$ и~$C_1$ соответственно.
Описанные окружности треугольников  $A B_1 C_1$, $B C_1 A_1$ и~$C A_1 B_1$
пересекаются в~точке~$P$ внутри треугольника $ABC$.
Точки $O_1$, $O_2$ и~$O_3$~--- центры этих окружностей.
Докажите, что $4 S(O_1 O_2 O_3) \geq S(ABC)$.
\solution\label{2014/ru/regatta/senior/geomt/4:solution}
Так~как $O_1 O_2 \perp P C_1$ и~$O_1 O_3 \perp P B_1$,
то~$\angle O_2 O_1 O_3 = 180^\circ - \angle B_1 P C_1 = \angle BAC$. 
Аналогично,
$\angle O_1 O_3 O_2 = \angle ACB$ и~$\angle O_3 O_2 O_1 = \angle ABC$. 
Тогда треугольники $ABC$ и~$O_1 O_2 O_3$ подобны.
Рис.~\ref{2014/ru/regatta/senior/geomt/4:solution:fig}.
Пусть $k = AB / O_1 O_2$. 
Пусть $M$ и~$N$~--- проекции точек $O_1$ и~$O_2$ на~прямую~$AB$.
Заметим, что $M$~--- середина~$A C_1$, а~$N$~--- середина $C_1B$. 
Тогда $AB = 2 MN \leq 2 O_1 O_2$ и~$k \leq 2$. 
С~другой стороны, $S(ABC) = k^2 S(O_1 O_2 O_3) \leq 4 S(O_1 O_2 O_3)$. 
\endproblem
% $problem-source: А.\,Смирнов по мотивам Macedonia 2014
