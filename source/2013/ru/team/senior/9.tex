\ifsolution
\begin{figure}[bt]\centering
    \jeolmfigure[width=0.5\textwidth]{solution}
    \caption{к задаче \ref{solution:2013/ru/team/senior/9}.}
    \label{fig:solution:2013/ru/team/senior/9}
\end{figure}
\fi % \ifsolution

\problem\problemscore{9}
Докажите, что в остроугольном треугольнике сумма длин медиан не превосходит
суммы радиусов вневписанных окружностей.
\solution
\label{solution:2013/ru/team/senior/9}
Рис.~\ref{fig:solution:2013/ru/team/senior/9}.
Обозначим вершины исходного треугольника через $A$, $B$, $C$.
Пусть $M$~--- середина стороны $AC$, а $I_A$ и $I_C$~--- соответственно центры
вневписанных в углы $A$ и $C$ окружностей.
Обозначим также через $A'$ и $C'$ соответственно точки касания вневписанных в
углы $A$ и $C$ окружностей с продолжением стороны $AC$, а через $L$~---
середину большей дуги $AC$ описанной окружности треугольника $ABC$.
Известно, что $AA' = CC' = p$, где $p$~--- полупериметр треугольника $ABC$.
Значит, $MA' = AA' - MA = CC' - MC = MC'$.
Так как $L$~--- середина дуги $AC$, то $LA = LC$, а следовательно, $LM$~---
медиана и высота, т.\,е. $LM \perp AC$.
Так как $A'$ и $C'$~--- точки касания, то и $I_A A' \perp AC \perp I_C C'$.
Следовательно, $A' I_A I_C C'$~--- трапеция.
Вспоминая, что $I_A I_C$~--- внешняя биссектриса треугольника $ABC$, замечаем,
что $I_A I_C$ проходит через $L$, а следовательно, $ML$~--- средняя линия
найденной трапеции.
Ввиду того, что треугольник $ABC$ остроугольный, его центр описанной окружности
$O$ будет лежать на отрезке $ML$.
Тогда $\frac{1}{2} (I_A A' + I_C C') = ML = MO + OL = MO + OA \geq MA$.
Складывая три таких неравенства, получаем требуемое.
\endproblem
% $problem-source: Ф.\,Ивлев
