\problem\problemscore{4}
% "\text{\,м}" -> "\text{метр}"
Есть куб со стороной $1 \text{\,м}$ и набор из пяти красок.
Сначала Петя режет куб на равные меньшие кубики размера не более
% "\text{\,см}" -> "\text{сантиметр}"
$1 \text{\,см}$ (размер он выбирает сам).
Затем Вася красит кубики как хочет (не обязательно все одинаково), но так,
чтобы каждая грань была одноцветна.
Наконец, Петя складывает куб, используя все кубики.
Докажите, что независимо от действий Васи Петя может добиться того, чтобы
каждая грань большого куба была одноцветной.

\solution Решение \texttt{team/senior/3.tex}.
\endproblem
% $problem-source: А.\,Шаповалов
