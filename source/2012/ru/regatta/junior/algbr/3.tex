\problem
Вася расставил по кругу числа от 1 до 10 в произвольном порядке и посчитал все
суммы трех подряд стоящих чисел.
Какое наибольшее значение может принимать наименьшая из этих сумм?
\solution
\emph{Ответ:} 15.
\par
Заметим, что сумма всех 10 троек подряд стоящих чисел~--- это утроенная сумма
всех чисел, то есть 165.
Значит средняя сумма в каждой тройке~--- это $165 / 10 = 16{,}5$.
Покажем, что сделать, чтобы минимальная сумма была 16 тоже невозможно.
Так как среднее значение суммы $16{,}5$~--- это означает, что тройки суммарно
могут превысить сумму 16 всего на 5.
Но заметим, что соседние тройки (которые имеют два общих числа) не могут иметь
равные суммы (иначе бы первое число первой тройки равнялось бы последнему
числу соседней тройки).
Но тогда единственным вариантом было бы чередование троек с суммой 16 и 17,
что привело бы к тому, что числа через два отличаются на один, но тогда числа
через 5 получаются равными.
Для суммы пятнадцать имеется пример: 1, 8, 7, 5, 3, 10, 2, 4, 9, 6 по кругу.
\endproblem
