\ifsolution
\begin{figure}\centering
    \jeolmfigure[height=0.5\textwidth]{solution}
    \caption{к решению задачи \ref{2013/ru/pers-geomt/junior/3:solution}.}
    \label{2013/ru/pers-geomt/junior/3:solution:fig}
\end{figure}%
\fi % \ifsolution

\problem
Во вписанном четырёхугольнике $ABCD$ выполнено $AB > CD$ и $BC > AD$.
На лучах $AB$ и $CD$ выбраны точки $K$ и $M$ соответственно так, что
$AK = CM = \frac{1}{2} (AB + CD)$,
а на лучах $BC$ и $DA$~--- точки $L$ и $N$ соответственно так, что
$BL = DN = \frac{1}{2} (BC + AD)$.
Докажите, что $KLMN$ --- прямоугольник и 
площадь его равна площади $ABCD$.
\solution 
\label{2013/ru/pers-geomt/junior/3:solution}%
Рис.~\ref{2013/ru/pers-geomt/junior/3:solution:fig}.
Во-первых, заметим, что в силу указанных соотношений $KB = MD$, причем точка
$K$ лежит на отрезке $AB$, а точка $D$ на отрезке $MC$.
Во-вторых, $NA = LC$ и точка $L$ лежит на отрезке $BC$, а точка $A$ лежит на
отрезке $ND$.
В треугольниках $MCL$ и $KAN$ верны соотношения $NA = CL$, $KA = MC$ и
$\angle NAK = \angle LCM$ (первый~--- внешний в четырехугольнике $ABCD$, а
второй~--- противоположный внутренний).
Следовательно, треугольники $MCL$ и $KAN$ равны и, в частности, равны $NK$ и
$ML$.
Аналогичными рассуждениями приходим к тому, что равны треугольники $KBL$ и
$MDN$ и отрезки $KL = MN$.
Из полученных равенств отрезков заключаем, что $KLMN$~--- параллелограмм.
Заметим, что
\(
    \angle KLM
=
    180^\circ - \angle KLB - \angle MLC
=
    180^\circ - \angle MAD - \angle KNA
=
    180^\circ - \angle KNM
\).
Таким образом, противоположные углы параллелограмма дополняют друг друга до
$180^\circ$, что возможно, только если он прямоугольник.
Осталось показать, что четырехугольник $ABCD$ и прямоугольник $KLMN$
равновелики.
Для этого отметим $F$~--- точку пересечения отрезков $ML$ и $AD$.
Верна следующая цепочка равенств
\begin{align*}
    S_{ABCD}
&{}=
    S_{AKLF} + S_{KBL} + S_{LCM} - S_{MFD}
=\\&{}=
    S_{AKLF} + S_{MDN} + S_{MAK} - S_{MFD}
=
    S_{KLMN}.
\end{align*}
\endproblem
% $problem-source: Eisso\,J.\,Atzema, предложил В.\,Дубровский
