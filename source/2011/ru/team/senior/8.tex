\problem\problemscore{9}
Дан треугольник $ABC$ и концентрические окружности $\omega_b$, $\omega_c$ с
центром в $A$.
Произвольный луч, выходящий из $A$, пересекает эти окружности в точках $B'$ и
$C'$ соответственно.
Серединные перпендикуляры к отрезкам $BB'$ и $CC'$ пересекаются в точке $X$.
Докажите, что точки $X$, построенные таким образом для всех лучей, выходящих из
$A$, лежат на одной прямой.
\solution
Обозначим радиусы $\omega_b$, $\omega_c$ через $u$ и $v$ соответственно.
Введем временно на оси $AB'C'$ координату так, что $A = 0$, $B' = u$, $C' = v$.
Пусть проекция точки $X$ на эту ось имеет координату $x$.
Имеем
$XB'^2 - XA^2 = (x - u)^2 - x^2$,
$XC'^2 - XA^2 = (x - v)^2 - x^2$,
так что 
\begin{align*}
    v (XB^2 - XA^2) - u (XC^2 - XA^2)
={}&
    v (XB'^2 - XA^2) - u (XC'^2 - XA^2)
=\\={}&
    u v (u - v)
.\end{align*}
Но левая часть есть линейная функция от декартовых координат точки $X$, причем
непостоянная (уменьшаемое не меняется, когда $X$ движется перпендикулярно $AB$,
а вычитаемое~--- когда перпендикулярно $AC$);
а правая часть~--- константа ($u$ и $v$ равны радиусам окружностей $\omega_b$ и
$\omega_c$ соответственно).
Таким образом, все точки $X$ удовлетворяют одному и тому же линейному
уравнению, и поэтому лежат на одной прямой.
\endproblem
