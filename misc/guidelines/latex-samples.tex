% Части условия выделены подпунктами
Условие\ldots
\\\emph{(1)}
\\\emph{(2)}
\\\emph{(3)}
\\
Продолжение условия\ldots

% Решение задачи без ответа (на доказательство).
\solution
<Решение.>
%Времена!

% Решение задачи с ответом
\solution
\emph{Ответ:} <ответ>.
<Решение.>

% Решение задачи с оценкой и примером.
\solution
\emph{Ответ:} <ответ>.
\emph{Оценка.}
<Оценка.>
\emph{Пример:} <пример>.

% Альтернативные решения
\solution
<Первое решение.>
\par
\emph{Другое решение.}
<Второе решение.>

% Решение из нескольких больших частей
\solution
<Лемма.>
\\\emph{Доказательство леммы.}
<Доказательство.>
\par
<Продолжение решения.>

% Картинка к решению
\ifsolution
\begin{figure}\centering
    \jeolmfigure[width=0.5\textwidth]{solution}
    \caption{к задаче \ref{solution:<problem metapath>}}
    \label{fig:solution:<problem metapath>}
\end{figure}
\fi % \ifsolution

% Ссылка на картинку из решения
\solution
\label{solution:<problem metapath>}
Рис.~\ref{fig:solution:<problem metapath>}.

% Дублирование задач: префикс задачи
% nospell begin
\def\ifsolutiondefined{%
    \csname ifsolution:<original metapath>\endcsname}
\expandafter\providecommand
    \csname ifsolution:<original metapath>\endcsname
    {\iffalse}
\def\definesolution{%
    \expandafter\gdef
    \csname ifsolution:<original metapath>\endcsname
    {\iftrue}}
% nospell end

% Дублирование задач: маркер оригинала
% duplicated in [<duplicate metapath>]

% Дублирование задач: маркер копии
% duplicate of [<original metapath>]

% Дублирование задач: решение
\label{solution:<problem metapath}%
\emph{Ответ:} <ответ>.
\ifsolutiondefined
См.~решение задачи \ref{solution:<other problem metapath>}.
\else
\definesolution
<Решение.>
\fi % \ifsolutiondefined
