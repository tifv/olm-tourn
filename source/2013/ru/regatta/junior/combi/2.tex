\problem
Фабрика выпускает наборы из $n > 2$ слоников различной величины.
По стандарту разница масс соседних слоников внутри каждого набора должна быть
одной и той же.
Контролер проверяет наборы по одному с помощью чашечных весов без гирь.
При каком наименьшем $n$ это возможно?

\solution
\emph{Ответ:} при $n = 5$.
Занумеруем слоников по возрастанию величины, тогда достаточно убедиться, что
$C_1 + C_4 = C_2 + C_3$, $C_1 + C_5 = C_2 + C_4$ и $C_2 + C_5 = C_3 + C_4$.
Эти равенства равносильны соответственно $C_4 - C_3 = C_2 - C_1$,
$C_2 - C_1 = C_5 - C_4$ и $C_5 - C_4 = C_3 - C_2$, то есть все соседние
разности равны.
При $n = 4$ проверить невозможно, потому что наборы весов $10, 11, 12, 13$ и
$10, 11, 13, 14$ дадут при любых взвешиваниях одинаковый результат.
\endproblem
% $criteria:
%   'Верный ответ и обоснование для 5~--- 2 балла.
%   Фразу <<никак нельзя сравнить разность между средними слониками с другими
%   разностями>> не считать обоснованием невозможности для $n = 4$
%   (можно спросить: а почему из набора равенств нельзя это алгебраически
%   вывести?)'
% $problem-source: А.\,Шаповалов
