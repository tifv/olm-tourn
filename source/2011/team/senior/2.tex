\textsf{(4)}
Клетки квадрата $50 \times 50$ покрашены в 50 цветов, причем клеток каждого
цвета ровно 50.
Докажите, что найдется линия (строка или столбец), содержащая клетки не менее
чем 8 разных цветов.

\solution
\emph{Решение.}
Докажем более общее утверждение: если $N = n^2 + 1$ и клетки квадрата
$N \times N$ покрашены в $N$ цветов, причем клеток всех цветов поровну, то
найдется ряд (строка или столбец) в котором встречаются клетки хотя бы $n + 1$
цвета.
Предположим противное, пусть в каждом ряду встречаются клетки не более, чем $n$
цветов.
Рассмотрим $k$-ый цвет.
Пусть клетки $k$-го цвета встречаются в $c_k$ столбцах и $r_k$ строках.
Тогда все клетки этого цвета могут встречаться только на пересечениях этих
строки и столбцов, и, значит, $N \leq r_k c_k \leq \frac{1}{4}(r_k + c_k)^2$,
откуда $(r_k + c_k)^2 \geq 4 N$, то есть $r_k + c_k \geq \sqrt{4 n^2 + 4}$, что
в свою очередь, означает, что $r_k + c_k \geq 2 n + 1$.
Суммируя, получаем, что  $\sum_{k = 1}^N (r_k + c_k) \geq N(2 n + 1)$.
\\
С другой стороны пусть в каждом ряду встречается не больше $n$ цветов.
Всего рядов $2 N$, и, значит, количество пар (цвет, ряд) таких, что цвет
встречается в данном ряду не больше, чем $2 N n$.
Однако эта сумма в точности равна оцененной выше, что и является противоречием.

