\problem
Найдите все такие натуральные $a$ и $b$, что $\sqrt{a} - \sqrt{b}$ является
корнем уравнения $x^2 + a x - b = 0$.

\solution
\emph{Ответ:} $a = 2$, $b = 1$.
Подставим в наше уравнение выражение $\sqrt{a} - \sqrt{b}$.
После несложных алгебраических преобразований мы получим равенство: 
\[
    \sqrt{b}
=
    \frac{\sqrt{a}(\sqrt{a} + 1)}{\sqrt{a} + 2}
.\]
Теперь возведем это равенство в квадрат и получится выражение
$(4 b - 2 a) \sqrt{a} = c$, где $c$~--- целое число.
Отсюда следует, что либо $a = 2 b$, либо $\sqrt{a}$~--- рациональное число.
Первый случай мы разберем позже, а из второго наблюдения немедленно следует,
что $\sqrt{a}$~--- целое.
В этом случае $\sqrt{b}$~--- также рациональное, а значит~--- целое.
Тогда, обозначая через $a_1$ значение $\sqrt{a}$ (целое число), получаем, что
$a_1 (a_1 + 1)$ должно делится на $a_1 + 2$.
Но это невозможно, так как $a_1 + 1$ взаимно просто с $a_1 + 2$, а $a_1$ просто
меньше.
Таким образом, обязательно $a = 2 b$.
Подставляя это в исходное уравнение и решая его относительно $\sqrt{b}$, мы
находим, что $\sqrt{b} = 1$.
То есть получаем, что $b = 1$ и $a = 2$.
Этот набор подходит.
\endproblem
% $problem-source: Д.\,Максимов
