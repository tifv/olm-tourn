Точки $A$, $B$, $C$, $D$ лежат на окружности в указанном порядке, причем $AB$ и
$CD$ непараллельны.
Длина дуги $AB$, содержащей точки $C$ и $D$, в два раза больше длины дуги $CD$,
не содержащей точек $A$ и $B$.
Точка $E$ задается условиями $AC = AE$ и $BD = BE$.
Оказалось, что перпендикуляр из точки $E$ на прямую $AB$ проходит через
середину дуги $CD$, не содержащей точек $A$ и $B$.
Найдите $\angle ACB$.

%Czech-Polish-Slovak Match, 2011

%Points $A$, $B$, $C$, $D$ lie on a circle (in that order) where $AB$ and $CD$
%are not parallel.
%The length of arc $AB$ (which contains the points $D$ and $C$)
%is twice the length of arc $CD$ (which does not contain the points $A$ and
%$B$).
%Let $E$ be a point where $AC = AE$ and $BD = BE$.
%Prove that if the perpendicular line from point $E$ to the line $AB$ passes
%through the center of the arc $CD$
%(which does not contain the points $A$ and $B$), then
%$\angle ACB = 108^\circ$.

\solution
[{\begin{figure}
\centering
    \jeolmfigure[width=0.5\textwidth]{3-solution}
\caption{к задаче \ref{solution:2011/pers-geomt/senior/3}.}
\label{fig:solution:2011/pers-geomt/senior/3}
\end{figure}}]%
\label{solution:2011/pers-geomt/senior/3}%
См. рис. \ref{fig:solution:2011/pers-geomt/senior/3}.
Как известно, для точек $X$, $Y$ на плоскости любая прямая, перпендикулярная
$XY$, задается условием
$\{Z \colon ZX^2 - ZY^2 = \const\}$.
Пусть $P$~--- середина $CD$.
Тогда условие $EP \perp AB$ влечет равенство
$P A^2 - P B^2 = E A^2 - E B^2 = A C^2 - B D^2$.
Значит, $A P^2 - A C^2 = B P^2 - B D^2 = B' P^2 - B' C^2$, где точка $B'$
симметрична $B$ относительно серединного перпендикуляра к $CD$
(в силу условия о непараллельности $B' \neq A$).
Отсюда получаем, что прямые $AB'$ и $PC$ перпендикулярны.
Для направленных углов между прямыми имеем 
\begin{gather*}
    \pi / 2
=
    \angle (AB', PC)
=
    \angle (AB', B'C) + \angle (B'C, CP)
=\\=
    \angle (AB, BC) + \angle (DP, DB)
=
    \angle (AB, BC) + \angle (AP, AB)
=
    \angle (AP, BC)
.\end{gather*}
Таким образом, прямые $AP$ и $BC$ перпендикулярны.
Это значит, что разность угловых мер дуг $AB$ и $PC$
(не содержащих точек $P$ и $A$ соответственно) равна $\pi$.
Если обозначит дугу $PC$ за $x$, то получаем, что дуга $AB$
(не содержащая $P$) равна $2 \pi - 4 x$, откуда $5 x = \pi$, $x = \pi / 5$,
дуга $AB$ равна $6 \pi / 5$, угол $\angle ACB$ равен $3 \pi / 5$.
\emph{Ответ:} $3 \pi / 5$.

