\problem
Вася выписал на доску числа от $1$ до $20132013$.
На сколько больше он написал единиц, чем троек?
\solution
\emph{Ответ:} $10041004$.
Заметим, что среди чисел от $1$ до $9999999$ будет написано поровну всех
ненулевых цифр (в том числе единиц и троек), потому что каждая ненулевая цифра
на каждом месте встретится одинаковое количество раз.
(можно просто ввести операцию замены троек на единицы и наоборот~--- это будет
разбиение на пары).
Среди чисел от $10000000$ до $19999999$ будет ровно $10000000$ лишних единиц.
После этого все числа начинаются на пару цифр $20$, и это начало не влияет на
разность между числом написанных единиц и числом написанных троек.
Значит надо рассмотреть числа от $1$ до $132013$.
Среди чисел от $1$ до $99999$ единиц и троек снова поровну.
После этого остаются числа от $100000$ до $132013$.
Эти числа содержат $32014$ единиц на первом месте и надо рассмотреть теперь
числа от $1$ до $32013$.
Теперь числа от $1$ до $29999$ содержат $10000$ лишних единиц.
Далее мы рассматриваем числа от $30000$ до $32013$ они содержат $2013$ лишних
троек (на первом месте), $1000$ лишних единиц (числа от $31000$ до $31999$) и
наконец среди чисел от $1$ до $13$ единиц на $4$ больше.
Итого, суммируя все вышесказанное, имеем, что единиц на
$10000000 + 32014 + 10000 - 2014 + 1000 + 4 = 10041004$ больше.
\endproblem
% $problem-source: фольклор, предложил Д.\,Максимов
