\problem
Класс выиграл приз в~виде мешка с~2014 монетами, который в~начале находится
у~старосты (а~у~других детей в~классе денег нет).
При встрече два одноклассника делят деньги поровну между собой, если у~них
четное количество монет в~сумме.
А~если в~сумме количество нечетное, то~одну монету они отдают классному
руководителю, а~остальные делят поровну между собой.
Через некоторое время все монеты оказались у~классного руководителя.
Какое наименьшее количество детей могло быть в~классе?
\solution
\emph{Ответ:} 12 детей.
\par
Покажем индукцией по~$k$, что $k+1$ людей могут избавиться от~$2^k-1$ монет,
находящихся изначально у~одного из~них (и~от~меньшего числа монет).
При $k = 1$ это ясно, индукционный переход от~$k$ к~$k + 1$ тоже ясен: после
первой встречи остается 2~человека, имеющих не~более чем по~$2^{k-1} - 1$ монет
и~$k$~человек без монет.
Тогда $k + 1$ человек по~индукционному предположению избавляются от~своих
монет, после чего то~же делает другой набор из~$k + 1$ человек.
\par
Теперь покажем, что если изначально у~одного из~людей было хотя~бы $2^k$ монет,
и~в~какой-то момент в~обменах участвовало $m$~людей, $m \leq k + 1$,
то~у~каждого из~них хотя~бы $2^{k + 1 - m}$ монет.
Тогда при $m = k + 1$ получим, что $k+1$ человек не~могут избавиться ото всех
монет.
Индукция по~числу обменов.
База для 0~обменов очевидна.
Если в~очередном обмене не~участвует новый человек, то~у~участвовавших остается
хотя~бы по~$2^{k+1-m}$ монет.
Если участвует, то~$m$ увеличивается на~1, а~величина $2^{k + 1 - m}$
уменьшается вдвое, так что утверждение по-прежнему верно.
\endproblem
% $problem-source: Skolornas Matematiktävling 2014
