Вписанная окружность равнобедренного треугольника $ABC$ ($AB = BC$) касается
его боковых сторон $BC$ и $AB$ в точках $E$ и $F$.
Через точку $A$ проведен внутри угла $EAB$ луч, пересекающий вписанную
окружность в точках $P$ и $Q$.
Прямые $EP$ и $EQ$ пересекают прямую $AC$ в точках $P'$ и $Q'$.
Докажите, что $P'A = Q'C$.

\solution
[{\begin{figure}
\centering
    \jeolmfigure[width=0.5\textwidth]{3-solution}
\caption{к задаче \ref{solution:2012/pers-geomt/junior/3}}
\label{fig:solution:2012/pers-geomt/junior/3}
\end{figure}}]%
\label{solution:2012/pers-geomt/junior/3}%
См.\,рис.\,\ref{fig:solution:2012/pers-geomt/junior/3}.
Пусть $D$~--- точка касания вписанной окружности со стороной $AC$,
точка $Q$ лежит на дуге $FE$, не содержащей $D$,
а точка $P$~--- на дуге $FD$, не содержащей $E$.
Тогда точка $Q'$ лежит на продолжении стороны $AC$ за точку $C$,
а $P'$ лежит на продолжении стороны $AC$ за точку $A$.
Докажем, что $\angle P'FA = \angle Q'EC$, тогда треугольники $Q'FA$ и $P'EC$
равны по стороне $FA = (AD = DC) = EC$
(это следует того, что точка касания основания вписанной окружностью
в равнобедренном треугольнике совпадает со серединой)
и равных пар углов $\angle P'FA = \angle Q'EC$ и $\angle FAP' = \angle ECQ'$
(последние углы дополняют до $180^o$ углы при основании равнобедренного
треугольника).
Отсюда $Q'A = CP'$.
Итак, покажем, что $\angle P'FA = \angle Q'EC$.
Заметим, что около четырехугольника $P'FPA$ можно описать окружность, так как
$\angle PP'A = \angle PEF = \angle AFP$
(первое равенство следует из равенства вертикальных углов при параллельных
прямых $FE$ и $P'A$ и секущей $PE$, а второе~--- как равенство угла между
касательной и хордой и угла, опирающегося на эту хорду).
Теперь $\angle Q'EC = \angle QEB = \angle QPE = \angle PP'A = \angle P'FA$
(первое и третье равенства~--- это равенства вертикальных углов,
второе~--- равенство угла между касательной и хордой и угла, опирающегося на
эту хорду, а четвертое~--- равенство углов, опирающихся на одну дугу $P'A$
в вписанном четырехугольнике $P'FPA$).
Полученное равенство завершает доказательство.
\par
\emph{Другое решение.}
Отметим точку касания вписанной окружности со стороной $AC$, точку $D$.
Так как треугольник равнобедренный, то $DA = DC$.
Следовательно, утверждение задачи равносильно равенству $DP' = DQ'$.
Заметим, что четырехугольник $FPDQ$ гармонический, поскольку касательные в
точках $F$ и $D$ пересекаются на его диагонали $PQ$.
Значит, двойное отношение точек $(F, P; D, Q)$ равно $-1$.
Спроектируем эту четверку из точки $E$ на прямую $AC$.
$P$ перейдет в $P'$, $Q$ перейдет в $Q'$, $D$ в себя, $F$ в бесконечно
удаленную точку прямой $AC$.
Следовательно, так как двойное отношение останется гармоническим, $D$ будет
серединой $P'Q'$.

