\ifsolution
\begin{figure}\centering
    \jeolmfigure[width=0.5\textwidth]{solution}
    \caption{к решению задачи \ref{solution:2013/ru/team/junior/4}.}
    \label{fig:solution:2013/ru/team/junior/4}
\end{figure}%
\fi % \ifsolution

\problem\problemscore{6}
Дан равнобедренный треугольник $ABC$ с основанием $AC$.
На продолжении стороны $AB$ за точку $B$ отмечена точка $D$ такая, что
$\angle DCB = \angle BCA$.
На высоте $BH$ треугольника $ABC$ отмечена точка $E$ такая, что $DE = DC$.
Докажите, что $\angle BDE = \frac{1}{3} \angle BDC$. 
\solution
\label{solution:2013/ru/team/junior/4}
Рис.~\ref{fig:solution:2013/ru/team/junior/4}.
Отметим точку $D'$ симметричную точке $D$ относительно прямой $BH$.
Мы получим равнобочную трапецию $AD'DC$, в которой, по условию, диагональ $CD'$
является биссектрисой угла $DCA$.
В силу параллельности оснований трапеции имеем
$\angle D D' C = \angle D' C A = \angle D' C D$, откуда $D'D = DC$.
Теперь продлим ось симметрии трапеции $BH$ до пересечения с $DD'$ в точке $F$.
Тогда $FD = \frac{1}{2} DD' = \frac{1}{2} DE$, откуда мы получаем, что
$\angle FED = 30^\circ$, так как в треугольнике $FED$ катет оказался равен
половине гипотенузы.
Теперь осталось просто аккуратно посчитать углы.
Для этого введем обозначения: $\angle BCD = x$ и $\angle BDE = y$.
Тогда мы получили, что $90^\circ + x + y + 30^\circ = 180^\circ$
(сумма углов треугольника $EBD$.)
Отсюда мы имеем, что $x + y = 60^\circ$ или $3 x + 3 y = 180^\circ$.
Теперь рассмотрим треугольник $DBC$: в нем $\angle BCD = x$ и
$\angle DBC = 2 x$ (как внешний угол треугольника $ABC$), а следовательно,
$\angle BDC = 3 y$ (из суммы углов треугольника $BDC$).
Мы получили то, что требовалось.
\endproblem
% $problem-source: Georgia TST 2005
