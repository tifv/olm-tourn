\problem
На боковой стороне $AB$ прямоугольной трапеции $ABCD$ ($AB \perp BC$)
построена полуокружность (как на диаметре), которая касается боковой стороны
$CD$ в точке $K$.
Диагонали трапеции пересекаются в точке $O$.
Найдите длину отрезка $OK$, если длины оснований трапеции $ABCD$ равны 2 и 3.

\solution
[{\begin{figure}
\centering
    \jeolmfigure[width=0.5\textwidth]{../2-solution}
\caption{к задаче \ref{solution:2012/regatta/junior/geomt/2}}
\label{fig:solution:2012/regatta/junior/geomt/2}
\end{figure}}]%
\label{solution:2012/regatta/junior/geomt/2}%
\emph{Ответ:} $6 / 5$.
См.\,рис.\,\ref{fig:solution:2012/regatta/junior/geomt/2}.
Заметим, что $CK = BC = 2$ и $DA = DK = 3$ по свойствам касательных,
проведённых к окружности.
Так как треугольники $BCO$ и $DAO$ подобны по двум углам
($\angle CBO = \angle ADO$ как накрест лежащий),
то $BO / OD = BC / AD = 2 / 3$.
По доказанному $CK / KD = 2 / 3$.
Значит, по теореме Фалеса $BC \parallel OK$.
Значит, треугольники $DOK$ и $DBC$ подобны и по свойству подобия треугольников
имеем $OK / BC = DK / DC = 3 / 5$, откуда имеем $OK = 6 / 5$.

\endproblem
