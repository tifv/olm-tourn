\providecommand\ifdiagonalsolutiondefined{\iffalse}
\problem
В клетках квадрата $100 \times 100$ вписаны натуральные числа так, что все
$200$ сумм в рядах (строках и столбцах) различны.
Какова наименьшая возможная сумма всех чисел в таблице?

\solution
[\ifdiagonalsolutiondefined\else{\begin{table}\centering
\fbox{\(\begin{array}{ccccccc}
    0 \\
    1 & 2 \\
      & 0 & 4 \\
      &   & 1 & 6 \\
      &   &   &   & \ddots \\
      &   &   &   &   & 196 \\
      &   &   &   &   & 1 & 198
\end{array}\)}
\caption{к задаче \ref{solution:2013/ru/pers-algbr/senior/1}.}
\label{table:solution:2013/ru/pers-algbr/senior/1}
\end{table}}\fi]%
\label{solution:2013/ru/pers-algbr/senior/1}%
\emph{Ответ:} 19950.
\ifdiagonalsolutiondefined
См.~решение задачи \ref{solution:2013/ru/regatta/junior/algbr/4}.
\else
Будем для удобства считать, что числа могу быть нулями
(потом добавим к каждому числу по единице).
Тогда сумма в каждом ряду хотя бы 0, и значит минимально возможные значения
этих сумм от 0 до $199$, а минимально возможная сумма~--- это
$199 \cdot 200 / 4 = 9950$.
Покажем, что существует пример, когда реализуются ровно такие суммы.
Для этого на главной диагонали расставим четные числа от нуля до 198, на второй
диагонали единицы в клеточках через 1
(таблица \ref{table:solution:2013/ru/pers-algbr/senior/1}).
Остальные числа объявим нулями.
Теперь, чтобы получился пример для исходной задачи, добавим к каждому числу в
таблице по единице.
\fi
\endproblem
% $problem-source: Д.\,Максимов
