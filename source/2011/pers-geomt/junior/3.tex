В треугольнике $ABC$ точки $M$ и $L$ на стороне $BC$ --- основания медианы и
биссектрисы соответственно, проведенных из вершины $A$.
Точки $P$ и $Q$ --- основания перпендикуляров, опущенных из точки $L$ на
стороны $AB$ и $AC$ соответственно.
Точка $X$ на медиане $AM$ такова, что $XL \perp BC$.
Докажите, что точки $P$, $X$ и $Q$ лежат на одной прямой. 

\solution
[{\begin{figure}
\centering
    \jeolmfigure[width=0.5\textwidth]{3-solution}
\caption{к задаче \ref{solution:2011/pers-geomt/junior/3}.}
\label{fig:solution:2011/pers-geomt/junior/3}
\end{figure}}]%
\label{solution:2011/pers-geomt/junior/3}%
См.\,рис.\,\ref{fig:solution:2011/pers-geomt/junior/3}.
Заметим, что треугольники $ALP$ и $ALQ$ равны
(по общей гипотенузе $AL$ и равным острым углам $BAL$ и $CAL$).
Отсюда имеем $PL = LQ$ и $AP = AQ$.
Имеем также
$\angle XLQ = 90^\circ - \angle QLC = \angle ACB$
и аналогично $\angle XLP = \angle ABC$.
Теперь мы покажем, что прямые $XL$ и $AM$ делят отрезок $PQ$ в одинаковом
отношении, что и будет означать требуемое.
Медиана $AM$ делит отрезок $PQ$ в отношении, равном отношению синусов углов
$QAX$ и $PAX$ (так как треугольник $APQ$ равнобедренный), а, как известно,
отношение синусов этих углов равно отношению синусов углов $ACB$ и $ABC$
треугольника $ABC$.
Прямая $LX$ делит отрезок $PQ$ в отношении, равно отношению синусов углов $XLQ$
и $XLP$ (так как треугольник $PLQ$ тоже равнобедренный), а эти углы, как мы
поняли, равны углам $ACB$ и $ABC$.
Тем самым мы показали, что $XL$ и $AM$ пересекаются на $PQ$, а значит точка
$X$~--- точка их пересечения лежит на отрезке $PQ$.

