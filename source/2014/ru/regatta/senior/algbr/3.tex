\problem
Барон Мюнхгаузен выписал на~доску $10$~действительных слагаемых, а~их~сумму
записал на~листок.
За~одну операцию он~заменял одно или несколько слагаемых на~доске на~обратные
величины, и~снова выписывал сумму на~листок.
Мог~ли он~в~результате $500$ таких операций выписать на~листок числа
$1, 2, \ldots, 500$?
\solution
\emph{Ответ:} мог.
\par
Пусть барон выберет в~качестве первых $9$~слагаемых такие, чтобы
$\frac{1}{a_0} - a_0 = 1 = 2^0$,
$\frac{1}{a_1} - a_1 = 2^1$,
\ldots,
$\frac{1}{a_8} - a_8 = 2^8$
(все эти уравнения сводятся к~квадратным и~решаются).
Слагаемое $a_9$ он~подберёт так, чтобы $a_0 + \ldots + a_9 = 1$.
Теперь он~может сделать сумму равной любому натуральному числу $n$ от~$1$
до~$511$, представив $n - 1$ как сумму степеней двойки и~заменив в~исходной
сумме на~соответствующие этим степеням слагаемые на~обратные.
Например, чтобы получить $101$, он~представит $101 - 1$ как $2^6 + 2^5 + 2^2$,
и~заменит в~исходной сумме на~обратные $a_2$, $a_5$ и~$a_6$.
\endproblem
% $problem-source: А.\,Шаповалов
