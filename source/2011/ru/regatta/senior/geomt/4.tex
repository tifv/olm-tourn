\ifsolution
\begin{figure}\centering
    \jeolmfigure[width=0.5\textwidth]{solution}
    \caption{к задаче \ref{solution:2011/ru/regatta/senior/geomt/4}.}
    \label{fig:solution:2011/ru/regatta/senior/geomt/4}
\end{figure}
\fi % \ifsolution

\problem
Во вписанном шестиугольнике $ABCDEF$ оказалось, что
$AB = BC$, $CD = DE$ и $EF = FA$.
Докажите, что $S_{ABCDEF} = 2 S_{BDF}$.
\solution
\label{solution:2011/ru/regatta/senior/geomt/4}%
Рис.~\ref{fig:solution:2011/ru/regatta/senior/geomt/4}.
Действительно, $\angle DBE = \angle DBC$ и $\angle EBF = \angle FBA$
(они опираются на равные хорды),
поэтому, когда мы отразим треугольник $BCD$ относительно прямой $BD$ и
треугольник $ABF$ относительно прямой $BF$, точки, симметричные точкам $C$ и
$A$, попадут на прямую $BE$, а значит в одну точку~--- пусть это точка $O$.
Применив аналогичное рассуждение к треугольникам $BCD$ и $DEF$, мы получим, что
и точка, симметричная точке $E$ относительно прямой $FD$, тоже попадает в точку
$O$.
Там самым мы доказали, что треугольник $BDF$ складывается из треугольников
$BCD$, $DEF$ и $FAB$, что и означает требуемое.
\endproblem
