\problem
Есть фонарик, в~который помещается $2$~батарейки, и~есть $10$~батареек,
из~которых $5$~хороших и~$5$~плохих.
За~одну попытку можно вставить в~фонарик $2$~батарейки.
Он~будет светить только когда обе батарейки~--- хорошие.
Как не~позднее чем на~$8$-й попытке наверняка добиться, чтобы фонарик светил?
\solution
Разобьем батарейки на~две тройки и~две пары.
Попробуем по~3 пары из~троек и~каждую отдельную пару.
Если в~тройках все пары плохие, то~в~каждой из~них не~менее двух плохих
батареек.
Если и~отдельные пары плохие, то~в~каждой есть плохая батарейка.
Итого, не~менее 6 плохих батареек.
Противоречие.
Значит, хотя~бы одна из~проверенных пар нам подойдет.
\endproblem
% $problem-source: А.\,Устинов
