% nospell begin
\def\ifsolutiondefined{%
    \csname ifsolution:2014/ru/team/junior/9\endcsname}%
\expandafter\providecommand
    \csname ifsolution:2014/ru/team/junior/9\endcsname
    {\iffalse}%
\def\definesolution{%
    \expandafter\gdef
    \csname ifsolution:2014/ru/team/junior/9\endcsname
    {\iftrue}}%
% nospell end

\begingroup % figure
\ifsolution\def\solutioninclude#1{#1}\else\def\solutioninclude#1{}\fi
\solutioninclude{\ifsolutiondefined\else
\begin{figure}\centering
    \jeolmfigure[width=0.25\textwidth]{solution-n2}
    \qquad
    \jeolmfigure[width=0.25\textwidth]{solution-n3}
    \caption{к решению задачи \ref{2014/ru/team/junior/9:solution}}
    \label{2014/ru/team/junior/9:solution:fig}
\end{figure}%
\fi}%
\endgroup % figure

\problem\problemscore{9}
\label{2014/ru/team/junior/9:problem}%
% duplicated in [2014/ru/team/senior/7]
Для каких $n$ выпуклый $n$-угольник можно разрезать на~выпуклые шестиугольники?
\solution
\label{2014/ru/team/junior/9:solution}%
\emph{Ответ:} $n = 6$ и~$n \geq 8$.
\par
\ifsolutiondefined
См.~решение задачи \ref{2014/ru/team/senior/7:solution}.
\else
\definesolution
Если $n \geq 9$, то~задачу о~разбиении $n$-угольника можно свести к~задаче
о~разбиении $(n-2)$- или $(n-3)$-угольника, отрезав один шестиугольник
(рис.~\ref{2014/ru/team/junior/9:solution:fig}).
Поэтому задача разрешима для любого $n \geq 8$
(и, очевидно, разрешима для $n = 6$).
Докажем, что при $n = 3$, $4$, $5$, $7$ задача неразрешима.
\par
Заметим, что среднее арифметическое всех углов шестиугольников равно
$120^{\circ}$.
В~то~же время в~каждой внутренней вершине, в~которой сходятся вершины хотя~бы
трех шестиугольников, оно не~больше $120^{\circ}$, в~каждой вершине, лежащей
на~стороне (разбиваемого многоугольника или одного из~шестиугольников)~---
не~больше $90^{\circ}$.
Если $n \leq 5$, то~среднее значение углов шестиугольников по~вершинам
разбиваемого многоугольника меньше $120^{\circ}$~--- всё это в~совокупности
дает противоречие.
Если $n = 7$ и~хотя~бы из~одной вершины разбиваемого многоугольника
выходят отрезки внутрь, верно то~же самое: сумма углов в~семи вершинах равна
$5 \cdot 180^{\circ}$,
а~их~количество не~меньше~8, итого среднее меньше $120^{\circ}$.
Наконец, если $n = 7$ и~из~вершин семиугольника не~выходит отрезков внутрь,
то~есть хотя~бы две стороны семиугольника, на~которых отмечены вершины
шестиугольников разбиения, среднее по~этим двум точкам и~вершинам
не~больше чем $(5 \cdot 180^{\circ} + 2 \cdot 180^{\circ}) / 11 < 120^{\circ}$,
дальше то~же самое.
\fi % \ifsolutiondefined
\endproblem
% $problem-source: В.\,Быковский, А.\,Устинов
