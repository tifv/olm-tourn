\ifsolution
\begin{figure}\centering
    \jeolmfigure[width=0.5\textwidth]{solution}
    \caption{к задаче \ref{2014/ru/pers-geomt/junior/3:solution}}
    \label{2014/ru/pers-geomt/junior/3:solution:fig}
\end{figure}%
\fi % \ifsolution

\problem
В~трапеции $ABCD$ сторона~$AB$ параллельна $CD$, $AB > CD$, а~прямая~$BD$
делит угол $\angle ADC$ пополам.
Прямая, проходящая через $C$ параллельно $AD$, пересекает отрезки $BD$ и~$AB$
в~точках $E$ и~$F$ соответственно.
Точка~$O$~--- центр описанной окружности треугольника $BEF$.
Предположим, что $\angle ACO = 60^{\circ}$.
Докажите, что $CF = AF + FO$.
\solution
\label{2014/ru/pers-geomt/junior/3:solution}%
Рис.~\ref{2014/ru/pers-geomt/junior/3:solution:fig}.
Заметим, что из~параллельности следует, что
$\angle CDE = \angle EBF = \angle ADE = \angle DEC$.
Тогда $AF = DC = CE$ и~$EF = FB$.
С~другой стороны, $OF = OE$, поэтому треугольники $OEF$ и $OFB$ равны
и~$\angle OFB = \angle OFE = \angle OEF$.
Следовательно, $\angle AFO = \angle CEO$ и~треугольник $AFO$ равен
треугольнику $CEO$ по~двум сторонам и~углу между ними.
Это означает, что $AO = OC$, $\angle COE = \angle AOF$ и, так как
$\angle ACO = 60^\circ$, то~треугольник $AOC$ равносторониий и
$\angle COA = 60^\circ$.
Тогда $\angle EOF = \angle COA + \angle AOF - \angle COE = 60^\circ$,
и~треугольник $ABC$ также равносторонний, откуда $EF = FO$.
Поэтому $AF + FO = CE + EF = CF$.
\endproblem
% $problem-source: Middle European 2012
