\problem
Пусть
\begin{gather*}
    x_1 + x_2 + x_3 + x_4 = y_1 + y_2 + y_3 + y_4
,\\
    x_1^2 + x_2^2 + x_3^2 + x_4^2
=
    y_1^2 + y_2^2 + y_3^2 + y_4^2
,\\
    x_1^3 + x_2^3 + x_3^3 + x_4^3
=
    y_1^3 + y_2^3 + y_3^3 + y_4^3
.\end{gather*}
Докажите, что
\[
    (x_1 - y_2) (x_1 - y_3) (x_1 - y_4)
=
    (y_1 - x_2) (y_1 - x_3) (y_1 - x_4)
.\]
\solution
Пусть для $1 \leq m \leq 4$
\begin{align*}&
    \sigma_m(x)
=
    \sum\limits_{\mathllap{1 \leq} i_1 < \ldots < i_m \mathrlap{\leq 4}}
        x_{i_1} \ldots x_{i_m}
,\\&
    \sigma_m(y)
=
    \sum\limits_{\mathllap{1 \leq} i_1 < \ldots < i_m \mathrlap{\leq 4}}
        y_{i_1} \ldots y_{i_m}
\end{align*}
--- элементарные симметрические многочлены от наборов $x$ и $y$, и
\[
    s_j(x)
=
    \sum\limits_{k = 1}^{4}
        x_k^j
,\quad
    s_j(y)
=
    \sum\limits_{k = 1}^{4}
        y_k^j
\qquad
    (j \geq 1)
.\]
Тогда из равенств $s_1(x) = s_1(y)$, $s_2(x) = s_2(y)$, $s_3(x) = s_3(y)$
следует равенство элементарных симметрических многочленов
от наборов $x$ и $y$:
\begin{equation} \label{2013/ru/pers-algbr/senior/3:solution:eq:newton}
    \sigma_1(x) = \sigma_1(y)
,\quad
    \sigma_2(x) = \sigma_2(y)
,\quad
    \sigma_3(x) = \sigma_3(y)
.\end{equation}
Это можно проверить, например, с помощью формул Ньютона
\begin{align*}&
    \sigma_1 = s_1
,\\&
    \sigma_2 = \sigma_1 s_1 - s_2
,\\&
    \sigma_3 = \sigma_2 s_1 - \sigma_1 s_2 + s_3
.\end{align*}
Рассмотрим многочлены
\[
    f(z) = (z - x_1) (z - x_2) (z - x_3) (z - x_4)
,\quad
    g(z) = (z - y_1) (z - y_2) (z - y_3) (z - y_4)
.\]
Из равенств~\eqref{2013/ru/pers-algbr/senior/3:solution:eq:newton} и теоремы Виета
следует, что многочлен $h(z) = f(z) - g(z)$ является константой.
Значит, $-g(x_1) = h(x_1) = h(y_1) = f(y_1)$, то есть
\begin{equation}
    - (x_1 - y_1) (x_1-y_2) (x_1 - y_3) (x_1 - y_4)
=
    (y_1 - x_1) (y_1 - x_2) (y_1 - x_3) (y_1 - x_4)
.\end{equation}
Если $x_1 \neq y_1$, то, сокращая на $(y_1 - x_1)$, приходим к нужному
равенству
\begin{equation} \label{2013/ru/pers-algbr/senior/3:solution:eq:desired}
    (x_1 - y_2) (x_1 - y_3) (x_1 - y_4)
=
    (y_1 - x_2) (y_1 - x_3) (y_1 - x_4)
.\end{equation}
Если же $x_1 = y_1$, то из системы
\begin{gather*}
    x_2 + x_3 + x_4 = y_2 + y_3 + y_4
,\\
    x_2^2 + x_3^2 + x_4^2 = y_2^2 + y_3^2 + y_4^2
,\\
    x_2^3 + x_3^3 + x_4^3 = y_2^3 + y_3^3 + y_4^3
,\end{gather*}
как и раньше, получаем равенство элементарных симметрических многочленов от
наборов $(x_2, x_3, x_4)$ и $(y_2, y_3, y_4)$.
По теореме Виета эти два набора чисел являются корнями одного и того же
кубического уравнения, то есть совпадают с точностью до перестановки.
Следовательно, равенство~\eqref{2013/ru/pers-algbr/senior/3:solution:eq:desired} будет
справедливо и в этом случае.
\endproblem
% $problem-source: А.\,Устинов
