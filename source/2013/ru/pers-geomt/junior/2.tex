\ifsolution
\begin{figure}\centering
    \jeolmfigure[width=0.5\textwidth]{solution}
    \caption{к решению задачи \ref{2013/ru/pers-geomt/junior/2:solution}.}
    \label{2013/ru/pers-geomt/junior/2:solution:fig}
\end{figure}%
\fi % \ifsolution

\problem
Шестиугольник $ABCDEF$ вписан в окружность.
Диагонали $AD$ и $BE$ пересекаются в точке $X$, диагонали $AD$ и $CF$~--- в
точке $Y$, а диагонали $BE$ и $CF$~--- в точке $Z$.
Оказалось, что $AX = DY$ и $CY = FZ$.
Докажите, что $BX = EZ$.
\solution
\label{2013/ru/pers-geomt/junior/2:solution}%
Рис.~\ref{2013/ru/pers-geomt/junior/2:solution:fig}.
Заметим, что $AX \cdot XD = AY \cdot YD$ (так как $AX = YD$ и $AY = XD$).
Далее, $AY \cdot YD = FY \cdot YC$ (так как произведения отрезков хорд равны).
Аналогично $FY \cdot YC = FZ \cdot ZC = EZ \cdot ZB$.
С другой стороны, $AX \cdot XD = BX \cdot XE$.
Используя все эти равенства, мы получаем $BX \cdot XE = BZ \cdot ZE$.
Отсюда легко вытекает, что $BX = ZE$, что и требовалось доказать.
\endproblem
% $problem-source: Д.\,Максимов, Ф.\,Петров
