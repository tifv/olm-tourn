В футбольном турнире принимали участие $30$ команд.
По окончании турнира оказалось, что среди любых трех команд найдутся две,
которые в трех матчах внутри этой тройки команд набрали поровну очков
(за победу дается $3$ очка, за ничью~--- $1$, за поражение~--- $0$).
Какое минимальное количество ничьих может быть в таком турнире?

\solution
Выберем произвольную команду $A$ и разобьем все команды на три группы:
команды, выигравшие у $A$, команды, проигравшие $A$, и команды, сыгравшие с
$A$ вничью.
К последней группе мы добавим и команду $A$.
Заметим, что любые две команды из одной группы сыграли вничью~--- это легко
следует из условия.
Пусть в группах $a$, $b$ и $c$ команд.
Тогда число ничьих не меньше, чем
\[
    \dfrac{a (a - 1)}{2} +
    \dfrac{b (b - 1)}{2} +
    \dfrac{c (c - 1)}{2}
=
    \dfrac{1}{2}(a^2 + b^2 + c^2 - 30)
.\]
Сумма чисел $a$, $b$ и $c$ равна 30, поэтому сумма квадратов минимальна, если
все числа равны.
Отсюда следует, что ничьих не меньше, чем 135.
\emph{Пример} получается, когда описанные группы действительно содержат ровно по 10
команд и команды групп выигрывают друг у друга по циклу.
\emph{Ответ:} $135$.

