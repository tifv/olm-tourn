\problem
Пусть $R(n)$~--- количество представлений натурального числа~$n$ в~виде суммы
$n = a + b$ двух простых чисел (например, $R(3) = 0$, $R(4) = 1$, $R(10) = 3$,
поскольку $10 = 3 + 7 = 5 + 5 = 7 + 3$).
Докажите, что если $p < q$~--- два последовательных простых числа, то~сумма
\[
    2 R (q - 2) + 3 R (q - 3) + 5 R (q - 5) + \ldots + p \cdot R (q - p)
\]
кратна $q$.
\solution
Случай $q = 3$ очевиден, пусть $q > 3$.
Обозначим искомую сумму через $S$.
Заметим, что она равна сумме от~$a$ по~всем решениям уравнения $q = a + b + c$
в~простых числах. 
Из~соображений симметрии понятно, что сумма от~$b$ по~тем~же тройкам также
равна $S$, как и~сумма от~$c$.
Таким образом, $3S$ есть сумма от~$a + b + c$ по~таким тройкам, 
то~есть $3 S = m q$, где $m$~--- количество этих троек.
Итак, $3 S$ кратно $q$, а~значит и~$S$ кратно $q$. 
\endproblem
% $problem-source: В.\,Быковский
