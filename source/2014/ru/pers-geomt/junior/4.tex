\ifsolution
\begin{figure}\centering
    \jeolmfigure[width=0.5\textwidth]{solution}
    \caption{к задаче \ref{2014/ru/pers-geomt/junior/4:solution}}
    \label{2014/ru/pers-geomt/junior/4:solution:fig}
\end{figure}%
\fi % \ifsolution

\problem
На~гипотенузе~$AB$ прямоугольного треугольника $ABC$ выбраны такие точки
$M$ и~$N$, что $AM < AN$.
Прямая, проходящая через $M$ и~перпендикулярная $CN$, пересекает прямую~$AC$
в~точке~$P$.
Прямая, проходящая через $N$ и~перпендикулярная $CM$, пересекает прямую~$BC$
в~точке $Q$.
Докажите, что описанные окружности треугольников $APM$ и~$BNQ$ и прямая~$PQ$
имеют общую точку.
\solution
\label{2014/ru/pers-geomt/junior/4:solution}%
Рис.~\ref{2014/ru/pers-geomt/junior/4:solution:fig}.
Пусть $D$~--- вторая точка пересечения окружностей, описанных вокруг треугольников
$ACN$ и~$BCM$.
Тогда
\begin{align*}
    \angle ADM
={}&
    \angle ADC - \angle MDC
=
    \angle ANC - \angle MBC
=\\={}&
    \angle NCB
=
    90^\circ - \angle ACN
=
    \angle APM
\,.\end{align*}
Следовательно, точки $A$, $M$, $P$ и~$D$ лежат на~одной окружности.
Аналогично, точки $B$, $N$, $Q$ и $D$ лежат на одной окружности.
С другой стороны,
\begin{align*}
    \angle PDQ
={}&
    \angle PDM + \angle MDC +\angle CDN + \angle NDQ
=\\={}&
    \angle CAM + \angle MBC +\angle CAN + \angle NBC
=
    180^\circ
\,.\end{align*}
Это означает, что точки $P$, $D$ и $Q$ лежат на одной прямой.
\endproblem
% $problem-source: Iran 2014
