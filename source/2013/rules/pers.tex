\subsection*{Правила подсчета личного рейтинга}

% $build$matter:
% - - .[no-header]

\paragraph{Рейтинг участника на турнире}
равен сумме рейтингов участника за каждое из трех личных зачетных соревнований
в отдельности.
Для равномерного влияния каждого мероприятия устанавливается, что личный
рейтинг по каждому мероприятию~--- целое неотрицательное число от 0 до 15
включительно, причем достижимые значения расположены почти непрерывно.
Таким образом, наибольший возможный личный рейтинг на Турнире равен 45,
наименьший~--- 0.

\paragraph{Рейтинг участника на личной олимпиаде по алгебре и теории чисел}
равен количеству его баллов, умноженной на $7/13$.
Если получившееся число нецелое (а с вероятностью около $12/13$ так и будет), то оно
округляется до ближайшего целого.
Каждая из 4 задач стоит 7 баллов, итого максимальная возможная сумма баллов
участника равна 28, после умножения~--- $15\frac{1}{13}$, а после
округления~--- 15.

\paragraph{Рейтинг участника на личной олимпиаде по комбинаторике и логике}
равен количеству его баллов, умноженной на 3.
Полное решение каждой из 5 задач стоит 1 балл, неполное~--- ноль баллов.

\paragraph{Рейтинг участника на личной олимпиаде по геометрии}
вычисляется так же, как рейтинг участника на
личной олимпиаде по алгебре и теории чисел.

