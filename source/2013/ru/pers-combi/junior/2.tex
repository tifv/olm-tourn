\problem
Центр правильного 55-угольника соединили отрезками со всеми его вершинами.
Вместе со сторонами 55-угольника получилось 110 отрезков.
Петя и Вася по очереди красят эти отрезки, по одному за ход.
Можно красить отрезок, только если он и выходящие из его концов отрезки не
окрашены.
Кто не может сделать ход~--- проигрывает.
Начинает Петя.
Кто из них может выигрывать, как бы ни играл соперник?
\solution
\emph{Ответ:} Вася.
% "уголь\-ника" -> "угольника"
Разобьем отрезки на пары, сопоставив каждой стороне 55-уголь\-ника отрезок,
соединяющий противоположную вершину с центром.
В ответ на первый ход Пети Вася красит второе отрезок из той же пары.
После этого можно красить только некоторые стороны, которые разбиваются на два
одинаковых не связанных между собой набора.
На ход Пети в любой из наборов Вася повторяет его в другом наборе.
\endproblem
% $problem-source: А.\,Шаповалов
