\ifsolution
\begin{figure}\centering
    \jeolmfigure[width=0.4\textwidth]{solution}
    \caption{on problem \ref{2013/en/regatta/senior/geomt/4:solution}.}
    \label{2013/en/regatta/senior/geomt/4:solution:fig}
\end{figure}%
\fi % \ifsolution


\problem
A nonisosceles triangle $ABC$ is inscribed in a circle $\Gamma$.
The bisector of angle $A$ meets $BC$ at $E$.
Let $M$ be the midpoint of the arc $BAC$.
The line $ME$ intersects $\Gamma$ again at $D$.
Show that the circumcentre of triangle $AED$ coincides with the intersection
point of the tangent to $\Gamma$ at $D$ and the line $BC$.
\solution
\label{2013/en/regatta/senior/geomt/4:solution}%
\ifx\wideparen\undefined\else\let\widearc\wideparen\fi
\ifx\overparen\undefined\else\let\widearc\overparen\fi
Fig. \ref{2013/en/regatta/senior/geomt/4:solution:fig}.
Assuming $AB < AC$ we get that point $D$ on the arc $BC$ is closer to $B$ than
to $C$.
The intersection point $O$ of the tangent at $D$ and the line $BC$ lies in the
same semiplane with respect to line $DE$ as $A$.
Then
\(
    \angle ODE = \frac{1}{2} \widearc{DBM}
=
    \frac{1}{2} (\widearc{BD} + \widearc{BM})
=
    \frac{1}{2} (\widearc{BD} + \widearc{MC})
=
    \angle OED
\).
Thus, it is enough to show that $\angle DOE = 2\angle DAE$.
This equality is obvious since
$\angle DOE = \frac{1}{2} (\widearc{CD} - \widearc{BD}) = \widearc{DX}$,
where by $X$ we denote the intersection point of $AE$ with $\Gamma$.
\endproblem
% $problem-source: Italy TST 2002
