\problem
Дан произвольный треугольник.
На каждой стороне треугольника отмечено 10 точек.
Каждая вершина треугольника соединена отрезками со всеми отмеченными точками
противолежащей стороны.
На какое наибольшее число частей отрезки могли разделить треугольник?
\solution
Будем проводить отрезки по одному.
Число добавленных частей равно числу частей, на которые разбивается отрезок
точками пересечения.
Значит, чем больше точек пересечения, тем больше частей.
Максимум частей будет, когда каждый отрезок пересекает каждый из отрезков,
проведенный из других вершин, причем в своей точке.
Для этого достаточно проводить отрезки не через имеющиеся точки пересечения.
Для 10 отрезков из первой вершины каждый добавит по одной части.
Для 10 отрезков из второй вершины на каждом будет по 10 точек пересечения,
значит, каждый добавит по 11 частей.
Для 10 отрезков из третьей вершины на каждом будет по 20 точек пересечения,
значит, каждый добавит по 21 частей.
Итого, $1 + 10 + 10 \cdot 11 + 10 \cdot 21 = 331$.
\emph{Ответ:} 331.
\endproblem
