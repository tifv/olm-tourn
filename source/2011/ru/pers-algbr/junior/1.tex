\problem
На доске записаны числа от $1$ до $2011$.
Двое играющих по очереди стирают по одному числу, пока не останутся два числа
$p$ и $q$.
Если ни одно из уравнений $x^2 + p x + q = 0$ и $x^2 + q x + p = 0$ не имеет
целых корней, то ходивший первым выигрывает.
Может ли второй ему помешать?
\solution
\emph{Ответ:} нет.
\par
Заметим, что если $p$ и $q$ будут оба нечетными, то оба уравнения не будут
иметь решения, так как
\[
   x^2 + p x + q
\equiv
   x + p x + q
\equiv
   x (p + 1) + q
\equiv
   q
\equiv
   1
\pmod{2}
.\]
Аналогично для уравнения $x^2 + q x + p = 0$.
Заметим, что первый игрок к тому моменту, когда останется $2$ числа, сделает
ровно $1010$ ходов, и, значит, сможет выкинуть все четные числа.
Тем самым, как мы продемонстрировали, он выиграет.
\endproblem
