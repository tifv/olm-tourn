На сторонах $AB$, $BC$ и $AC$ треугольника $ABC$ выбраны соответственно точки
$K$, $L$ и $M$.
Оказалось, что радиусы описанных окружностей треугольников
$AKM$, $BKL$, $CLM$ и $KLM$ равны.
Докажите, что треугольники $ABC$ и $KLM$ подобны.

\solution
\label{solution:2012/regatta/senior/geomt/1}%
Пусть $R$~--- указанный радиус.
По теореме синусов
\[
    KL = 2 R \cdot \sin \angle B = 2 R \cdot \sin \angle KML
.\]
Отсюда $\angle KML = \angle B$ или $\angle KML = 180^\circ - \angle B$
(дополнителен к $\angle B$).
Аналогично,
$\angle KLM = \angle A$ или $\angle KLM = 180^\circ - \angle A$,
$\angle LKM = \angle C$ или $\angle LKM = 180^\circ - \angle C$.
Если дополнительных углов нет, то углы равны и треугольники подобны.
Если дополнительный~--- один, скажем, $180^\circ - \angle A$, то,
суммируя углы треугольников $ABC$ и $KLM$, получим
$\angle A = 180^\circ - \angle A$~--- опять все углы равны.
Если дополнительных два, то получим, например,
$\angle B + \angle C = 360^\circ - \angle B - \angle C$,
откуда $\angle B + \angle C = 180^\circ$, что невозможно.
При трех дополнительных углах
$\angle A + \angle B + \angle C = 540^\circ - \angle A - \angle B - \angle C$,
противоречие.

