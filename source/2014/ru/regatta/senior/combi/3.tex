\problem
При любой раскраске клеток клетчатой доски в~черный и~белый цвета доска делится
на~одноцветные области (при шахматной раскраске все области~--- одноклеточные).
Каждым ходом Петя выбирает одну область и~перекрашивает её~в~противоположный
цвет.
Перекрашенная область склеивается в~одну с~соседними областями того~же цвета,
и~число областей уменьшается.
За~какое наименьшее число ходов Петя сможет из~шахматно раскрашенной доски
$13 \times 13$ сделать одноцветную доску? 
\solution
\emph{Ответ:} за~$12$.
\par
\emph{Пример.}
Каждым ходом перекрашиваем область, содержащую центральную клетку. 
\par
\emph{Оценка.}
Склеиваем одноцветные клетки в~одну область не~только если они соприкасаются
сторонами, но~и~соприкасаются правым нижним и~левым верхним углами.
Тогда у~нас изначально всего $17$ областей-диагоналей, их~граф соседства
образует цепочку из~$25$ рёбер.
При каждом перекрашивании уничтожается не~более двух рёбер, значит, нужно
не~менее $12$ перекрашиваний.
\endproblem
% $problem-source: А.\,Шаповалов
