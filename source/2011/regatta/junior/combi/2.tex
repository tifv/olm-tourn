Двое игроков по очереди проводят красные и синие прямые на плоскости, так,
чтобы они не проходили через точки пересечения других прямых и не были им
параллельны.
При этом каждый игрок на каждом ходу выбирает, будет проведенная им прямая
красной или синей.
Игра заканчивается, когда оба игрока проведут по $20$ прямых.
Второй старается сделать как можно больше точек, где пересекаются прямые
разного цвета, первый ему мешает.
Какого наибольшего количества таких точек может гарантированно добиться второй?

\solution
\emph{Ответ:}
$400$.
Пусть синих прямых получилось $a$, а красных $b$.
<<Разноцветных>> пересечений получилось ровно $a b$.
Тогда
\(
    4 a b
\leq
    4 a b - (a + b)^2 + 1600
=
    -(a - b)^2 + 1600
\leq
    1600
\),
то есть $a b \leq 400$.
Результат $400$ достигается, когда красных и синих прямых по 10.
На каждую синюю прямую соперника второй может провести красную, и наоборот.

