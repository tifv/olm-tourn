\ifsolution
\begin{figure}\centering
    \jeolmfigure[height=0.5\textwidth]{solution}
    \caption{к задаче \ref{solution:2013/ru/team/senior/7}.}
    \label{fig:solution:2013/ru/team/senior/7}
\end{figure}
\fi % \ifsolution

\problem\problemscore{6}
На сторонах $AB$, $BC$, $CD$ и $DA$ ромба $ABCD$ соответственно выбраны точки
$E$, $F$, $G$ и $H$ таким образом, что отрезки $EF$ и $GH$ касаются вписанной в
ромб окружности.
Докажите, что прямые $EH$ и $FG$ параллельны.
\solution
\label{solution:2013/ru/team/senior/7}
Рис.~\ref{fig:solution:2013/ru/team/senior/7}.
Обозначим через $K$, $L$, $M$ и $N$ точки касания вписанной окружности со
сторонами ромба $AB$, $BC$, $CD$ и $DA$ соответственно.
Так как $ABCD$ ромб отрезки $KM$ и $NL$~--- диаметры вписанной окружности.
Через $X$ обозначим точку касания $EF$ со вписанной окружностью, а через
$Y$~--- точку касания $GH$.
Пусть лучи $XK$ и $YN$ пересекаются в точке $P$, а лучи $XL$ и $YM$~--- в точке
$Q$.
Точка $E$~--- поляра прямой $KX$, точка $H$~--- поляра прямой $NY$, откуда мы
получаем, что точка $P$~--- поляра прямой $EH$.
Аналогично, точка $Q$~--- поляра прямой $FG$.
Значит, достаточно доказать, что центр окружности лежит на прямой $PQ$
(потому что из~этого будет следовать, что обе данные прямые перпендикулярны
прямой $PQ$).
Последнее утверждение будет следовать из теоремы Паскаля для шестиугольника
$KMYNLXK$.
\endproblem
%% $problem-source: Georgia TST 2005
% Устинов говорит, боян
% $problem-source: фольклор
