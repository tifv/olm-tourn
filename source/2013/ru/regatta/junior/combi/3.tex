\problem
Шахматная доска покрыта 32 домино (каждое домино покрывает ровно два поля).
Докажите, что эти домино можно повернуть на 90 или на 180 градусов
(каждое~--- вокруг центра одной из закрываемых им клеток, поворачивать можно
независимо друг от друга и в любую сторону), чтобы по-прежнему вся доска была
покрыта.
\solution
Сначала найдем другое покрытие домино, в котором никакая фигурка не расположена
так же, как в исходном покрытии.
Для этого разобьем доску на клетки $2 \times 2$.
Если в клетке в исходном покрытии лежит целиком горизонтальное домино, в новом
кладем на нее два вертикальных, иначе~--- два горизонтальных.
В результате ни одно домино в новом покрытии не совпадет по положению с
исходным.
\par
Теперь покажем, что можно осуществить требуемые повороты исходных фигурок и
перевести их в новые.
Вспомним о шахматной раскраске доски и повернем каждое исходное домино
так, чтобы оно заняло положение нового домино, закрывающего ту же белую
клетку.
\endproblem
% $criteria: >
%   Если доказано, что существует второе разбиение, где ни одна доминошка не
%   совпадает с первой~--- дать 2 балла за частичное продвижение.
% $problem-source: А.\,Шаповалов
