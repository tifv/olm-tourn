% nospell begin
\def\ifsolutiondefined{%
    \csname ifsolution:2013/ru/team/junior/7\endcsname}
\expandafter\providecommand
    \csname ifsolution:2013/ru/team/junior/7\endcsname
{\iffalse}
\def\definesolution{%
    \expandafter\gdef
    \csname ifsolution:2013/ru/team/junior/7\endcsname
    {\iftrue}}
% nospell end

\problem\problemscore{8}
% дублирутеся в 5 старших
Три бегуна тренируются на одной прямой дорожке.
Их скорости различны, но постоянны.
Добежав до конца дорожки, бегун мгновенно разворачивается и бежит обратно,
затем разворачивается на другом конце, и т.\,д.
Пять раз случалось, что все бегуны оказывались в одной точке.
Докажите, что такие встречи всех троих будут продолжаться и впредь.
\solution
\label{solution:2013/ru/team/junior/7}
\ifsolutiondefined
См.~решение задачи \ref{solution:2013/ru/team/senior/5}.
\else
\definesolution
Пусть длина дорожки равна $0.5$.
Пусть бегун находится на расстоянии $s$ от начала дорожки.
Его координатой назовем $s$ когда он бежит от начала и $1 - s$ когда он бежит к
началу.
При встрече двоих сумма их координат равна $1$, при обгоне координаты равны.
Если бегун стартовал из точки $b$ и бежит со скоростью $v$, то в момент времени
$t$ его координата равна $[b + v t]$.
Пометим точки старта и скорости бегунов 1, 2, 3 соответственными индексами.
Тогда встреча 1-го и 2-го бегунов происходит когда сумма
$v_1 t + b_1 + v_2 t + b_2$~--- целая, а обгоны одного из них другим~---
когда разность $v_1 t + b_1 - (v_2 t + b_2)$ целая.
Аналогично для 1-го и 3-го.
Парные совпадения положений бывают двух типов~--- встречи (лицом к лицу) или
обгоны.
Соответственно, тройные совпадения бывают четырех типов: первый встречает
обоих, встречает 2-го и обгоняет 3-го, встречает 3-го и обгоняет 2-го, обгоняет
обоих.
Ясно, что какой то тип тройного совпадения произойдет дважды, скажем, в моменты
$t$ и $T$ ($T > t$).
Рассмотрим для определенности случай, когда 1-й и 2-й при этом встретились, а у
1-го и 3-го был обгон.
Тогда
$v_1 t + b_1 + v_2 t + b_2$,
$v_1 T + b_1 + v_2 T + b_2$,
$v_1 t + b_1 - (v_3 t + b_3)$ и
$v_1 T + b_1 - (v_3 T + b_3)$~--- целые.
Покажем, что тогда и в момент $2 T - t = T + (T - t)$ было тройное совпадение.
Действительно,
\(
    v_1 (2 T - t) + b_1 + v_2 (2 T - t) + b_2
=
    2 (v_1 T + b_1 + v_2 T + b_2) - (v_1 t + b_1 + v_2 t + b_2)
\)~--- целое и
\(
    v_1 (2 T - t) + b_1 - (v_3 (2 T - t) + b_3)
=
    2 (v_1 T + b_1 - (v_3 T + b_3)) - (v_1 t + b_1 - (v_3 t + b_3))
\)~--- целое.
Положение первого совпало с обоими другими, значит произошло тройное совпадение
того же типа.
\fi
\endproblem
% $problem-source: А.\,Шаповалов
