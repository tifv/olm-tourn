\ifsolution
\begin{figure}\centering
    \jeolmfigure[width=0.5\textwidth]{solution}
    \caption{к решению задачи \ref{solution:2011/ru/team/senior/4}.}
    \label{fig:solution:2011/ru/team/senior/4}
\end{figure}%
\fi % \ifsolution

\problem\problemscore{6}
Отрезок $AL$~--- биссектриса треугольника $ABC$, $I$ и $J$~--- центры
окружностей, вписанных в треугольники $ABL$ и $ACL$ соответственно.
Прямая $IJ$ пересекает стороны $AB$ и $AC$ в точках $C_1$ и $B_1$
соответственно.
Докажите, что прямые $B B_1$, $C C_1$ и $AL$ пересекаются в одной точке.
\solution
\label{solution:2011/ru/team/senior/4}%
Рис.~\ref{fig:solution:2011/ru/team/senior/4}.
Пусть прямая $IJ$ пересекает сторону $BC$ в точке $P$, а прямые $AI$ и $AJ$
пересекают сторону $BC$ в точках $X$ и $Y$ соответственно.
Напишем теорему Менелая для треугольника $AXY$:
\(
    AJ / JY
    \cdot
    YP / PX
    \cdot
    XI / IA
=
    1
\).
Так как $LI$ и $LJ$~--- биссектрисы углов $ALX$ и $ALY$ соответственно, то мы
имеем по свойству биссектрисы:
$AJ / JY = AL / LY$ и $XI / IA = XL / AL$.
Подставляя это в написанное выше равенство из теоремы Менелая, будем иметь
$YL / LX = YP / XP$.
Так как $AL$ также является биссектрисой треугольника $AXY$ мы знаем, что
$YL / LX = AY / AX$, то есть точка $P$ делит отрезок $XY$ внешним
образом в отношении, равном отношению сторон $AX$ и $AY$ треугольника $AXY$.
Но это означает, что точка $P$~--- основание внешней биссектрисы треугольника
$AXY$, а, значит, и треугольника $ABC$, то есть
$CP / PB = CA / AC = CL / LB$.
Применяя теперь теорему Менелая к треугольнику $ABC$ мы получим, что
\(
    A C_1 / C_1 B
    \cdot
    BP / PC
    \cdot
    C B_1 / B_1 A
=
    1
\).
Подставляя сюда отмеченное выше равенство, мы получаем условие теоремы Чевы для
точек $B_1$, $C_1$ и $L$, что и доказывает требуемое.
\endproblem
