\problem
На~доске написано $100$ различных целых чисел.
Каждое число Вася возвел то~ли в~квадрат, то~ли в~куб и~записал получившиеся
$100$ чисел на~второй доске.
Затем Вася возвел то~ли в~квадрат, то~ли в~куб каждое из~чисел на~второй доске
(каждый раз выбирая степень наугад) и~записывал результаты на~третьей доске.
Какое наименьшее количество различных чисел могло быть записано на~третьей
доске?
\solution
\emph{Ответ:} $20$.
\par
Каждое число на~$3$-й доске может быть $4$-й, $6$-й или $9$-й степенью числа
с~$1$-й доски.
Следовательно, оно могло получиться из~корней $4$-й, $6$-й или $9$-й степени.
Из~положительного числа есть $2$ корня $4$-й степени
(положительный и~отрицательный), $2$ корня $6$-й степени и~один корень $9$-й
степени.
Итого, число на~$3$-й доске могло получиться не~более, чем из~$5$ различных
чисел.
Пример: выберем $20$ различных простых чисел, и~для каждого выбранного $p$
запишем на~$1$-ю доску $p^4$,  $p^6$, $-p^6$, $p^9$, $-p^9$.
Из~каждой такой пятерки на~$3$-й доске получим $p^{36}$.
\endproblem
% $problem-source:
%   А.\,Шаповалов по мотивам дополнительной задачи Кенгуру 2014
