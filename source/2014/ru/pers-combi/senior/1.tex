%\ifproblem
%\begin{figure}[p]\centering
%    \jeolmfigure[width=0.5\textwidth]{octahedron}
%%    \caption{к задаче \ref{2014/ru/pers-combi/senior/1:problem}}
%%    \label{2014/ru/pers-combi/senior/1:problem:octahedron}
%\end{figure}%
%\fi % \ifsolution

\problem
\label{2014/ru/pers-combi/senior/1:problem}%
Муха ползает по~ребрам октаэдра, причем, выползая из~одной
вершины по~некоторому ребру, она доползает до~другой (а~не~возвращается обратно
с~полпути).
Может~ли однажды оказаться, что она в~одной из~вершин побывала 2014 раз,
а~в~каждой из~остальных~--- 650 раз?
\solution
\emph{Ответ:} нет, не~может.
\par
Пусть так случилось.
В~двух противоположных вершинах она побывала 2664 раза, после каждого такого
раза (кроме последнего) она должна была попасть в~одну из~4 других вершин,
но~в~них она была всего лишь 2600 раз.
Противоречие.
\endproblem
% $problem-source: Math League Tournaments% (China part)
