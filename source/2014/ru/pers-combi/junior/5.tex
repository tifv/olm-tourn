\problem
Фокусник и~ассистент хотят показать следующий фокус.
В~отсутствие фокусника зрители пишут в~секторах круглого барабана $n$~чисел,
каждое из~которых $0$ или $1$.
Затем ассистент закрывает одно из~чисел черной карточкой и~зрители вращают
барабан.
Зашедший после этого фокусник должен угадать число под карточкой.
При каких $n$ фокусник и~ассистент могут договориться, чтобы этот фокус
гарантированно получался? 
\solution
\emph{Ответ:} при $n > 2$.
\par
Легко видеть, что при $n = 1, 2$ ничего не~получится.
Пусть $n$ нечетно.
Тогда найдутся два одинаковых числа подряд.
Ассистент закроет из~них то, что идет вторым по~часовой стрелке.
Фокусник гарантированно угадает число, назвав число, написанное предыдущим
перед карточкой по~часовой стрелке.
\par
Пусть $n > 2$ четно.
Если есть три одинаковых числа подряд, то~ассистент закроет из~них последнее
по~часовой стрелке.
Если есть две отдельные пары одинаковых чисел, идущих подряд, то~ассистент
в~одной из~них закроет второе по~часовой стрелке.
Если~же не~найдется ни~такой тройки, ни~таких двух пар, то~это значит, что
числа попросту чередуются.
Тогда ассистент закрывает любое из~них.
Если зашедший фокусник видит, что есть пара одинаковых чисел подряд,
то~он~понимает, что реализовался один из~первых двух случаев, и~называет число,
идущее перед карточкой по~часовой стрелке.
В~противном случае фокусник понимает, что реализовался последний случай,
и~называет число, отличное от~любого из~соседей. 
\endproblem
% $problem-source: А.\,Устинов
