\ifsolution
\begin{figure}\centering
    \jeolmfigure[width=0.5\textwidth]{solution}
    \caption{к решению задачи \ref{2013/ru/team/junior/5:solution}.}
    \label{2013/ru/team/junior/5:solution:fig}
\end{figure}%
\fi % \ifsolution

\problem\problemscore{6}
В~треугольнике $ABC$ выбрана точка~$P$ такая, что $\angle ABP = \angle CPM$,
где $M$~--- середина стороны~$AC$.
Прямая $MP$ повторно пересекает описанную окружность треугольника $APB$
в~точке~$Q$.
Докажите, что $QA = PC$.
\solution
\label{2013/ru/team/junior/5:solution}%
Рис.~\ref{2013/ru/team/junior/5:solution:fig}.
Заметим, что $\angle AQP = \angle ABP$ (так как опираются на одну дугу $AP$),
откуда $\angle AQP = \angle CPM$.
Теперь продлим медиану $PM$ треугольника $APC$ до точки $P'$ такой, что
$MP = MP'$.
Мы получим параллелограмм $APCP'$, в котором $\angle CPP' = \angle AP'P$.
Значит, мы получили равнобедренный треугольник $AQP'$ и значит $QA = AP' = PC$,
что и требовалось. 
\endproblem
% $problem-source: unknown
