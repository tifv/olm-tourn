\problem
Имеется куча из нескольких конфет.
Сначала Малыш съедает из этой кучи одну конфету, затем Карлсон съедает из этой
кучи две конфеты, затем Малыш~--- три, Карлсон~--- четыре и так далее.
Если в какой-то момент число оставшихся конфет меньше, чем должен съесть Малыш
или Карлсон очередным ходом, то он доедает все конфеты.
Оказалось, что Малыш съел 101 конфету.
Сколько всего конфет было изначально?

\solution
\emph{Ответ:} 211.
Заметим, что число конфет, съеденных Малышом,~--- это сумма первых нескольких
нечетных чисел, то есть точный квадрат.
Но в итоге Малыш съел 101 конфету, а значит в свой последний ход Малыш доел
оставшиеся конфеты.
Значит, в этот момент могла остаться только одна конфета
(иначе бы последним ходом Малыш съел не менее 20 конфет и значит у него должно
было уже быть не менее $10^2$ конфет, что не так).
Значит Малыш съел от 1 до 19 конфет (только нечетные числа) и еще одну, а
Карлсон съел количества конфет от 2 до 20.
Значит всего изначально было $1 + 2 + \ldots + 19 + 20 + 1 = 211$ конфет.
\endproblem
% $problem-source: А.\,Шаповалов
