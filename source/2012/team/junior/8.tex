\problem
\problemscore{9}
В ряд стоит 1111 блюдец, на них лежат
1, 2, 3, \ldots, 555, 556, 555, 554, \ldots, 2, 1 орехов.
За ход разрешается переложить любое число (не меньше одного) орехов с любого
блюдца на соседнее слева или съесть любое число орехов из самого левого блюдца.
Петя и Вася делают ходы по очереди, начинает Петя.
Проигрывает тот, у кого нет хода.
Кто из них может выиграть, как бы не играл соперник?

\solution
Отметим блюдца с нечетным числом орехов (они идут через одно).
Разобьем их на пары с одинаковым числом орехов.
Пусть Вася все ходы делает только из отмеченных блюдец.
Каждым ходом Петя нарушает равенство для какой-то пары отмеченных блюдец.
Пусть Вася равенство восстанавливает:
если Петя добавил орехи в блюдце пары, Вася их убирает;
а если Вася убрал какое-то число орехов с одного блюдца пары, Вася убирает
столько же из другого.
В результате Вася всегда может сделать ход, поэтому он не проиграет,
а так как игра конечна, то выиграет.
\emph{Ответ:} выигрывает Вася.

\endproblem
