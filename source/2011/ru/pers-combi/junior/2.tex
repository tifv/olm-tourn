\problem
По кругу стоит $2011$ представителей трех племен: рыцари, лжецы и конформисты.
Рыцарь всегда говорит правду, лжец всегда лжет, а конформист может лгать,
только если стоит рядом с лжецом (а может сказать правду).
Каждый заявил: <<Мои соседи из разных племен>>.
Какое минимальное количество лжецов может быть среди них?
\solution
\emph{Ответ:} 1.
\par
\emph{Оценка.}
Покажем, что есть хоть один лжец.
Пусть не так.
Тогда все люди, стоящие по кругу говорят правду.
Значит, рядом с каждым рыцарем есть еще один рыцарь и один конформист.
% spell РРККРРКК\ldots
Получаем такую последовательность: РРККРРКК\ldots
В этом случае, количество людей делилось бы на два, что не так.
Значит есть хотя бы один лжец.
% spell ЛРРККРР\ldotsККРР
\par
\emph{Пример:} ЛРРККРР\ldotsККРР.
\endproblem
