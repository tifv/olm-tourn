\problem
За~круглым столом сидели несколько лжецов и~рыцарей.
Первый сказал: <<Не~считая меня, здесь лжецов на~одного больше, чем рыцарей.>>.
Второй сказал: <<Не~считая меня, здесь лжецов на~два больше, чем рыцарей.>>,
и~так далее вплоть до~последнего.
Сколько человек могло сидеть за~столом?
\solution
\emph{Ответ:} $2$ или $3$.
\par
Пусть всего $n$~человек.
Все лжецами быть не~могут, иначе $(n - 1)$-й сказал правду.
Все заявления противоречат друг другу, поэтому рыцарь ровно~$1$.
Он~сидит на~$(n - 1)$-м  месте.
Для лжеца верным будет утверждение: <<Не~считая меня, здесь лжецов на~$(n - 3)$ больше, чем рыцарей.>>.
Чтобы оно не~прозвучало, должно быть $n - 3 < 1$, то~есть $n < 4$.
\endproblem
% $problem-source: А.\,Шаповалов
