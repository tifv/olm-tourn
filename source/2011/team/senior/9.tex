\problemmark{10}
Пусть $B_k$~--- количество способов разбить $k$-элементное множество на
непустые подмножества
(например, $B_3 = 5$, потому что множество $\{1, 2, 3\}$ имеет 5 разбиений:
\textsf{(123)}, \textsf{(1,2,3)}, \textsf{(12,3)}, \textsf{(13,2)},
\textsf{(1,23)}).
Докажите, что если $p$~--- простое число, а $k$~--- натуральное, то
$B_{k + p^{p} - 1} - B_k$ делится на $p$.

\solution
Все равенства суть сравнения по модулю $p$.
\par
\emph{Лемма 1.}
$B_{k+p} = B_{k+1} + B_k$ при всех $k$. 
\emph{Доказательство.}
Рассмотрим множество $M$ мощности $k + p$ и выделим в нём множество $S$ из $p$
элементов, которые расположим в вершинах правильного $p$-угольника.
Каждое множество из разбиения $M$ представим в виде $A_i \cap B_i$, где $A_i$
содержит невыделенные элементы, а $B_i$ выделенные
($A_i$ или $B_i$ может быть пустым).
Для каждого поворота $R$, переводящего $S$ в себя, рассмотрим новое разбиение $M$ на множества $A_i \cap R(B_i)$.
Таким образом, все разбиения $M$ разобьются на наборы по $p$ разбиений, кроме:
\\\emph{(1)}
тех разбиений, для которых одно из $B_i$ совпадает с $S$, а остальные $B_i$
пусты;
\\\emph{(2)}
тех разбиений, для которых каждый элемент $S$ есть отдельное множество в
разбиении.
\\
Разбиений второго типа, очевидно, $B_k$, а первого~--- $B_{k+1}$
(все $S$ можно рассматривать как один новый элемент).
\par
\emph{Лемма 2.}
Если $f(k + M) = f(k + 1) + f(k)$ при всех $k$ для некоторой функции $f$,
действующей из натуральных чисел в остатки по модулю $p$, то
$f(k + M p) = f(k) + f(k + p)$.
\emph{Доказательство.}
Имеем
\(
    f(k + 2 M)
=
     f(k + M) + f(k + M + 1)
=
     f(k) + 2 f(k + 1) + f(k + 2)
\),
\(
    f(k + 3 M)
=
    f(k + 2 M) + f(k + 2 M + 1)
=
    f(k) + 3 f(k + 1) + 3 f(k + 2) + f(k + 3)
\),
и так далее будем получать коэффициенты из треугольника Паскаля, для
\(
    f(k + p M)
=
    f(k) + p f(k + 1) + \frac{p (p - 1)} 2 f(k + 2)
    + \ldots +
    p f(k + p - 1) + f(k + p)
\).
Поскольку все биномиальные коэффициенты $C_p^i$ для $i = 1, 2, \ldots, p - 1$
кратны $p$, получаем требуемое.
\par
\emph{Лемма 3.}
$B_{k + p^s} = B_{k + 1} + s B_{k}$ при $1 \leq s \leq p$.
\emph{Доказательство.}
Индукция по $s$.
База $s = 1$ проверена в лемме 1.
Переход от $s < p$ к $s + 1$.
Положим $f(k) = B_k / s^k$.
По малой теореме Ферма имеем $s^{p} = s$, $s^{p^i} = s$ при натуральных $i$,
так что сокращая данное в условии уравнение на $s^{k + 1}$ получим
$f(k + p^s) = f(k + 1) + f(k)$.
По лемме 2 получаем $f(k + p^{s + 1}) = f(k) + f(k + p)$, то есть
\[
    B_{k + p^{s + 1}}
=
    s B_k + B_{k + p}
=
    (s + 1) B_k + B_{k + 1},
\]
что и требовалось.
Осталось положить $s = p$ в лемме 3.

