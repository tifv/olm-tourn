\problem
По кругу стоит 101 блюдце, на каждом по конфете.
Сначала Малыш выбирает натуральное $m < 101$ и сообщает его Карлсону,
затем Карлсон~--- натуральное $k < 101$.
Малыш берет конфету с любого блюдца.
Отсчитав от этого блюдца $k$-е блюдце по часовой стрелке,
Карлсон берет с него конфету.
Отсчитав уже от этого блюдца $m$-е блюдце по часовой стрелке,
Малыш берет с него конфету (если она там еще есть).
Отсчитав от блюдца Малыша $k$-е блюдце по часовой стрелке,
Карлсон берет с него конфету (если она там еще есть), и т.~д.
Какое наибольшее число конфет может гарантировать себе Карлсон?
\solution
\emph{Ответ:} 99.
\par
\emph{Оценка.}
Легко видеть, что если на втором ходу Малышу не досталось конфеты,
то $m + k = 101$, и игра зациклится на двух блюдцах.
Следовательно, если Карлсон получит больше одной конфеты, то Малышу
достанется не меньше двух.
\par
\emph{Пример.}
Карлсону достаточно взять $k = 101 - 2 m$ (при $m < 51$) или $k = 202 - 2 m$
(при $m \geq 51$).
Тогда Карлсон фактически отсчитывает $2 m$ против часовой стрелки.
Обозначив первое блюдце Малыша через 0, далее указываем номер при отсчете от
него против часовой стрелки.
Тогда Малышу достаются блюдца $0$, $m$, $2 m$, $3 m$, \ldots,
а Карлсону~--- $2 m$, $3 m$, $4 m$, \ldots.
Как видим, начиная с третьего хода Малышу достается блюдце, опустошенное
Карлсоном 2 хода назад.
Поэтому больше двух конфет Малыш не получит.
С другой стороны, так как число 101 простое, то числа $m$ и 101 взаимно просты,
поэтому среди чисел $2 m$, $3 m$, $4 m$ встретятся всевозможные остатки по
модулю $m$.
Значит, конфеты будут взяты со всех блюдец, поэтому Карлсон захватит
$(101 - 2)$ конфеты.
\endproblem
