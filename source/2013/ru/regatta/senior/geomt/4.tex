\problem
Неравнобедренный треугольник $ABC$ вписан в окружность $\Gamma$.
Биссектриса угла $A$ пересекает отрезок $BC$ в точке $E$.
Пусть $M$~--- середина дуги $BAC$.
Прямая $ME$ повторно пересекает $\Gamma$ в точке $D$.
Докажите, что центр описанной окружности треугольника $AED$ совпадает с точкой
пересечения касательной к $\Gamma$ в точке $D$ и прямой $BC$.

\solution
[{\begin{figure}\centering
    \jeolmfigure[width=0.4\textwidth]{solution}
\caption{к задаче \ref{solution:2013/ru/regatta/senior/geomt/4}.}
\label{fig:solution:2013/ru/regatta/senior/geomt/4}
\end{figure}}]%
\label{solution:2013/ru/regatta/senior/geomt/4}%
Рис.~\ref{fig:solution:2013/ru/regatta/senior/geomt/4}.
Предположим, что $AB < AC$, тогда точка $D$ располагается на дуге $BC$ ближе к
точке $B$, чем к $C$.
Тогда $O$~--- точка пересечения касательной в точке $D$ и прямой $BC$, лежит в
той же полуплоскости относительно $DE$, что и точка $A$.
Тогда
\(
    \angle ODE
=
    \frac{1}{2} \wideparen{DBM}
=
    \frac{1}{2}(\wideparen{BD} + \wideparen{BM})
=
    \frac{1}{2} (\wideparen{BD} + \wideparen{MC})
=
    \angle OED
\).
Следовательно, достаточно проверить, что $\angle DOE = 2\angle DAE$, но это
равенство очевидно, поскольку
$\angle DOE = \frac{1}{2} (\wideparen{CD}-\wideparen{BD}) = \wideparen{DX}$,
где $X$~--- точка пересечения $AE$ с окружностью.
\endproblem
% $problem-source: Italy TST 2002
