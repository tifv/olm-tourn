\problem
\problemscore{5}
Дан равнобедренный треугольник $ABC$ ($AB = BC$).
На стороне $AB$ выбирается точка $K$, а на стороне $BC$~--- точка $L$ так, что
$AK + CL = \frac{1}{2} AB$.
Найдите геометрическое место середин отрезков $KL$.

\solution
Отметим на $AB$ точку $M$, а на $BC$~--- точку $N$ так, чтобы
$AM = CN = \frac{1}{4} AB$.
Отметим также точки $M_1$ и $N_1$~--- середины $AB$ и $BC$.
На отрезке $MN$ отметим точки $M_2$ и $N_2$ пересечения отрезков
$AN_1$ и $CM_1$ с отрезком $MN$.
Нетрудно убедиться, что $M_2M = N_2N = AB / 4$.
Покажем, что искомое ГМТ совпадает с отрезком $M_2N_2$.
Ясно, что $L$ и $K$ лежат по разные стороны от прямой $MN$ и $KM = LN$.
Без ограничения общности считаем, что $K$ лежит на отрезке $BM$.
Проведем через $K$ прямую параллельно $BN$ до пересечения с $MN$ в некоторой
точке $D$.
Стороны треугольников $MKD$ и $ABC$ параллельны, поэтому $MKD$~---
равнобедренный, $MK = KD$.
Отрезки $KD$ и $NL$ равны и параллельны, значит, $KNLD$~--- параллелограмм,
и середина $E$ отрезка $KL$ лежит на $DN$;
кроме того, $NE > N_2N$ и $ME > M_2M$, так как если $K'$ точка симметричная $L$
относительно $N$ и $L'$ точка симметричная $K$ относительно $M$, то
$NE = KK' / 2 > M_1 N_1 / 2 = NN_2$ и $ME = LL' / 2 > M_1N_1 / 2 = MM_2$.
Значит $E \in M_2N_2$.
Обратно, пусть $E$~--- точка на $M_2N_2$, и, скажем, $EN_2 < EM_2$.
Отложив на $EM$ отрезок $ED = EN$, проведя через $D$ прямую параллельно $BN$ до
пересечения с $BM$ в точке $K$ и отложив на луче $NC$ отрезок $NL = DK$,
получим нужные нам отрезок $KL$ с серединой $E$
(несложно проверить, что $L$ будет лежать на $BC$). 

\endproblem
