\problem
% spell "\text{м}" -> "\text{метров}"
Секретный объект представляет собой в плане квадрат $40 \times 40\,\text{м}$,
разбитый коридорами на квадратики $5 \times 5\,\text{м}$.
В каждой вершине такого квадратика~--- выключатель.
Щелчок выключателя действует сразу на все выходящие из этой вершины
пятиметровые коридоры, меняя их освещенности на противоположные.
Сторож находится в углу полностью неосвещенного объекта.
Он может ходить только по освещенным коридорам и щелкать выключателями любое
число раз.
Может ли он добиться того, чтобы от любого выключателя к любому другому он мог
пройти, не щелкая выключателями?
\solution
\emph{Ответ:} нет.
Ясно, что сторож не может осветить все коридоры~--- последним щелчком он
выключит свет в коридоре, по которому пришел в узел.
Раскрасим узлы в шахматном порядке в черный и белый цвета.
Чтобы коридор был освещен, необходимо и достаточно, чтобы на одном его конце
выключателем щелкнули четное число раз, а на другом~--- нечетное.
Допустим, нам удалось осветить объект как требуется в условии.
Двигаясь от узла к узлу по освещенным коридорам, мы будем чередовать цвет и
четность, поэтому все узлы одинакового цвета будут иметь одинаковую четность, а
разного~--- разную.
Но тогда концы каждого коридора~--- разной четности, то есть будут освещены все
коридоры~--- противоречие.
\endproblem
