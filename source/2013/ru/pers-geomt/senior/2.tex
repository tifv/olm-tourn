\ifsolution
\begin{figure}\centering
    \jeolmfigure[width=0.5\textwidth]{solution}
    \caption{к решению задачи \ref{2013/ru/pers-geomt/senior/2:solution}.}
    \label{2013/ru/pers-geomt/senior/2:solution:fig}
\end{figure}%
\fi % \ifsolution

\problem
Даны концентрические окружности $\omega$ и $\gamma$, причем $\gamma$ лежит
внутри $\omega$.
На окружности $\omega$ выбрана произвольная точка $O$, из которой проведены
касательные $OA$ и $OB$ к окружности $\gamma$.
Окружность с центром в точке $O$ и радиусом $OA$ пересекает окружность $\omega$
в точках $C$ и $D$.
Докажите, что прямая $CD$ содержит среднюю линию треугольника $OAB$.
\solution
\label{2013/ru/pers-geomt/senior/2:solution}%
Рис.~\ref{2013/ru/pers-geomt/senior/2:solution:fig}.
Окружность с центром $O$ и радиусом $OA$ обозначим за $\Omega$.
Достаточно проверить, что середина $OA$ имеет одинаковые степени точки
относительно окружностей $\omega$ и $\Omega$.
Тогда средняя линия треугольника $AOB$ окажется радикальной осью этих
окружностей и будет проходить через точки их пересечения.
Итак, пусть точка $M$~--- середина отрезка $AO$, и прямая $AO$ повторно
пересекает окружность $\omega$ в~точке $F$.
Из симметрии $AF = AO = 2 MO$.
Значит, степень точки $M$ относительно окружности $\omega$ равна
$- OM \cdot FM = - 3 OM^2$.
Относительно окружности $\Omega$ степень точки $M$ вычислим как разность
квадрата расстояния до центра и квадрата радиуса:
$MO^2 - AO^2 = MO^2 - 4 MO^2 = - 3 AO^2$.
Что и требовалось.
\endproblem
% $problem-source: Ф.\,Ивлев
