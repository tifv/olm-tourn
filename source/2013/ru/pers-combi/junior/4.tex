\problem
33 компьютера попарно соединены проводами.
Каждый провод покрашен в один из 32 цветов.
Назовем компьютер \emph{радужным}, если из него выходят провода всех цветов.
Какое наибольшее число компьютеров могут быть радужными?
\solution
\emph{Ответ:} 32.
\par
Заметим, что все 33 компьютера радужными быть не могут.
Действительно, если так получилось, то рассмотрим провода первого цвета.
Из каждого компьютера в этом случае выходит ровно один такой провод, и это
означает, что компьютеры разбились на пары.
Но их 33, поэтому это невозможно.
Теперь покажем, что 32 компьютера могут быть радужными.
Для это сперва выберем один компьютер и соединим его с остальными 32 проводами
одного цвета.
Теперь задача свелась к тому, что 32 компьютера соединить проводами 31 цвета
так, чтобы все 32 компьютера были радужными.
Это просто означает, что нужно разбить все ребра полного графа на 32 вершинах
на 31 группу по 16 ребер в каждой так, чтобы в каждой группе ребра попарно не
имели общих концов.
Для это мы расставим все 32 компьютера по кругу.
Теперь рассмотрим циклы с шагом $k$.
Ребра каждого такого цикла раскрасим в цвет $2 k - 1$ и $2 k$ так, чтобы они
чередовались (это возможно, так как любой такой цикл имеет четную длину).
Наконец, для $k = 16$ таких циклов нет, такие ребра мы все покрасим в 31-ый
цвет.
Полученная раскраска будет требуемой.
\endproblem
% $problem-source: Iran NO 2012
