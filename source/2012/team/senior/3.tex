\problem\problemscore{5}
По кругу лежит 17 одинаковых на вид монет, из которых две лежащие рядом~---
фальшивые.
Все настоящие монеты весят одинаково, и обе фальшивые монеты весят одинаково и
при этом легче настоящих на 1 грамм.
Имеются \emph{хлипкие} весы~--- это чашечные весы, которые ломаются, если
разность весов на чашах больше 1 грамма, однако, показывают при этом, какая
чаша перевесила.
Как за два взвешивания на хлипких весах без гирь найти обе фальшивые монеты?

\solution
Пронумеруем монеты от 1 до 17.
Положим на одну чашу весов монеты 1, 2, 4, 6, 14, 16,
а на вторую чашу весов монеты с номерами 3, 5, 7, 8, 10, 12.
Если весы сломались, то мы нашли фальшивую пару.
Если весы в равновесии, то фальшивые монеты есть на обеих чашах по одной
(так как, очевидно, наши взвешивания задели любую пару последовательных монет),
и значит, пара фальшивых монет~--- это любая пара от $(2, 3)$ до $(6, 7)$.
Если первая чаша легче второй, то на ней ровно одна фальшивая монета,
и значит пара фальшивых монет~--- это пара от $(13, 14)$ до $(17, 1)$.
Наконец, если легче вторая чаша, то пара фальшивых монет~--- это пара
от $(8, 9)$ до $(12, 13)$.
Таким образом, в любом случае, если мы еще не нашли фальшивую пару, то мы имеем
задачу поиска фальшивой пары среди шести последовательных монет.
Покажем, что это возможно.
Пронумеруем заново монеты от 1 до 6.
Взвесим на одной чаше монеты 1 и 2, а на второй~--- монеты 5 и 6.
Если весы сломались, то мы нашли фальшивую пару.
Если равенство, то фальшивые монеты~--- это $(3, 4)$.
Если легче первая чаша, то пара фальшивых монет~--- это $(2, 3)$.
Наконец, если легче вторая чаша, то фальшивая пара~--- это пара $(4, 5)$.

\endproblem
