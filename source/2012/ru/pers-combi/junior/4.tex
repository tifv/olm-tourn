\problem
Паули и Бор играют в следующую игру.
Имеется куча из $99!$ молекул.
За один ход из кучи разрешается взять не более, чем $1\%$ от оставшихся
молекул.
Проигрывает тот, кто не может сделать ход.
Ходят поочередно, начинает Паули.
Кто из них может выиграть, как бы ни играл соперник?
\solution
Рассмотрим игру по тем же правилам для кучи с $99! - 1$ молекулой
(назовем эту игру \emph{меньшей}).
Поскольку игра закончится за конечное число ходов и ничьих нет, то у одного из
игроков есть выигрышная стратегия.
Возможны два случая.
\\
\emph{(1)}
В меньшей игре выигрывает второй.
Тогда в исходной игре Паули первым ходом берет одну молекулу.
Тем самым он сведет ситуацию к меньшей игре, где ходит вторым, и поэтому
выиграет.
\\
\emph{(2)}
В меньшей игре выигрывает первый, взяв первым ходом $x$ камней.
Тогда по условию $x \leq \frac{99! - 1}{100}$.
Так как $99!$ кратно $100$, то $x + 1 \leq \frac{99!}{100}$.
Это значит, что в исходной игре Паули имеет право первым ходом взять $x + 1$
камень.
Тогда он попадет в ситуацию после выигрышного хода первого игрока в меньшей
игре.
Действуя далее как этот игрок, он выиграет.
\\
\emph{Ответ:} Паули.
\endproblem
