\ifsolution
\begin{figure}\centering
    \jeolmfigure[width=0.5\textwidth]{solution}
    \caption{к задаче \ref{solution:2013/ru/pers-geomt/junior/1}.}
    \label{fig:solution:2013/ru/pers-geomt/junior/1}
\end{figure}
\fi % \ifsolution

\problem
На сторонах $AC$ и $BC$ треугольника $ABC$ лежат точки $D$ и $E$
соответственно, такие, что $\angle ABD = \angle CBD = \angle CAE$.
Кроме того, $\angle ACB = \angle BAE$.
Обозначим через $F$ точку пересечения $BD$ и $AE$.
Докажите, что $AF = DE$.
\solution
\label{solution:2013/ru/pers-geomt/junior/1}
Рис.~\ref{fig:solution:2013/ru/pers-geomt/junior/1}.
Заметим, что
\(
    \angle BAD = \angle BAE + \angle EAD
=
    \angle BCD + \angle CBD = \angle BDA
\)
(мы воспользовались тем, что внешний угол треугольника равен сумме двух, не
смежных с ним).
Значит $AB = BD$.
Аналогично
\(
    \angle BFE = \angle FBA + \angle FAB
=
    \angle EAD + \angle ECD = \angle BEF
\), откуда $BF=BE$.
Значит треугольники $ABF$ и $DBE$ равны (по двум сторонам и углу между ними).
Отсюда следует, что $AF = DE$.
\endproblem
% $problem-source: Ф.\,Ивлев, Ф.\,Бахарев
