\problem
По кругу стоят несколько гирь, не все они одного веса, и веса не обязательно
целые.
В каждой тройке подряд стоящих гирь вес одной из них равен среднему
арифметическому весов гирь в тройке.
Докажите, что число гирь делится на 3.

\solution
Посчитаем все разности, вычитая из веса гири вес правого соседа.
Пусть $d$~--- наименьшая по модулю разность весов.
Если $d < 0$, сменим все знаки разностей, отразив круг симметрично.
Поделив все веса на $d$, условия не нарушим, но теперь минимальный модуль
разности равен 1.
Заметим, что если в тройке средний вес~--- у средней гири, то в этой тройке обе
разности равны, а если у~крайней, то отношение разностей равно $-2$.
Поэтому все разности принадлежат списку: $1$, $-2$, $4$, $-8$, \ldots
Пройдя по кругу, мы вернемся к тому же весу, поэтому сумма всех разностей
равна~0.
С другой стороны, по модулю 3 все разности равны 1.
Заменив все разности на 1, мы сумму по модулю 3 не изменим.
Значит, эта сумма (то есть число гирь) делится на 3.
\endproblem
% $problem-source: А.\,Шаповалов
