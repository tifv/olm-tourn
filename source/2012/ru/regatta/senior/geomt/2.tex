\problem
Внутри квадрата $ABCD$ выбрана точка $X$ и через неё проведены отрезки
$PQ$ и $EF$, параллельные сторонам квадрата $AD$ и $AB$ соответственно,
с концами, лежащими на сторонах квадрата ($P$ на $AB$, $F$ на $AD$).
Оказалось, что $S_{ECQX} = 2 S_{PXFA}$.
Чему равен $\angle EAQ$?
\solution
[{\begin{figure}
\centering
    \jeolmfigure[width=0.5\textwidth]{solution}
\caption{к задаче \ref{solution:2012/regatta/senior/geomt/2}}
\label{fig:solution:2012/regatta/senior/geomt/2}
\end{figure}}]
\label{solution:2012/regatta/senior/geomt/2}%
\emph{Ответ:} $45^\circ$.
Рис.~\ref{fig:solution:2012/regatta/senior/geomt/2}.
Обозначим сторону квадрата через $a$, а длины отрезков $AF$ и $AP$ через
$x$ и $y$ соответственно.
Тогда длины отрезков $EC$ и $CQ$ равны $a - x$ и $a - y$ соответственно.
Условие на площади можно переформулировать так:
$EC \cdot CQ = AF \cdot AP$ или $(a - x) (a - y) = 2 x y$.
Запишем тангенсы углов $BAE$ и $QAD$:
\[
    \tg \angle BAE = \dfrac{BE}{AB} = \dfrac{x}{a}
,\qquad
    \tg \angle QAD = \dfrac{DQ}{AD} = \dfrac{y}{a}
.\]
Теперь найдем тангенс суммы этих углов:
\[
    \tg (\angle EAB + \angle QAD)
=
    \dfrac{
        \tg \angle EAB + \tg \angle QAD
    }{
        1 - \tg \angle EAB \cdot \tg \angle QAD
    }
=
    \dfrac{x / a + y / a}{1 - x / a \cdot y / a}
=
    \dfrac{a(x + y)}{a^2 - xy}
.\]
Но из уравнения на площади имеем
\[
    2 x y = (a - x) (a - y) = a^2 - a(x + y) + xy
\quad\Leftrightarrow\quad
    a (x + y) = a^2 - x y
.\]
То есть $\tg (\angle EAB + \angle QAD) = 1$.
А значит, так как сумма этих углов очевидно меньше $90^\circ$,
она равна $45^\circ$.
Следовательно,
\[
    \angle EAQ = \angle BAD - \angle BAE - \angle QAD
=
    90^\circ - (\angle BAE + \angle QAD) = 45^\circ
.\]
\endproblem
