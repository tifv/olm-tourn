\ifsolution
\begin{figure}\centering
    \jeolmfigure[width=0.5\textwidth]{solution}
    \caption{к задаче \ref{solution:2012/ru/team/junior/3}}
    \label{fig:solution:2012/ru/team/junior/3}
\end{figure}%
\fi % \ifsolution

\problem\problemscore{3}
Бумажный треугольник со сторонами $a$, $b$, $c$ перегнули по прямой так,
что вершина, противолежащая стороне длины~$c$, попала на эту сторону.
Известно, что в получившемся четырехугольнике равны два угла,
примыкающие к линии сгиба.
Найдите длины отрезков, на которые делит сторону $c$ попавшая туда вершина.
\solution
\label{solution:2012/ru/team/junior/3}
\emph{Ответ:} $a \cdot c / (a + b)$, $b \cdot c / (a + b)$.
Рис.~\ref{fig:solution:2012/ru/team/junior/3}.
Пусть $KL$~--- линия сгиба, $C$~--- упомянутая вершина, $C'$~--- ее
положение после перегиба.
Отрезок $CC'$ составлен из высот равных треугольников $KCL$ и $KC'L$.
Углы $CKL$ и $CLK$ равны как смежные к равным углам четырехугольника, значит,
треугольник $KCL$~--- равнобедренный.
Поэтому $CC'$~--- биссектриса угла $C$, и делит сторону $c$ на отрезки,
пропорциональные двум другим сторонам.
\endproblem
