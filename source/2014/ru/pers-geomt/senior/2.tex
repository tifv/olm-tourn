\ifsolution
\begin{figure}\centering
    \jeolmfigure[width=0.6\textwidth]{solution}
    \caption{к задаче \ref{2014/ru/pers-geomt/senior/2:solution}}
    \label{2014/ru/pers-geomt/senior/2:solution:fig}
\end{figure}%
\fi % \ifsolution

\problem
В~выпуклом четырехугольнике $ABCD$ углы $\angle B$ и~$\angle D$ равны.
Оказалось, что точки пересечения биссектрис соседних углов $ABCD$ образуют
выпуклый четырехугольник $EFGH$
($E$ лежит на~биссектрисах $\angle A$ и~$\angle B$,
$F$~--- $\angle B$ и~$\angle C$, и~т.~д.).
Пусть $K$~--- точка пересечения диагоналей $EFGH$.
Лучи $AB$ и~$DC$ пересекаются в~точке~$P$, а~лучи $BC$ и~$AD$~--- в~точке~$Q$.
Докажите, что $P$ лежит на~описанной окружности треугольника $BKQ$.
\solution
\label{2014/ru/pers-geomt/senior/2:solution}%
Рис.~\ref{2014/ru/pers-geomt/senior/2:solution:fig}.
Заметим, что
$\angle PBQ = 180^\circ - \angle ABC = 180^\circ - \angle ADC = \angle PDQ$.
Тогда четырехугольник $PBDQ$ вписанный.
Обозначим его описанную окуржность через $s$.
Так как $E$~--- точка пересечения биссектрис углов $A$ и~$B$, то~$E$ лежит
на~биссектрисе угла $BQD$.
Так как $G$~--- точка пересечения биссектрис углов $C$ и~$D$, то~$G$ лежит
на~этой биссектрисе $BQD$.
Тогда и~$K$ лежит на~этой биссектрисе.
Аналогично, $K$ лежит на~биссектрисе угла $BPD$.
Осталось заметить, что эти биссектрисы пересекаются в~середине дуги~$BD$
окружности~$s$.
Это означает, что точка~$K$ является серединой этой дуги, и, следовательно,
лежит на~окружности~$s$.
\endproblem
% $problem-source: Middle European 2012
