\problemmark{5}
Существует ли возрастающая арифметическая прогрессия, состоящая из 2012
натуральных чисел, в разложении каждого из которых на простые множители четное
число различных простых чисел?

\solution
\emph{Ответ:} да, существует.
Выберем 2012 последовательных чисел $n + 1$, $n + 2$, \ldots, $n + 2012$ так,
чтобы у каждого из них был простой делитель, не входящий в разложения остальных
(например, построим такие числа по китайской теореме об остатках).
Пусть эти простые числа~--- $p_1$, $p_2$, \ldots, $p_{2012}$.
Кроме этого, рассмотрим еще 2012 простых чисел
$q_1$, $q_2$, $\ldots$, $q_{2012}$,
которых нет в разложениях выбранных 2012 последовательных чисел.
Теперь будем строить число $A$, на которое потом умножим все выбранные нами
числа, чтобы получилась требуемая прогрессия.
Заметим, что добавление в число $A$ пары множителей $p_k q_k$ меняет четность
количества простых множителей в точности только в числе $A (n + k)$.
Строя из таких множителей число $A$ мы добьемся того, чтобы у всех чисел вида
$A (n + k)$ было четное число простых множителей в разложении.

