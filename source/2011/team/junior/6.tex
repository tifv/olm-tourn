\problemmark{10}
На доске написаны числа от $1$ до $1000$.
Петя и Вася ходят по очереди, начинает Петя.
За один ход нужно заменить любые два числа на их сумму.
Если одно из двух последних оставшихся на доске чисел делится на второе, то
выигрывает Вася, иначе~--- Петя.
Кто из них может выиграть, как бы ни играл соперник?

\solution
\emph{Ответ:} Петя.
Назовем набор чисел хорошим, если можно выделить одно число с остатком 4 при
делении на 8 (назовем его особым), а все остальные не кратные 8 разбить на пары
с суммами, кратными 8.
Изначальный набор хорош: особое число~--- 500, а все остальные не кратные 8
разбиваются на 437 пар с суммой 1000.
Первая цель Пети: оставлять после себя хороший набор с меньшим числом пар, чем
после прошлого хода. уменьшая каждый раз число пар.
Нехороший набор Вася может сделать двумя способами.
\\\emph{(1)}
Вася складывает особое число с числом из пары.
Тогда Петя прибавляет к результату второе число пары, восстанавливая особое
число.
\\\emph{(2)}
Вася складывает два числа из разных пар.
Тогда Петя складывает два других числа из этих пар, и объединяет сумму в пару с
суммой Васи.
\\
Как видим, число пар в обоих случаях стало на 1 меньше.
Тем самым, не позднее 437-го хода Пети останется хороший набор без пар, то есть
все числа, кроме особого, станут кратны 8.
С этого момента Петя каждым ходом складывает особое число с наибольшим из
остальных.
После хода Пети особое будет наибольшим из всех.
Ответным ходом Вася может создать не более одного числа больше особого.
Поэтому особое всегда будет одним из двух наибольших.
Перед последним ходом Пети на доске 4 числа, причем особое~--- одно из двух
самых больших.
Петя складывает эти два числа.
Ответным ходом Вася либо сложит два наименьших числа, либо увеличит особое.
В обоих случаях особое число больше другого числа, кратного 8, поэтому не
делится на него.
\\
\emph{Замечание.}
Алгоритм работает для любого четного числа кроме степеней двойки.
%\emph{По мотивам Н. Чернятьева, А.\,Шаповалов}

