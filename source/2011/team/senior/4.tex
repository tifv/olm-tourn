\problemmark{6}
Отрезок $AL$~--- биссектриса треугольника $ABC$, $I$ и $J$~--- центры
окружностей, вписанных в треугольники $ABL$ и $ACL$ соответственно.
Прямая $IJ$ пересекает стороны $AB$ и $AC$ в точках $C_1$ и $B_1$
соответственно.
Докажите, что прямые $B B_1$, $C C_1$ и $AL$ пересекаются в одной точке.

\solution
[{\begin{figure}
\centering
    \jeolmfigure[width=0.5\textwidth]{4-solution}
\caption{к задаче \ref{solution:2011/team/senior/4}.}
\label{fig:solution:2011/team/senior/4}
\end{figure}}]%
\label{solution:2011/team/senior/4}%
Пусть прямая $IJ$ пересекает сторону $BC$ в точке $P$, а прямые $AI$ и $AJ$
пересекают сторону $BC$ в точках $X$ и $Y$ соответственно
(см. рис. \ref{fig:solution:2011/team/senior/4}).
Напишем теорему Менелая для треугольника $AXY$:
\(
    \frac{AJ}{JY}
    \cdot
    \frac{YP}{PX}
    \cdot
    \frac{XI}{IA}
=
    1
\).
Так как $LI$ и $LJ$~--- биссектрисы углов $ALX$ и $ALY$ соответственно, то мы
имеем по свойству биссектрисы:
$\frac{AJ}{JY} = \frac{AL}{LY}$ и $\frac{XI}{IA} = \frac{XL}{AL}$.
Подставляя это в написанное выше равенство из теоремы Менелая, будем иметь
$\frac{YL}{LX} = \frac{YP}{XP}$.
Так как $AL$ также является биссектрисой треугольника $AXY$ мы знаем, что
$\frac{YL}{LX} = \frac{AY}{AX}$, то есть точка $P$ делит отрезок $XY$ внешним
образом в отношении, равном отношению сторон $AX$ и $AY$ треугольника $AXY$.
Но это означает, что точка $P$~--- основание внешней биссектрисы треугольника
$AXY$, а, значит, и треугольника $ABC$, то есть
$\frac{CP}{PB} = \frac{CA}{AC} = \frac{CL}{LB}$.
Применяя теперь теорему Менелая к треугольнику $ABC$ мы получим, что
\(
    \frac{A C_1}{C_1 B}
    \cdot
    \frac{BP}{PC}
    \cdot
    \frac{C B_1}{B_1 A}
=
    1
\).
Подставляя сюда отмеченное выше равенство, мы получаем условие теоремы Чевы для
точек $B_1$, $C_1$ и $L$, что и доказывает требуемое.

