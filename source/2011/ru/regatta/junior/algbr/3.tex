\problem
Пусть $P(x)$~--- квадратный трехчлен с целыми коэффициентами
(и ненулевым старшим коэффициентом) такой, что $P(1) = 2011$ и $P(2011) = 1$.
Может ли оказаться, что $P(m) = m$ при некотором целом $m$?
\solution
\emph{Ответ:} нет.
Пусть $P(x) = a x^2 + b x + c$.
Тогда
\begin{gather*}
    m - 2011 = P(m) - P(1) = a (m - 1) (m + 1) + b (m - 1)
\\
    m - 1 = P(m) - P(2011) = a (m - 2011) (m + 2011) + b (m - 2011)
\end{gather*}
Получаем, что $m - 1 \mid m - 2011$ и $m - 2011 \mid m - 1$.
Отсюда $|m - 1| = |m - 2011|$, то есть $m = 1006$.
Но это значит, что
$P(1) = 2011$, $P(1006) = 1006$ и $P(2011) = 1$.
Таким условиям удовлетворяет линейная функция $P(x) = 2012 - x$.
Парабола не может иметь трех общих точек с прямой, что означает, что это
невозможно.
\endproblem
