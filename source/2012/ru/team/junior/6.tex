\ifsolution
\begin{figure}\centering
    \jeolmfigure[height=0.5\textwidth]{solution}
    \caption{к задаче \ref{solution:2012/ru/team/junior/6}}
    \label{fig:solution:2012/ru/team/junior/6}
\end{figure}
\fi % \ifsolution

\problem\problemscore{5}
Дан равнобедренный треугольник $ABC$ ($AB = BC$).
На стороне $AB$ выбирается точка $K$, а на стороне $BC$~--- точка $L$ так, что
$AK + CL = \frac{1}{2} AB$.
Найдите геометрическое место середин отрезков $KL$.
\solution
\label{solution:2012/ru/team/junior/6}
Рис.~\ref{fig:solution:2012/ru/team/junior/6}.
Отметим на $AB$ точку $M_1$, а на $BC$~--- точку $N_1$ так, чтобы
$A M_1 = C N_1 = \frac{1}{4} AB$.
Отметим также точки $M$ и $N$~--- середины $AB$ и $BC$.
На отрезке $M_1 N_1$ отметим точки $M_2$ и $N_2$ пересечения отрезков
$A N$ и $C M$ с отрезком $MN$.
Покажем, что \emph{искомое ГМТ совпадает с отрезком $M_2 N_2$}.
Нетрудно убедиться, что $M_2 M_1 = N_2 N_1 = AC / 4$.
Ясно, что $L$ и $K$ лежат по разные стороны от прямой $M_1 N_1$ и
$K M_1 = L N_1$.
Без ограничения общности считаем, что $K$ лежит на отрезке $B M_1$.
Проведем через $K$ прямую параллельно $B N_1$ до пересечения с $M_1 N_1$ в
некоторой точке $D$.
Стороны треугольников $M_1 K D$ и $ABC$ параллельны, поэтому $M_1 K D$~---
равнобедренный, $M_1 K = KD$.
Отрезки $KD$ и $N_1 L$ равны и параллельны, значит, $K N_1 L D$~---
параллелограмм, и середина $E$ отрезка $KL$ лежит на $D N_1$;
кроме того, $N_1 E > N_2 N_1$ и $M_1 E > M_2 M$, так как если
$K'$~--- точка, симметричная $L$ относительно $N_1$, и
$L'$~--- точка, симметричная $K$ относительно $M_1$, то
$N_1 E = K K' / 2 > MN / 2 = N_1 N_2$ и $M_1 E = L L' / 2 > M N / 2 = M_1 M_2$.
Значит, $E \in M_2 N_2$.
Обратно, пусть $E$~--- точка на $M_2 N_2$, и, скажем, $E N_2 < E M_2$.
Отложив на $E M_1$ отрезок $ED = E N_1$, проведя через $D$ прямую параллельно
$B N_1$ до пересечения с $B M_1$ в точке $K$ и отложив на луче $N_1 C$ отрезок
$N_1 L = DK$, получим нужные нам отрезок $KL$ с серединой $E$
(несложно проверить, что $L$ будет лежать на $BC$).
\endproblem
