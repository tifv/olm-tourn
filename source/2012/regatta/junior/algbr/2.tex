\problem
%{\centering
%\(\begin{aligned}
%    & 1 \\[-5pt]
%    & 2 && 3 \\[-5pt]
%    & 4 && 5 && 6 \\[-5pt]
%    & 7 && 8 && 9 && 10
%\end{aligned}
%\)\\
%\ldots}
Маша пишет в таблицу числа:
в первой строчке она пишет число 1, во второй строчке она пишет числа 2 и 3
(2 оказывается под 1), в третьей строчке 4, 5, 6 (4 под 2) и т.\,д. до тех
пор, пока она не напишет 2012.
В каком столбце будет наибольшая сумма?

\solution
\emph{Ответ:} в 10 столбце.
Заметим, что
$1 + 2 + \ldots + 63 = 63 \cdot 64 / 2 = 2016$.
Значит, в таблице будет 63 строки, причем в последней строке не будет хватать
чисел 2014, 2015, 2016 до полной таблицы.
Будем для простоты полагать, что они присутствуют, как мы увидим, это не
повлияет на ответ.
Сравним сумму чисел в $k$-ом и $k + 1$-столбце.
В $k + 1$ столбце ровно $63 - k$ чисел, которые на 1 больше стоящих рядом с
ними чисел $k$-ого столбца.
Однако в $k$-ом столбце есть на одно число больше, и это число
$\dfrac{k (k + 1)}{2}$.
Значит, пока $63 - k > \dfrac{k (k + 1)}{2}$ сумма чисел в столбце будет
возрастать.
Первый раз это неравенство будет нарушаться при $k = 10$ и значит сумма в
десятом столбце будет самой большой.
Добавленные в начале числа, как мы видим, не повлияли на результат.

\endproblem
