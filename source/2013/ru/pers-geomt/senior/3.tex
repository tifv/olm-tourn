\problem
При каких $a > 1$ существует выпуклый многогранник, отношение площадей любых
двух граней которого больше $a$?

\solution
\emph{Ответ:} при всех $a \in (1, 2)$.
Мы будем пользоваться следующей леммой. 
Пусть грани выпуклого
многогранника имеют площади $S_1$, $S_2$, $\ldots$, $S_n$.
Тогда выполняется неравенство: $S_n \leq S_1 + S_2 + \ldots + S_{n-1}$.
(Аналог неравенства треугольника для многранника).
\\\emph{Доказательство леммы:}
спроектируем многогранник на плоскость грани с площадью $S_n$.
Тогда проекции остальных граней, очевидно, накроют грань $S_n$ (в силу выпуклости многогранника).
Отсюда следует, что $S_n$ не превосходит суммы площадей проекций остальных граней.
Но площадь проекции каждой грани не превосходит площади самой грани
(так как получается из нее домножением на косинус угла между гранями),
откуда и следует требуемое неравенство.

Теперь покажем, что при $a \geq 2$ такого многогранника существовать не может.
Пусть, площади граней этого многогранника
$S_1 < S_2 < S_3 < \ldots < S_{n-1} < S_n = 1$.
Тогда из условия следует, что $S_{k-1} \leq \frac{1}{a} S_k$, то есть $S_k \leq \frac{1}{a^{n-k}}$.
Напишем теперь неравенство из леммы:
\[
    1
=
    S_n \leq S_1 + S_2 + \ldots + S_{n-1}
\leq
    \frac{1}{a^{n-1}} + \frac{1}{a^{n-2}}
    + \ldots +
    \frac{1}{a}
\leq
    \frac{1}{2^{n-1}} + \frac{1}{2^{n-2}}
    + \ldots +
    \frac{1}{2}
<
    1
.\]
Полученное противоречие означает, что при $a \geq 2$ такого
многогранника существовать не может.

Теперь построим пример такого многогранника для $a_0<2$.
Выберем $a$ так, что $2>a>a_0$.
Сперва заметим, что найдется такое натуральное $n$,
что $\frac{1}{a}+\frac{1}{a^2}+\ldots+\frac{1}{a^{n-1}}>1$.
Действительно, сумма геометрической прогресси в левой
части это неравенства равна $\frac{a^{n-1}-1}{a^n-a^{n-1}}$.
Условие, что последняя дробь больше 1 равносильно тому,
что $2a^{n-1}>a^n+1$. Последнее неравенство перепишем
в виде $2>a+\frac{1}{a^{n-1}}$. Отсюда уже очевидно,
что при $a<2$ мы сможем подобрать подходящее $n$.

Зафиксируем найденное $n$ и построим многоугольник $M$ со сторонами $1$, $\frac{1}{a}$, $\ldots$, $\frac{1}{a^{n-1}}$
(условие, которого мы добились на предыдущем шаге, означает, что это возможно).
Пусть площадь этого многоугольника равна 1.
Выберем внутри многоугольника $M$ произвольную точку $O$.
Пусть расстояния от нее до сторон равны $h_0$, $h_1$, $\ldots$, $h_{n-1}$ соответственно.
Теперь выберем точку $S$ вне плоскости построенного многоугольника, но так, что бы ее проекция попадала в точку $O$.
По смыслу она будет расположена <<далеко>> от этой плоскости.
Пусть расстояния от нее до плоскости равно $h$.
Покажем, что подходящим выбором $h$ мы сможем добиться того, чтобы пирамида с вершиной в $S$ и основанием $M$ будет искомой.
Для этого посчитаем площадь треугольной грани:
$S_k=\frac{1}{2} \cdot \frac{\sqrt{h^2+h_k^2}}{a^k}$.
Это означает, что отношение площадей соседних граней $\frac{S_k}{S_{k+1}}$ имеет вид 
\[
    a \cdot \sqrt{\frac{h^2 + h_k^2}{h^2 + h_{k+1}^2}}.
\] 
Выбирая $h$ достаточно большим, мы сможем добиться, чтобы любое такое отноешение было больше $a_0$. Действительно, преобразовывая неравенство
\[
    a \cdot \sqrt{\frac{h^2 + h_k^2}{h^2 + h_{k+1}^2}}
>
    a_0
\]
мы получаем неравенство 
\[
    h
>
    \sqrt{\frac{a_0^2h_{k+1}^2-a^2h_k^2}{a^2-a_0^2}}
.\]
Подходящим выбором $h$ мы добьемся выполнения всех таких
неравенств для всех $k$.
остается лишь выбрать $h$ еще и таким, чтобы
площадь $\frac{1}{2} \cdot \frac{\sqrt{h^2+h_{n-1}^2}}{a^{n-1}}$ был бы больше, чем $a_0$ (то есть больше площади основания, умноженной на $a_0$).
Очевидно,
что этого также можно добиться.
\endproblem
% $problem-source: А.\,Шаповалов
