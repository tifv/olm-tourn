\problem
Дан неравнобедренный треугольник $ABC$.
Центр вневписанной окружности треугольника, касающейся стороны $BC$
треугольника, обозначим через $I_A$, а точку её касания с этой стороной~---
через $A_1$.
Аналогично определим точки $I_B$, $I_C$, $B_1$, $C_1$.
Докажите, что описанные окружности треугольников
$A I_A A_1$, $B I_B B_1$ и $C I_C C_1$ имеют две общие точки.

\solution
[{\begin{figure}
\centering
    \jeolmfigure[width=0.5\textwidth]{../4-solution}
\caption{к задаче \ref{solution:2012/pers-geomt/senior/4}}
\label{fig:solution:2012/pers-geomt/senior/4}
\end{figure}}]%
\label{solution:2012/pers-geomt/senior/4}%
См.\,рис.\,\ref{fig:solution:2012/pers-geomt/senior/4}.
Проведем описанную окружность треугольника $ABC$.
Отметим центр $I$ вписанной окружности $\triangle ABC$ и середину дуги $BC$
описанной окружности $\triangle ABC$, не содержащей точку $A$, точку $A'$.
Тогда по лемме о трезубце $A'$ есть середина отрезка $I I_A$.
Значит, $I A \cdot I I_A = 2 I A \cdot I A'$.
То есть степень точки $I$ относительно описанной окружности
$\triangle A A_1 I_A$ в два раза больше степени точки $I$ относительно описанной
окружности $\triangle ABC$.
Аналогичное можно сказать и про другие окружности из условия.
Значит, $I$ имеет одинаковую степень относительно всех трех окружностей.
Проведем внешнюю биссектрису угла $A$ до пересечения со стороной $BC$ в
точке $L_A$.
Аналогично определим точки $L_B$ и $L_C$.
Отметим, что точки $L_A$, $L_B$, $L_C$ лежат на одной прямой.
Так как $AI_A$~--- биссектриса угла $A$, то угол $L_AAI_A$ прямой.
Следовательно, точки $L_A$, $I_A$, $A_1$ и $A$ лежат на одной окружности с
диаметром $L_AI_A$.
Аналогичное верно и для других окружностей.
Значит, центры рассматриваемых окружностей суть середины отрезков
$I_A L_A$, $I_B L_B$, $I_C L_C$.
Но они лежат на прямой Гаусса четырехсторонника, образованного внешними
биссектрисами треугольника $ABC$ и прямой, проходящей через основания внешних
биссектрис.
А раз центры лежат на одной прямой, то либо окружности не имеют радикального
центра, либо у них общая радикальная ось.
Но, так как есть точка, имеющая одинаковую степень относительно всех трех
окружностей и лежащая внутри каждой из этих окружностей
(на хордах $A I_A$, $B I_B$, $C I_C$),
то окружности имеют общую радикальную ось и все её пересекают.
Следовательно, они имеют две общие точки.
\par
\emph{Другое решение.}
Воспользуемся доказательством того, что рассматриваемые окружности суть
окружности с диаметрами $L_A I_A$, $L_B I_B$, $L_C I_C$.
Тогда эти окружности имеют общие точки просто по теореме Плюккера.
Другими словами, все эти окружности имеют общую радикальную ось~---
прямую Обера соответственного четырехсторонника.
Значит, имеют две общие точки.

\endproblem
