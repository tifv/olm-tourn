\ifsolution
\begin{figure}\centering
    \jeolmfigure[width=0.5\textwidth]{solution}
    \caption{к задаче \ref{2014/ru/pers-geomt/junior/1:solution}}
    \label{2014/ru/pers-geomt/junior/1:solution:fig}
\end{figure}%
\fi % \ifsolution

\problem
В~прямоугольном треугольнике $ABC$ проведена биссектриса~$CL$.
Известно, что точка~$L$ равноудалена от вершины прямого угла $B$ и~середины
гипотенузы~$AC$.
Найдите угол $BAC$.
\solution
\label{2014/ru/pers-geomt/junior/1:solution}%
\emph{Ответ:} $30^\circ$.
\par
Пусть $M$~--- середина $AC$.
Опустим из~точки~$L$ перпендикуляр~$LN$ на~гипотенузу~$AC$.
Поскольку $L$ находится на~биссектрисе угла $BCA$, то $LB = LN$.
По~условию $LB = LM$.
Значит, точки $M$ и~$N$ совпадают.
Рис.~\ref{2014/ru/pers-geomt/junior/1:solution:fig}.
Тогда $LM$ является одновременно медианой и~высотой в~треугольнике $ALC$, т.~е.
этот треугольник равнобедренный.
Поэтому $\angle LAC = \angle ACL = \angle BCL$.
В~то~же время $\angle LAC + \angle ACL + \angle BCL = 90^{\circ}$.
Таким образом, $\angle BAC = \angle LAC = 30^{\circ}$.
\endproblem
% $problem-source: Ф.\,Нилов
