\problem\problemscore{6}
Назовем натуральное число \emph{интересным}, если оно представимо в виде
$a^2 + 2011 b^2$ для некоторых натуральных $a$ и $b$.
Докажите, что если для простого $p$ число $p^2$ является интересным, то хотя бы
одно из чисел $p$, $2 p$ тоже интересное.

\solution
Пусть для некоторого простого $p$ мы имеем $a^2 + 2011 b^2 = p^2$.
Перенося $a$ в правую часть, имеем $(p - a) (p + a) = 2011 b^2$.
Так как $p - a + p + a = 2 p$ и $a < p$, то мы получаем, что множители $p - a$
и $p + a$ либо взаимно просты, либо имеют общий делитель 2.
Если $p - a$ и $p + a$ взаимно просты, то мы имеем для некоторых натуральных
$c$ и $d$ равенство $p - a = 2011 c^2$ и $p + a = d^2$, или наоборот.
В любом случае, складывая эти равенства, получаем требуемое представление для
числа $2 p$.
Если же числа $p - a$ и $p + a$ имеет общий делитель 2, то мы имеем
$p - a = 2 \cdot 2011 c^2$ и $p + a = 2 d^2$ (или наоборот), откуда, складывая,
получаем требуемое представление для числа $p$.

\endproblem
