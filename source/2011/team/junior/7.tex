\problemmark{10}
Точки $P$ и $Q$ на стороне $AB$ выпуклого четырехугольника $ABCD$ таковы, что
$AP = QB$.
Точка $X$~--- отличная от $D$ точка пересечения описанных окружностей
треугольников $APD$ и $DQB$, а точка $Y$~--- отличная от $C$ точка пересечения
описанных окружностей треугольников $ACP$ и $QCB$.
Докажите, что точки $C$, $D$, $X$ и $Y$ лежат на одной окружности.

\solution
[{\begin{figure}
\centering
    \jeolmfigure[width=0.5\textwidth]{7-solution}
\caption{к задаче \ref{solution:2011/team/junior/7}.}
\label{fig:solution:2011/team/junior/7}
\end{figure}}]%
\label{solution:2011/team/junior/7}%
Заметим, что $DX$~--- радикальная ось окружностей, описанных вокруг
треугольников $ADP$ и $QDB$.
Пусть $K$~--- точка пересечения $DX$ и $AB$
(см.\,рис.\,\ref{fig:solution:2011/team/junior/7}).
Так как $K$ лежит на радикальной оси, значит $KP \cdot KA = KQ \cdot KB$.
Это означает (в силу условия), что $K$~--- середина отрезка $AB$.
Аналогично, получаем, что $CY$ тоже проходит через $K$.
Следовательно, $KX \cdot KD = KP \cdot KA = KQ \cdot KB = KY \cdot KC$, что и
означает, что требуемые точки лежат на одной окружности.

