\problem\problemscore{7}
Дана последовательность натуральных чисел $u_0$, $u_1$, $\ldots$, причем
$u_0 = 1$.
Оказалось, что существует натуральное $k$ такое, что равенство
$u_{n+1} u_{n-1} = k u_n$ верно при всех натуральных $n$.
Известно также, что $u_{2013} = 100$.
Найдите $k$.
\solution
\emph{Ответ:} $k = 10$.
\par
Обозначим $u_1$ через $a$.
Простыми вычислениями убеждаемся,что
$u_2 = k a$, $u_3 = k^2$, $u_4 = k^2 / a$, $u_5 = k / a$, $u_6 = 1$, $u_7 = a$,
то есть данная нам последовательность периодична с периодом 6.
Значит, $u_{2013} = u_3 = k^2$, откуда $k = 10$.
\endproblem
% $problem-source: British Olympiad 1995 
