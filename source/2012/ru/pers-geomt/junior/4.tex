\problem
Верно ли, что любой треугольник площади 3 можно покрыть выпуклым
многоугольником площади 5, имеющим ось симметрии?
\solution
\emph{Ответ:} да.
\par
Пусть нам дан треугольник $ABC$.
Не умаляя общности обозначим длины его сторон $BC$, $AC$ и $AB$ через
$a \geq b \geq c$ соответственно.
Отложим на луче $AB$ точку $D$ такую, что $AD = AC = b$.
Тогда треугольник $CAD$ равнобедренный, а следовательно, имеет ось симметрии
(серединный перпендикуляр к основанию).
Проведем из точки $C$ высоту $CH$ на сторону $AB$ и обозначим её длину через
$h$.
Тогда $S_{ABC} = (h \cdot c) / 2$, а $S_{CAD} = (h \cdot b) / 2$.
То есть
\[
    \dfrac{S_{CAD}}{S_{ABC}}
=
    \dfrac{h \cdot b}{2} \cdot \dfrac{2}{h \cdot c}
=
    \dfrac{b}{c}
.\]
Аналогично можно построить равнобедренный треугольник с вершиной в точке $C$ и
двумя сторонами, равными $a$, и площадью, превосходящей $S_{ABC}$ в
$a / b$ раз.
Нам осталось показать, что либо $a / b \leq 5 / 3$, либо
$b / c \leq 5 / 3$.
Предположим противное.
Тогда $\frac{5}{3} b < a$ и $c < \frac{3}{5} b$.
Отсюда $\frac{5}{3} b < a < b + c < b + \frac{3}{5} b = \frac{8}{5} b$.
Получаем противоречие так, как $\frac{8}{5} < \frac{5}{3}$.
\endproblem
