\problem
Дано натуральное число $k > 1$.
Если $a$, $b$, $c$ натуральные числа и
$a$ делит $b^k$, $b$ делит $c^k$, $c$ делит $a^k$,
при каком наименьшем $n = n(k)$ можно утверждать наверняка, что
$a b c$ делит $(a + b + c)^n$?
%Norway99 upgrade
\solution
\emph{Ответ:} $n(k) = k^2 + k + 1$.
\par
Пусть $p$~--- некоторый простое число, которое входит в $a$, $b$, $c$
соответственно в степенях $\alpha$, $\beta$, $\gamma$.
Тогда
$\alpha \leq k \beta$, $\beta \leq k \gamma$, $\gamma \leq k \alpha$.
Отсюда получаем $\alpha \leq k^2 \gamma$.
Без ограничения общности считаем, что $\gamma$ не больше $\alpha$ и $\beta$.
Тогда $p$ входит в $a + b + c$ хотя бы в степени $\gamma$, а в $a b c$ входит в
степени не более $(k^2 + k + 1) \gamma$.
Следовательно, достаточно $n = k^2 + k + 1$.
Оценку снизу дает набор $a = 2^{k^2}$, $b = 2^k$, $c = 2$.
\endproblem
