\problem
В~школе учатся $400$ человек, из~них $200$ двоечников и~$200$ отличников.
На~Новый Год Дед Мороз привез мешок, в~котором есть 800 конфет
<<Миндаль Иванович>>.
Он~хочет раздать их~все детям, причем каждый двоечник должен получить не~больше
одной конфеты, а~каждый отличник~--- хотя~бы 2~конфеты, причем четное
количество.
Директор школы решил, кроме того, поощрить отличников, приготовив
600 мандаринов, которые хочет раздать так, чтобы каждому отличнику досталось
не~менее одного.
У~кого больше способов раздать свои подарки и~во~сколько раз?
\solution
\emph{Ответ:} способов поровну.
\par
Построим взаимно-однозначное соответствие.
Пусть каждый отличник возьмет шефство над одним из~двоечников
(разные отличники над разными двоечниками).
Если директор дает $2 k + 1$ мандаринов некому отличнику, Дед Мороз может дать
ему $2 k + 2$ конфеты, а~его подшефному 0~конфет.
Если~же директор дает отличнику $2 k$ мандаринов, пусть Дед Мороз даст его
подшефному 1~конфету, а~ему самому $2 k$~конфет. 
\endproblem
% $problem-source: фольклор
