\problem
На~столе стоит $17$ стаканов компота, наполненных в~разной степени.
Общий объем сухофруктов составляет $10\%$ от~всего компота.
Петя и~Вася выбирают и~выпивают стаканы по~очереди (начинает Петя), пока
не~выпьют всё.
Докажите, что Петя всегда может добиться, чтобы в~выпитом им компоте доля
сухофруктов отличалась от~$10\%$ не~больше, чем доля сухофруктов у~Васи.
\solution
Пусть Петя каждым ходом выпивает стакан, где больше всего компота.
Вася каждым ответным ходом будет выпивать не~больше компота.
Кроме того, Петя выпьет лишний стакан, поэтому в~сумме выпьет больше Васи.
В~результате Петя выпьет объем $A$, а~Вася~--- объем $B$, где $A>B$.
Общий объем сухофруктов равен $0.1 (A + B) = 0.1 A + 0.1 B$.
Пусть объем сухофруктов у~Пети равен $0.1 A + x$ (где $x$ может быть
и~отрицательным).
Тогда объем сухофруктов у~Васи равен $0.1 B - x$.
Доля сухофруктов у~Пети равна $(0.1 A + x) / A = 0.1 + x / A$.
Она отличается от~$10\%$ на~$|x| / A$.
Аналогично, отличие доли Васи от~$10\%$ равно $|x| / B$.
Это не~меньше чем у~Пети, так как $A > B$.
\endproblem
% $problem-source: А.\,Шаповалов
