\problem
Two circles $\omega$ and $\gamma$ have the same center, and $\gamma$ lies
inside $\omega$.
Let $O$ be an arbitrary point on $\omega$.
$OA$ and $OB$ are the tangent lines through $O$ to $\gamma$.
Circle with center $O$ and radius $OA$ meets $\omega$ at points $C$ and $D$.
Prove that the line $CD$ contains the middle line of the triangle $OAB$.
\solution
Circumference with center $O$ and radius $OA$ denote by $\Omega$. It is enough to prove that the center of segment $OA$ has the same powers with respect to $\omega$ and $\Omega$. If so the midline of $AOB$ is a radical axis of this two circles and though passes through the intersection points. So let $M$ be a midpoint of segment $AO$, and let $F$ be a second point of intersection of line $AO$ circumference $\omega$. From the symmetry one get $AF=AO=2MO$. Consequently the power of point $M$ with respect to $\omega$ equal to $-OM\cdot FM=-3OM^2$. The power of $M$ with respect to $\Omega$ let us calculate as a difference between square of the distance to the center and square of the radius: $MO^2-AO^2=MO^2-4MO^2=-3AO^2$. We obtained the required.
\endproblem
% $problem-source: Ф.\,Ивлев
